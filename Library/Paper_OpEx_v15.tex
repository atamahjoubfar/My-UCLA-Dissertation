%%%%%%%%%%%%%%%%%%%%%%% preamble %%%%%%%%%%%%%%%%%%%%%%%%%%%
\documentclass[10pt,letterpaper]{article}
\usepackage{opex3}

% \usepackage{ae} %%for Computer Modern fonts

%%%%%%%%%%%%%%%%%%%%%%% begin %%%%%%%%%%%%%%%%%%%%%%%%%%%%%%
\begin{document}

%%%%%%%%%%%%%%%%%% title page information %%%%%%%%%%%%%%%%%%
\title{Analysis of enhanced sensitivity\\ in Raman-amplified photodetection\\ for biomedical sensing and imaging}

\author{Ata Mahjoubfar,$^{1,2}$ Keisuke Goda,$^{1,2}$ Gary Betts,$^{3}$ and Bahram Jalali$^{1,2}$}

\address{$^{1}$Photonics Laboratory, Electrical Engineering Department, University of California, \\Los Angeles, California 90095, USA

$^{2}$California NanoSystems Institute, Los Angeles, California 90095, USA

$^{3}$, , California ?????, USA}

\email{ata@ee.ucla.edu} %% email address is required
\homepage{http://www.photonics.ucla.edu} %% author's URL, if desired

%%%%%%%%%%%%%%%%%%% abstract and OCIS codes %%%%%%%%%%%%%%%%
%% [use \begin{abstract*}...\end{abstract*} if exempt from copyright]

\begin{abstract}
Optical sensing and imaging methods for biomedical applications such as spectroscopy and laser-scanning fluorescence microscopy are incapable of performing sensitive detection at high scan rates due to the fundamental trade-off between sensitivity and speed. This is because fewer photons are detected during short integration times and hence the signal is buried below detector noise. Optical pre-amplification can, however, overcome this challenge by amplifying the optical signal before photodetection. Here we theoretically analyze the sensitivity of high-speed detection systems enhanced by optical pre-amplifiers based on stimulated Raman scattering. We show that the Raman amplifier can achieve a sensitivity improvement of about 20 dB in the visible - near infrared spectral range without sacrificing scan rates. This analysis is expected to be valuable for design of optical pre-amplifiers in biomedical sensing and imaging applications.
\end{abstract}

\ocis{(120.1880) Detection; (060.2320) Fiber optics amplifiers and oscillators; (170.5810) Scanning microscopy; (190.5650) Raman effect; (280.1415) Biological sensing and sensors.} 

%%%%%%%%%%%%%%%%%%%%%%% References %%%%%%%%%%%%%%%%%%%%%%%%%
\begin{thebibliography}{99}

\bibitem{ohki2005nature} K. Ohki, S. Chung, Y. H. Ch'ng, P. Kara, and R. C. Reid, ``Functional imaging with cellular resolution reveals precise micro-architecture in visual cortex,'' \nat {\bf 433}, 597-603 (2005).

\bibitem{golshani2009jneurosci} P. Golshani, J. T. Goncalves, S. Khoshkhoo, R. Mostany, S. Smirnakis, and C. Portera-Cailliau, ``Internally mediated developmental desynchronization of neocortical network activity,'' J. Neurosci. {\bf 29}, 10890-10899 (2009).

\bibitem{slade2002bantam} S. G. Slade, R. Baker, D. K. Brockman, and S. P. Thornton, {\it The complete book of laser eye surgery} (Sourcebooks, Naperville, Illinois, 2002).

\bibitem{delius2002eursurgres} M. Delius, ``Twenty years of shock wave research at the Institute for Surgical Research,'' Eur. Surg. Res. {\bf 34}, 30-36 (2002).

\bibitem{riehle1987churchill} R. A. Riehle Jr., {\it Principles of extracorporeal shock wave lithotripsy} (Churchill Livingston, New York, 1987).

\bibitem{goda2009nature} K. Goda, K. K. Tsia, and B. Jalali, ``Serial time-encoded amplified imaging for real-time observation of fast dynamic phenomena,'' \nat {\bf 458}, 1145-1149 (2009).

\bibitem{squires2005revmodphys} T. M. Squires and S. R. Quake, ``Microfluidics: fluid physics at the nanoliter scale,'' \rmp {\bf 77}, 977-1026 (2005).

\bibitem{watson2004cambridgeunivpress} J. V. Watson, {\it Introduction to flow cytometry} (Cambridge Univ. Press, Cambridge, United Kingdom, 2004).

\bibitem{hult2007optexpress} H. Hult, R. S. Watt, and C. F. Kaminski, ``High bandwidth absorption spectroscopy with a dispersed supercontinuum source,'' \opex {\bf 15}, 11385-11395 (2007).

\bibitem{epstein1998oxfordunivpress} I. R. Epstein and J. A. Pojman, {\it An introduction to nonlinear chemical dynamics} (Oxford Univ. Press, New York, 1998).

\bibitem{petty2004biosystems} H. R. Petty, ``Spatiotemporal chemical dynamics in living cells: from information trafficking to cell physiology,'' Biosystems {\bf 83}, 217-224 (2004).

\bibitem{siesler2002wiley} H. W. Siesler, Y. Ozaki, S. Kawata, and H. M. Heise, {\it Near-infrared spectroscopy: Principles, instruments, applications} (Wiley, Weinheim, Germany, 2002).

\bibitem{diaspro2001wiley} A. Diaspro, {\it Confocal and two-photon microscopy: Foundations, applicaitons and advances} (Wiley, Hoboken, New Jersey, 2001).

\bibitem{goda2008applphyslett} K. Goda, K. K. Tsia, and B. Jalali, ``Amplified dispersive Fourier-transform imaging for ultrafast displacement sensing and barcode reading,'' Appl. Phys. Lett. {\bf 93}, 131109 (2008).

\bibitem{goda2008physreva} K. Goda, D. R. Solli, K. K. Tsia, and B. Jalali, ``Theory of amplified dispersive Fourier transformation,'' Phys. Rev. A {\bf 80}, 043821 (2009). 

\bibitem{tsia2010optexpress} K. K. Tsia, K. Goda, D. Capewell, and B. Jalali, ``Performance of serial time-encoded amplified microscope,'' Opt. Express {\bf 18}, 10016-10028 (2010).

\bibitem{horowitz1989cambridgeunivpress} P. Horowitz and W. Hill, {\it The art of electronics} (Cambridge Univ. Press, Cambridge, United Kingdom, 1989).

\bibitem{agrawal2006academic} G. P. Agrawal, {\it Nonlinear fiber optics} (Academic, Burlington, Massachusetts, 2006).

\bibitem{islam2002ieee} M. N. Islam, ``Raman amplifiers for telecommunications,'' IEEE J. Sel. Top. Quantum Electron. {\bf 8}, 548-559 (2002).

\bibitem{goda2009applphyslett} K. Goda, A. Mahjoubfar, and B. Jalali, ``Demonstration of Raman gain at 800 nm in single-mode fiber and its potential application to biological sensing and imaging,'' \apl {\bf 95}, 251101 (2009).

\bibitem{mahjoubfar2010cleo} A. Mahjoubfar, K. Goda, and B. Jalali, ``Raman amplification at 800 nm in single-mode fiber for biological sensing and imaging,'' in {\it Conference on lasers and electro-optics (CLEO),} (San Jose, Calif., 2010), CFA4.

\bibitem{agrawal2002wiley} G. P. Agrawal, {\it Fiber-optic communication systems} (Wiley, New York, 2002).

\bibitem{kim2002ieeephotonictechl} C. H. Kim, J. Bromage, and R. M. Jopson, ``Reflection-induced penalty in Raman amplified systems,'' IEEE Photonic. Tech. L. {\bf 14,} 573--575 (2002).

\bibitem{islam2004springer} M. N. Islam, {\it Raman Amplifiers for Telecommunications} (Springer, New York, 2004).

\bibitem{lewis2000ieeephotonictechl} S. A. E. Lewis, S. V. Chernikov, and J. R. Taylor, ``Characterization of double Rayleigh scatter noise in Raman amplifiers,'' IEEE Photonic. Tech. L. {\bf 12,} 528--530 (2000).

\bibitem{hansen1998ieeephotonictechl} P. B. Hansen, L. Eskildsen, A. J. Stentz, T. A. Strasser, J. Judkins, J. J. DeMarco, R. Pedrazzani, and D. J. DiGiovanni, ``Rayleigh scattering limitations in distributed Raman pre-amplifiers,'' IEEE Photonic. Tech. L. {\bf 10,} 159--161 (1998).

\bibitem{nissov1999electronlett} M. Nissov, K. Rottwitt, H. D. Kidorf, and M. X. Ma, ``Rayleigh crosstalk in long cascades of distributed unsaturated Raman amplifiers,'' Electron. Lett. {\bf 35,} 997--998 (1999).

\bibitem{headley2005elsevieracademicpress} C. Headley and G. P. Agrawal, {\it Raman amplification in fiber optical communication systems} (Elsevier Academic Press, Burlington, Massachusetts, 2005).

\bibitem{fludger2001jlightwavetechnol} C. R. S. Fludger, V. Handerek, and R. J. Mears, ``Pump to signal RIN transfer in Raman fiber amplifiers,'' J. Lightwave Technol. {\bf 19,} 1140--1148 (2001).

\bibitem{agrawal2001academic} G. P. Agrawal, {\it Applications of nonlinear fiber optics} (Academic Press, San Diego, California, 2001).

\bibitem{mochizuki1986jlightwavetechnol} K. Mochizuki, N. Edagawa, and Y. Iwamoto, ``Amplified spontaneous Raman scattering in fiber Raman amplifiers,'' J. Lightwave Technol. {\bf 4,} 1328--1333 (1986).

\bibitem{smith1972appliedoptics} R. G. Smith, ``Optical power handling capacity of low loss optical fibers as determined by stimulated Raman and Brillouin scattering,'' \ao {\bf 11,} 2489--2494 (1972).

\bibitem{kogelnik1964proceedieee} H. Kogelnik and A. Yariv, ``Considerations of noise and schemes for its reduction in laser amplifiers,'' in {\it Proceedings of IEEE Conference on Electron Device Research} (Institute of Electrical and Electronics Engineers, New York, 1964), pp. 165--172.

\bibitem{rottwitt2003jlightwavetechnol} K. Rottwitt, J. Bromage, A. J. Stentz, L. Leng, M. E. Lines, and H. Smith, ``Scaling of the Raman gain coefficient: applications to Germanosilicate fibers,'' J. Lightwave Technol. {\bf 21,} 1652--1662 (2003).

\end{thebibliography}

%%%%%%%%%%%%%%%%%%%%%%%%%%  body  %%%%%%%%%%%%%%%%%%%%%%%%%%
\section{Introduction}

Fast real-time optical sensing and imaging are essential tools for studying fast transient processes in biomedical applications such as neuroscience \cite{ohki2005nature,golshani2009jneurosci}, laser surgery \cite{slade2002bantam}, and extracorporeal shockwaves \cite{delius2002eursurgres,riehle1987churchill}. They are equally important for microfluidic biotechnology applications (e.g., flow cytometry) \cite{goda2009nature,squires2005revmodphys,watson2004cambridgeunivpress} that require high-throughput analysis of a large population of cells and pathogens with minimum error. Another important application is Raman spectroscopy in which high-speed sensing capability is needed to investigate fast chemical changes such as those that occur during combustion \cite{hult2007optexpress} and chemical reactions between biological molecules \cite{epstein1998oxfordunivpress,petty2004biosystems,siesler2002wiley}. Some examples of applications that require both high scan rates and high detection sensitivity are laser-scanning confocal microscopy, fluorescence microscopy, and two-photon microscopy used to observe neural activity in real time \cite{diaspro2001wiley}. Here even higher throughput is desirable to monitor the dynamics of a large number of neurons simultaneously. 

The central requirement for these studies is a signal integration time that is much shorter than the time scale of dynamic processes. This requirement must be satisfied, but is difficult to meet due to the fundamental trade-off between sensitivity and speed; at high scan rates, fewer photons are detected during the short integration time within each scan period. This leads to the loss of sensitivity in high-speed detection as it is limited by detector noise (typically thermal noise in the photodetector). 

The reduced sensitivity can be compensated for by the use of high-intensity illumination, which, in fact, is frequently used in industrial applications in the form of shadowgraph imaging. However, this approach is not suitable for biomedical applications as it can damage the biological sample. This problem becomes mcuh more severe in microscopy because focusing the light onto the sample with an objective lens results in an etremely high intensity. Another technique used to reduce detector thermal noise is the use of cooling. However, this is undesirable as it requires a refrigeration unit to accompany the detector and hence adds complexity to the detector design. 

Optical amplification before photon-to-electron conversion can circumvent this fundamental challenge when the detection sensitivity is limited by detector noise \cite{goda2009nature, goda2008applphyslett, goda2008physreva, tsia2010optexpress}. It eliminates the need for high-intensity illumination and cooling, representing a powerful method for detecting weak optical signals in biomedical sensing and imaging applications. As shown in Fig. \ref{fig1}, an optical pre-amplifier before the photodetector increases the optical signal while the detector noise is intact, resulting in an improvement in the sensitivity of the detection system. 

\begin{figure}[t]
\centering\includegraphics[width=12cm]{fig1.eps}
\caption{Detector sensitivity enhanced by an optical pre-amplifier in high-speed detection. (a) The sensitivity of the detector is limited by thermal noise. (b) An electrical amplifier does not improve the detector sensitivity as it amplifies both the signal and thermal noise. (c) An optical pre-amplifier before the photodetector increases the optical signal while the thermal noise is intact, resulting in an improvement in the sensitivity of the detection system.}
\label{fig1}
\end{figure}

Optical amplification is fundamentally different from electronic signal intensifiers [e.g. micro-channel plates used in photomultiplier tubes (PMTs)] in that amplification occurs in the optical domain whereas in micro-channel plates, it occurs in the electronic domain. Electronic signal intensifiers are complex vacuum tube devices that require high voltage sources. Also, their scan rates are limited by the fundamental trade-off between gain and bandwidth in all electronic systems \cite{horowitz1989cambridgeunivpress}. While PMTs are extremely sensitive detectors useful for capturing faint light or even counting single photons, they are unsuitable for continuous high-speed detection due to the dead time caused by their gated operation. 

Among different approaches to optical amplification, stimulated Raman scattering (SRS) \cite{agrawal2006academic,islam2002ieee} provides several advantages over other methods such as rare-earth doped fiber amplifiers and semiconductor optical amplifiers (SOAs). First, gain is possible at any wavelength as long as a pump is available at a frequency blue-shifted from the signal by the optical-phonon vibrational frequency \cite{agrawal2006academic}. Second, a broad and flexible gain spectrum can be obtained by using multiple pump fields \cite{islam2002ieee}. Finally, Raman amplifiers have a lower noise figure than rare-earth doped fiber amplifiers and SOAs \cite{goda2009applphyslett,islam2002ieee}. For these reasons, Raman amplifiers are routinely employed in fiber-optic communication \cite{islam2002ieee}. 

Recently we have reported Raman amplification of a weak optical signal in a single-mode fiber in the 800 nm spectral range for the first time \cite{goda2009applphyslett,mahjoubfar2010cleo}. This proof-of-principle demonstration has shown potential utility of Raman amplifiers to biomedical sensing and imaging applications. While noise sources associated with Raman amplifiers have been extensively studied in the fiber-optic communication band (1300 - 1600 nm), virtually no such work has been done in the optical spectrum outside of the fiber-optic communication band. However, the short wavelength band (500 - 900 nm) is important for biomedical applications as most of them are conducted in the band in favor of low water absorption and high spatial resolution in imaging. 

In this paper, we theoretically study the sensitivity of a detection system enhanced by a Raman amplifier as an optical pre-amplifier at such short wavelengths. First, we review the basic concept of the Raman amplifier in Section \ref{raman_amplifier}. We then derive the variances of noise photocurrents associated with the Raman amplifier as well as photodetection in Section \ref{noise_sources}. With the results in the last section, we evaluate and compare the noise photocurrent variances and then study the sensitivity of the detection system enhanced by the amplifier in Section \ref{sensitivity_analysis}. We also perform numerical simulations to obtain a set of parameter values that optimize the performance of the detection system at various wavelengths. In Section \ref{conclusion}, we conclude this paper.

%----------------------------------------------------------------------
\section{Raman amplification as an optical pre-amplifier}
\label{raman_amplifier}
Raman amplification is an optical process based on the phenomenon of SRS \cite{agrawal2006academic,islam2002ieee}. As shown in Fig. \ref{fig2}, in SRS an input field (called the Stokes field) stimulates the inelastic scattering of a blue-shifted pump field inside an optical medium mediated by its vibrational modes (called optical phonons). As with rare-earth doped fiber amplifiers, Raman amplifiers based on the process of Raman amplification are often used to increase optical signals in fiber-optic communications with transmission fibers as gain media \cite{agrawal2006academic,islam2002ieee}.  

\begin{figure}[t]
\centering\includegraphics[width=12cm]{fig2.eps}
\caption{Energy diagram of stimulated Raman scattering (SRS) and basic schematic of a SRS-based optical pre-amplifier in an optical fiber as a gain medium. (a) Raman amplification is an optical process based on the phenomenon of SRS in which an input field (called the Stokes field) stimulates the inelastic scattering of a blue-shifted pump field inside an optical medium mediated by its vibrational modes (called optical phonons). (b) A typical Raman amplifier consists of a single-mode silica fiber as a gain medium, an input weak field (Stokes field), and one or two pump fields that couple into and out of the fiber via duplexers such as wavelength-division multiplexers or dichroic beamsplitters. In many cases, the fiber is bidirectionally pumped in the forward and backward directions to optimize the performance of the amplifier.}
\label{fig2}
\end{figure}

A typical Raman amplifier is schematically shown in Fig. \ref{fig2}. It consists of a single-mode silica fiber as a gain medium, an input weak field (Stokes field), and one or two pump fields that couple into and out of the fiber via duplexers such as fiber-based wavelength-division multiplexers or dichroic beamsplitters in free space. In many cases, the fiber is bidirectionally pumped in the forward and backward directions to optimize the performance of the amplifier. 

Raman amplification in an optical fiber is governed by the following coupled equations:
\begin{eqnarray}
\frac{dI_s}{dz} &=& g_R I_p I_s - \alpha_s I_s, \label{eq:CoupledStokes} \\
\frac{dI_p}{dz} &=& -\frac{\omega_p}{\omega_s} g_R I_s I_s- \alpha_p I_p, \label{eq:CoupledPump}
\end{eqnarray}
where $I_s$ and $I_p$ are the intensities of the Stokes and pump fields, respectively, $\omega_s$ and $\omega_p$ are the optical frequencies of the Stokes and pump fields, respectively, $\alpha_s$ and $\alpha_p$ are the loss coefficients of the fiber at the Stokes and pump frequencies, respectively, $g_R$ is the Raman gain coefficient, and $z$ is the propagation distance. 

Assuming that the pump field is much more powerful than the Stokes field (which is often the case) and hence undepleted, the first term on the right hand side in Eq. \ref{eq:CoupledPump} can be ignored and the pump intensity decreases exponentially in the fiber. If the fiber is bidirectionally pumped, the total pump intensity at an arbitrary position in the fiber can be obtained from Eq. \ref{eq:CoupledPump} and is given by 
\begin{equation}
I_p(z) = I_{p}^f(z) + I_{p}^b(z), 
\label{eq:PumpIntensity}
\end{equation}
where 
\begin{equation}
I_p^f(z) = I_{p}^f(0) \mathrm{e}^{-\alpha_p z}, \hspace{1cm}
I_p^b(z) = I_{p}^b(L) \mathrm{e}^{-\alpha_p (L-z)}. 
\end{equation}
Here $L$ is the total length of the fiber and $I_{p}^f(0)$ and $I_{p}^b(L)$ are the intensities of the forward and backward pump fields before entering the fiber, respectively, which can be obtained from the pump powers in the forward and backward directions,
\begin{equation}
I_{p}^{f}(0) = \frac{P_{p}^{f}(0)}{\pi(d_p/2)^2}, \hspace{1cm}
I_{p}^{b}(L) = \frac{P_{p}^{b}(L)}{\pi(d_p/2)^2},
\end{equation}
where $P_{p}^{f}(0)$ and $P_{p}^{b}(L)$ are the forward and backward pump powers, respectively, and $d_p$ is the mode field diameter of the pump fields in the fiber. Likewise, the Stokes intensity can also be obtained from the Stokes power,
\begin{equation}
I_s(z) = \frac{P_s(z)}{\pi(d_s/2)^2},
\end{equation}
where $d_s$ is the mode field diameter of the Stokes field in the fiber. Here we have assumed that the forward and backward pump fields are uncorrelated and do not interfere with each other in the fiber. 

Substituting Eq. \ref{eq:PumpIntensity} into Eq. \ref{eq:CoupledStokes}, the intensity of the Stokes field at an arbitrary point in the fiber, $I_s(z)$, can be obtained, and hence, the net gain is also found to be
\begin{equation}
G(z)=\frac{I_s(z)}{I_{s}(0)}=\exp\left\{\frac{g_R}{\alpha_p} \left[I_{p}^f(0)\left(1-\mathrm{e}^{-\alpha_p z}\right)+I_{p}^b(L)\left(\mathrm{e}^{-\alpha_p (L-z)}-\mathrm{e}^{-\alpha_p L}\right)\right]- \alpha_s z\right\}, \label{eq:Gain}
\end{equation}
where $I_s(0)$ is the intensity of the input Stokes field before entering the fiber. 

%----------------------------------------------------------------------
\section{Limiting noise sources}
\label{noise_sources}

In order to evaluate the performance of the Raman amplifier, or more specifically, to study the sensitivity of a detection system enhanced by the amplifier, it is important to understand limiting noise sources assocaited with the amplifier. In this section, we derive analytical forms for such noise components. We then evaluate and compare them using realistic parameter values in the next section. 

%----------------------------------------------------------------------
\subsection{Double Rayleigh backscattering noise}
Rayleigh scattering is the elastic scattering of light by particles (i.e., atoms and molecules) with dimensions which are much smaller than the wavelength of the light. Rayleigh scattering by molecules in the fiber medium (typically glass) is a fundamental loss mechanism in all optical fibers. While most of the scattered light escapes through the cladding, a small portion of the light can stay in the fiber core and propagate backward in the fiber. This backward-propagating scattered light is normally very weak, but it can be amplified by a distributed Raman amplifier in the fiber medium. The amplified backward-propagating scattered light can again be scattered and propagate along with the signal in the forward propagation direction. This causes a delayed crosstalk in-band noise that is amplified by the process of SRS in the fiber \cite{agrawal2002wiley,kim2002ieeephotonictechl}. This is called double Rayleigh backscattering (DRB) noise \cite{islam2004springer,lewis2000ieeephotonictechl} and is known to play a critical role in limiting the performance of Raman amplifiers in long-haul fiber-optic communication systems \cite{hansen1998ieeephotonictechl}. 

The crosstalk-to-signal ratio or fraction of the Stokes power that is scattered by the process of DRB and amplified by the distributed Raman amplifier throughput the fiber is given by \cite{nissov1999electronlett}
\begin{equation}
f_{\rm DRB}=r_s^2 \int_0^L{G^{-2}(z) \int_z^L{G^2(z') dz'} dz},
\end{equation}
where $r_s$ is the Rayleigh baskscattering coefficient. Since this is proportional to the Rayleigh scattering loss, it is inversely proportional to $\lambda_s^4$, where $\lambda_s$ is the wavelength of the Stokes field. Therefore, the DRB noise is an important limiting noise factor at shorter wavelengths than in the fiber-optic communication band. 

The variance of the DRB noise photocurrent produced by the photodiode is given by \cite{headley2005elsevieracademicpress}
\begin{equation}
\left\langle \delta i_{\rm DRB}^2\right\rangle = 2 f_{\rm DRB} [\rho G(L) P_s(0)]^2,
\label{eq:DRB}
\end{equation}
where $\rho$ is the responsivity of the photodetector and $P_s(0)$ is the input Stokes power. The factor of 2 in the equation accounts for the two polarization modes of the fiber. 

%----------------------------------------------------------------------
\subsection{Pump-to-Stokes relative intensity noise transfer}
Every laser including pump lasers in Raman amplifiers has intensity fluctuations called relative intensity noise (RIN) which typically originate from laser cavity vibration, instability in the laser gain medium, or transferred intensity noise from a pump source. As the Raman amplifier is powered by one or more pump fields, their RIN can transfer to the Stokes field through the process of SRS. In fiber-optic communication, RIN is an important noise factor at low frequencies while less important at high frequencies due to the sharp roll-off at frequencies above 10 MHz. 

Using the RIN calculation technique in the case of a singly-pumped Raman amplifier in Ref. \cite{fludger2001jlightwavetechnol}, the transfer function of the pump RIN to the Stokes RIN in bidirectionally pumped Raman amplifiers is found to be
\begin{eqnarray}
&&\hspace{-0.7cm}RIN_s(f) = \nonumber\\
&&\hspace{-0.5cm}\left\{\frac{v_s\left[\alpha_s L + \ln G(L)\right]}{L_{\rm eff}}\right\}^2  
 \left\{RIN_p^f(f) \left[\frac{P_p^f(0)}{P_p^f(0)+P_p^b(L)}\right]^2 \left[\frac{1-2\mathrm{e}^{-\alpha_p L}\cos(bfL/v_s)+\mathrm{e}^{-2 \alpha_p L}}{(\alpha_p v_s)^2+(b f)^2}\right] \right.\nonumber\\
&&\hspace{-0.5cm} \left.+RIN_p^b(f) \left[\frac{P_p^b(L)}{P_p^f(0)+P_p^b(L)}\right]^2 \left[\frac{1-2\mathrm{e}^{-\alpha_p L}\cos(4 \pi f L/v_s)+\mathrm{e}^{-2 \alpha_p L}}{(\alpha_p v_s)^2+(4 \pi f)^2}\right]\right\}, 
\label{eq:RINtf}
\end{eqnarray}
where $f$ is the detection frequency (also called the sideband frequency with respect to the carrier frequency), $RIN_p^f(f)$ and $RIN_p^b(f)$ are the RIN coefficient of the forward and backward pump fields, respectively, $v_s$ is the group velocity of the Stokes field, $L_{\rm eff}$ is the effective length of the fiber at the pump wavelength and is given by
\begin{equation}
L_{\rm eff}=\frac{1-\exp(-\alpha_p L)}{\alpha_p},
\end{equation}
and a constant, $b$, is defined in terms of the difference in group velocity between the pump and Stokes fields by  
\begin{equation}
b=2\pi\left(1-\frac{v_s}{v_p}\right).
\end{equation}
Here $v_p$ is the group velocity of the pump fields. The first and second terms in Eq. \ref{eq:RINtf} account for the RIN transfer from the forward and backward pump RINs to the Stokes RIN, respectively. 

The variance of the RIN transfer noise photocurrent produced by the photodiode is found by integrating the beat between the Stokes field and the pump-RIN-transferred Stokes noise over the electrical bandwidth of the photodetector, $B_e$, to be
\begin{equation}
\left\langle \delta i_{\rm RIN}^2\right\rangle=\int_0^{B_e}RIN_s(f) [\rho G(L) P_s(0)]^2 df.
\label{eq:RIN}
\end{equation}

%----------------------------------------------------------------------
\subsection{Amplified spontaneous emission noise}
In a laser medium, the luminescence from spontaneous emission can be amplified by the process of stimulated emission in the gain medium. This process is called amplified spontaneous emission (ASE). These ASE photons have random phases relative to the signal and appear as noise on the signal. ASE is a limiting noise source in rare-earth doped fiber amplifiers such as erbium-doped fiber amplifiers and directly affects their noise figure in fiber-optic communication \cite{agrawal2001academic}. The effect of ASE also occurs in the process of SRS, producing amplified spontaneous Raman emission noise \cite{mochizuki1986jlightwavetechnol,smith1972appliedoptics}. 

The SRS-induced ASE noise gives rise to two beat-noise signals that appear at baseband frequency: (1) beating of the Stokes field with the ASE and (2) beating of the ASE with itself. The variances of these noise photocurrents generated at the photodiode are given by \cite{headley2005elsevieracademicpress}
\begin{eqnarray}
\label{eq:sASE}
\left\langle \delta i_{\rm s-ASE}^2\right\rangle &=& 4 \rho^2 G(L) P_s(0) S_{\rm ASE} B_e, \\
\label{eq:ASEASE}
\left\langle \delta i_{\rm ASE-ASE}^2\right\rangle &=& 4 \rho^2 S_{\rm ASE}^2 B_e \left(B_o-\frac{B_e}{2}\right),
\end{eqnarray}
where $B_o$ is the optical bandwidth of the photodetector, $B_e$ is the electrical bandwidth of the detection system (assumed to be $<B_o/2$ in these formulae), and $S_{\rm ASE}$ is the spectral desnity of the ASE defined by \cite{kogelnik1964proceedieee}
\begin{equation}
S_{\rm ASE} = n_{\rm sp} \hbar \omega_s g_R G(L) \int_0^L {\frac{I_p(z)}{G(z)} dz}.
\label{eq:ASE}
\end{equation}
Here $\hbar$ is the Planck constant and $n_{\rm sp}$ is the spontaneous-scattering factor given by
\begin{equation}
n_{\rm sp}=\left\{1-\exp\left[-\frac{\hbar (\omega_p-\omega_s)}{k_B T}\right]\right\}^{-1},
\end{equation}
where $k_B$ is the Boltzmann constant and $T$ is the temperature of the gain medium. In Eqs. \ref{eq:sASE} and \ref{eq:ASEASE}, the factor of 4 comes from the beating of two fields having two polarization modes. Eq. \ref{eq:ASE} indicates that $S_{\rm ASE}$ is white and exists at all frequencies within the gain spectrum of the Raman amplifier. Hence, the effect of the ASE on the Stokes field can be reduced by narrowing the optical bandwidth with an optical filter. 

%----------------------------------------------------------------------
\subsection{Detection noise}
The performance of a photodetector system is limited by thermal noise from thermal fluctuations in the detector resistance, shot noise from the photocurrent, dark current noise from the photodiode, and flicker noise. The flicker noise (also called $1/f$ noise) is inversely proportional to the measurement frequency and hence ignored in high-speed detection. 

Thermal noise (also called Johnson-Nyquist noise) is the electronic noise produced by the thermal agitation of the electrons inside an electrical conductor (typically the load) at equilibrium. Thermal noise is approximately white, and hence, its power spectral density is constant throughout the entire detection bandwidth. The variance of the Norton-equivalent thermal noise photocurrent is given by 
\begin{equation}
\left\langle \delta i_{\rm thermal}^2\right\rangle = \frac{4 k_B T B_e}{R},
\label{eq:thermal}
\end{equation}
where $R$ is the load resistance of the photodetector circuit. 

Shot noise originates from the fluctuations of detected photons and is a direct consequence of the quantum nature of photons. Shot noise follows a Poisson distribution and is translated into the variance of the shot noise photocurrent through the photodiode,
\begin{equation}
\left\langle \delta i_{\rm shot}^2\right\rangle = 2 q \rho G(L) P_s(0) B_e,
\label{eq:shot}
\end{equation}
where $q$ is the elementary charge. Similar to thermal noise, shot noise is also white with constant power spectral density throughout the detection bandwidth. 

Dark current is the relatively small electric current that flows through the photodiode when it is not exposed to light. It is due to the random generation of free carriers within the depletion region of the photodiode. The variance of the dark noise photocurrent through the photodiode is given by
\begin{equation}
\left\langle \delta i_{\rm dark}^2\right\rangle = 2 q \left\langle i_{\rm dark}\right\rangle B_e,
\label{eq:dark}
\end{equation}
where $\left\langle i_{\rm dark}\right\rangle$ is the average dark current through the photodiode. 

%----------------------------------------------------------------------
\section{Sensitivity analysis}
\label{sensitivity_analysis}

In the last section, we have derived the noise photocurrent variance of various noise components that can limit the sensitivity of the detection system enhanced by the Raman amplifier. In this section, we evaluate and compare these variances at 800 nm using realistic parameter values to visualize dominant noise factors, depending on the input optical power to the Raman amplifier. Then, we study the sensitivity of the detection system enhanced by the Raman amplifier at 800 nm. Finally, we investigate sensitivity improvements by the amplifier at wavelengths other than 800 nm. 

\begin{table}[t]
\begin{centering}
\begin{tabular}{|c|c|c|c|} \hline
Name & Parameter & Unit & Value\\ \hline\hline
Planck Constant$/2\pi$ & $\hbar$ & J$\cdot$s & $1.05457 \times 10^{-34}$ \\ \hline
Boltzmann Constant & $k_B$ & J/K & $1.38065 \times 10^{-23}$  \\ \hline
Elementary Charge & $q$ & C & $1.60218 \times 10^{-19}$  \\ \hline 
Stokes Wavelength & $\lambda_s$ & nm & 800 \\ \hline
Pump Wavelength & $\lambda_p$ & nm & 773.2 \\ \hline
Fiber Attenuation at Stokes Wavelength & $\alpha_s$ & m$^{-1}$ & $5.32 \times 10^{-4}$ \\ \hline
Fiber Attenuation at Pump Wavelength & $\alpha_p$ & m$^{-1}$ & $6.12 \times 10^{-4}$ \\ \hline
Raman Gain Coefficient & $g_R$ & m/W & $4.57 \times 10^{-13}$ \\ \hline
Group Velocity at Stokes Wavelength & $v_s$ & m/s & $2.053373 \times 10^8$ \\ \hline
Group Velocity at Pump Wavelength & $v_p$ & m/s & $2.052073 \times 10^8$ \\ \hline
Mode Field Diameter at Stokes Wavelength & $d_s$ & $\mu$m & 5.52 \\ \hline
Mode Field Diameter at Pump Wavelength & $d_p$ & $\mu$m & 5.34 \\ \hline
Fiber Length & $L$ & m & 1040 \\ \hline
Temperature of Raman Gain Medium & $T$ & K & 300 \\ \hline
Rayleigh Backscattering Coefficient & $r_s$ & km$^{-1}$ & $1.41 \times 10^{-3}$ \\ \hline
Forward Input Pump Power & $P_p^f(0)$ & mW & 200 \\ \hline
Backward Input Pump Power & $P_p^b(L)$ & mW & 200 \\ \hline
Forward Pump Relative Intensity Noise & $RIN_p^f$ & Hz$^{-1}$ & $10^{-13}$ \\ \hline
Backward Pump Relative Intensity Noise & $RIN_p^b$ & Hz$^{-1}$ & $10^{-13}$ \\ \hline
Photodiode Responsivity & $\rho$ & A/W & 0.5 \\ \hline
Load Impedance & $R$ & $\Omega$ & 50\\ \hline
Photodiode Dark Current & $\left\langle i_{\rm dark}\right\rangle$ & nA & 20 \\ \hline
Optical Bandwidth of Photodetector & $B_o$ & THz & 0.5 \\ \hline
Electrical Bandwidth of Photodetector & $B_e$ & MHz & 100 \\ \hline
\end{tabular}
\caption{Parameter values used to evaluate and compare the noise power of different noise components associated with the detection system enhanced by the Raman amplifier. These values are used for Figs. \ref{fig3}, \ref{fig4}, and \ref{fig5}.}
\label{table1}
\end{centering}
\end{table}

%----------------------------------------------------------------------
\subsection{Comparison of various noise components}

To evaluate the noise photocurrent variances derived in the last section, we use realistic values for the parameters. Table \ref{table1} shows a summary of such parameter values. 

\begin{figure}[t]
\centering\includegraphics[width=10cm]{fig3.eps}
\caption{Noise photocurrent variance with an electrical bandwidth of 100 MHz vs. the input Stokes power to the optical pre-amplifier: (a) DRB noise, (b) pump-to-Stokes RIN transfer, (c) ASE-signal beating noise, (d) ASE-ASE beating noise, (e) thermal noise, (f) shot noise, and (g) dark current noise. The ASE-ASE beating noise and thermal noise are dominant at low input Stokes powers while the RIN and DRB noise are significant at high input Stokes powers.}
\label{fig3}
\end{figure}

For the Raman amplifier, we choose the Stokes wavelength to be 800 nm. This is because it is easy to compare our theoretical results with our recent experimental demonstration of Raman amplification which was performed at 808 nm \cite{goda2009applphyslett}. To maximize the Raman gain, the Stokes frequency shift is assumed to be 13 THz, which corresponds to the pump wavelength of 773.2 nm. The Raman gain coefficient at $\lambda_s=800$ nm can be estimated from \cite{rottwitt2003jlightwavetechnol}
\begin{equation}
g_R(\lambda_s)=\frac{\Lambda_s}{\lambda_s}\frac{A_{\rm eff}(\Lambda_s)}{A_{\rm eff}(\lambda_s)}g_R(\Lambda_s),
\end{equation}
where $g_R(\Lambda_s)$ is the Raman gain coefficient at a reference wavelength $\Lambda_s$, and $A_{\rm eff}$ is the average mode field diameter between the pump and Stokes fields and is given by $A_{\rm eff}=[\pi(d_s/2)^2+\pi(d_p/2)^2]/2$ evaluated at each wavelength. Using the well-known Raman gain coefficient of $g_R(\Lambda_s)= 6 \times 10^{-14}$ m/W at $\Lambda_s = 1500$ nm \cite{islam2002ieee}, the Raman gain coefficient is found to be $g_R(\lambda_s) =  4.57 \times 10^{-13}$ m/W. The fiber attenuation of typical fibers at the Stokes and pump wavelengths are 2.31 dB/km and 2.66 dB/km, respectively, which correspond to the fiber loss coefficients of $5.32 \times 10^{-4}$ m$^{-1}$ and $6.13 \times 10^{-4}$ m$^{-1}$, respectively. The input power of the forward and backward pump fields and the fiber length are chosen to be both 200 mW and 1040 m, respectively, to optimize the performance of the Raman amplifier. The temperature of the fiber is assumed to be the same as room temperature (300 K). 

For the DRB noise, the Rayleigh backscattering coefficient, $r_s$, needs to be estimated. Since it is about $10^{-4}$ km$^{-1}$ for fiber-optic transmission fibers in the 1550 nm spectral band, the coefficient at 800 nm can be assumed to be $r_s \simeq (1550/800)^4 \times 10^{-4} = 1.41 \times 10^{-3}$ km$^{-1}$. 

For the pump-to-Stokes RIN transfer, given the typical pump RIN noise of 0.1 $\%$ over a bandwidth of 10 MHz, the RIN of the forward and backward pump fields is found to be $RIN_p^f = RIN_p^b = (0.1~\%)^2/(10 ~{\rm MHz}) = 10^{-13}$ Hz$^{-1}$. The group velocity at the Stokes and pump wavelengths is estimated from the Sellmeier equation to be $2.051968 \times 10^8$ m/s and $2.053373 \times 10^8$ m/s, respectively.

To evaluate the noise photocurrents associated with the photodetection, we assume the responsivity and the dark current of the photodiode from the typical Si photodiode at 800 nm to be 0.5 A/W and 20 nA, respectively. The load impedance of the photodetector circuit is 50 $\Omega$ (standard in radio-frequency electronics). The optical and electrical bandwidths of the photodetector are assumed to be 0.5 THz and 100 MHz, respectively. The narrow optical bandwidth is preferred to reduce the ASE-ASE beating noise and can be implemented by using an optical band-pass filter.

With these values for the parameters in Eqs. \ref{eq:DRB}, \ref{eq:RIN}, \ref{eq:sASE}, \ref{eq:ASEASE}, \ref{eq:thermal}, \ref{eq:shot}, and \ref{eq:dark}, the variance of each noise photocurrent is evaluated and shown in Fig. \ref{fig3}. The dark noise, thermal noise, and ASE-ASE beating noise are independent of the input Stokes power. The ASE-signal beating noise and shot noise are linearly proportional to the input Stokes power whereas the RIN and DRB noise are quadratically proportional to the input Stokes power. Therefore, the ASE-ASE beating noise and thermal noise are dominant at low input Stokes powers while the RIN and DRB noise are significant at high input Stokes powers. 

%----------------------------------------------------------------------
\subsection{Signal-to-noise ratio of the detection system enhanced by the Raman amplifier}

\begin{figure}[t]
\centering\includegraphics[width=10cm]{fig4.eps}
\caption{SNR achievable with various detection methods vs. the input Stokes power: (a) a photodiode without a Raman amplifier, (b) a photodiode with a Raman amplifier, (c) an APD without a Raman amplifier, and (d) an APD with a Raman amplifier. An electrical bandwidth of 100 MHz is used for all the cases. Comparing (a) and (b), the use of the Raman amplifier improves the sensitivity of the photodiode detection system by $P_1/P_2=19.4$ dB. Comparing (b) and (c), the combination of the Raman amplifier and photodiode is better in sensitivity than an APD alone by $P_3/P_2=9.1$ dB.}
\label{fig4}
\end{figure}

The signal-to-noise ratio (SNR) of the detection system is given by the power of the signal photocurrent over the total noise current variance,
\begin{equation}
{\rm SNR} = \frac{i_s^2}{\left\langle \delta i_{\rm total}^2\right\rangle}=\frac{[\rho G(L) P_s(0)]^2}{\left\langle \delta i_{\rm total}^2\right\rangle},
\end{equation}
where the total noise current variance is the sum of all the variances of the different noise currents in Eqs. \ref{eq:DRB}, \ref{eq:RIN}, \ref{eq:sASE}, \ref{eq:ASEASE}, \ref{eq:thermal}, \ref{eq:shot}, and \ref{eq:dark} such that
\begin{eqnarray}
\left\langle \delta i_{\rm total}^2\right\rangle &=&\left\langle \delta i_{\rm DRB}^2\right\rangle+\left\langle \delta i_{\rm RIN}^2\right\rangle+\left\langle \delta i_{\rm s-ASE}^2\right\rangle+\left\langle \delta i_{\rm ASE-ASE}^2\right\rangle \nonumber\\
&&+\left\langle \delta i_{\rm thermal}^2\right\rangle+\left\langle \delta i_{\rm shot}^2\right\rangle+\left\langle \delta i_{\rm dark}^2\right\rangle.
\end{eqnarray}

Fig. \ref{fig4} shows the SNR of the detection system in various detector configurations, indicating an improvement in detection sensitivity by the use of the Raman amplifier before the photodetector. Given an electrical bandwidth of 100 MHz, the Raman amplifier improves the sensitivity of the photodiode detection system by 19.4 dB compared to the detection system without the amplifier. The combination of the Raman amplifier and photodiode is better in SNR than an avalanche photodiode (APD) alone by 9.1 dB. Note that the combination of the Raman amplifier and APD does not improve the sensitivity significantly (only 6.1 dB) because the thermal noise in the APD circuitry is not as high as in the photodiode circuitry.

%----------------------------------------------------------------------
\subsection{Sensitivity improvements by the Raman amplifier at various wavelengths}

\begin{table}[t]
\begin{centering}
\begin{tabular}{|c|c|c|c|c|c|c|c|} \hline
Parameter & Unit & \multicolumn{6}{|c|}{Value} \\ \hline\hline
$\lambda_s$ & nm & 500 & 600 & 700 & 800 & 900 & 1000 \\ \hline
$\lambda_p$ & nm & 489.4 & 584.8 & 679.4 & 773.2 & 866.2 & 958.4 \\ \hline
$d_s$ & $\mu$m & 3.5 & 4.2 & 4.9 & 5.5 & 6.2 & 6.9 \\ \hline
$d_p$ & $\mu$m & 3.4 & 4.1 & 4.7 & 5.3 & 6.0 & 6.6 \\ \hline
$g_R$ & pm/W & 1.79 & 1.06 & 0.674 & 0.457 & 0.324 & 0.238 \\ \hline
$P_p^f(0)$ &mW & 200 & 175 & 175 & 200 & 250 & 400 \\ \hline
$P_p^b(L)$ &mW & 200 & 175 & 175 & 200 & 250 & 400 \\ \hline
$L$ &m & 100 & 270 & 610 & 1040 & 1420 & 1360 \\ \hline
$P_1/P_2$ &dB & 17.6 & 18.2 & 18.8 & 19.4 & 20.0 & 20.6 \\ \hline
$P_3/P_2$ &dB  & 7.3 & 7.9 & 8.5 & 9.1 & 9.7 & 10.3 \\ \hline
$P_3/P_4$ &dB  & 4.2 & 5.5 & 6.1 & 6.1 & 6.7 & 7.3 \\ \hline
$P_1$ &nW  & 533.67 & 533.67 & 533.67 & 533.67 & 533.67 & 533.67 \\ \hline
$P_2$ &nW  & 9.33 & 8.11 & 7.05 & 6.14 & 5.34 & 4.64 \\ \hline
$P_3$ &nW  & 49.77 & 49.77 & 49.77 & 49.77 & 49.77 & 49.77 \\ \hline
$P_4$ &nW  & 1.87 & 1.42 & 1.23 & 1.23 & 1.07 & 9.33 \\ \hline
\end{tabular}
\caption{Sensitivity improvement and minimum detectable optical power (detection sensitivity) in various optical configurations. $P_1$: Detection sensitivity of a photodiode without a Raman amplifier, $P_2$: Detection sensitivity of a photodiode with a Raman amplifier, $P_3$: Detection sensitivity of an APD without a Raman amplifier, $P_4$: Detection sensitivity of an APD with a Raman amplifier, $P_1/P_2$: Sensitivity improvement by the Raman amplifier for the photodiode, $P_3/P_4$: Sensitivity improvement by the Raman amplifier for the APD, $P_3/P_2$: Sensitivity improvement by the Raman amplifier for the photodiode in comparison with the APD without the Raman amplifier. Note that in the amplifier-enhanced detection system, the degree of the sensitivity improvement improves as the Stokes wavelength increases. Also, the detection sensitivity decreases at longer Stokes wavelengths. At 500 nm, a fiber of only 100 m is required to reach the optimum sensitivity improvement. The increase in the required pump power at 500 nm is due to the interplay of the DRB noise, ASE-signal beating noise, and ASE-ASE beating noise.}
\label{table2}
\end{centering}
\end{table}

So far we have focused on the Stokes wavelength of 800 nm because we previously reported the experimental demonstration of Raman gain at this wavelength. Here we extend the analysis performed at 800 nm to other wavelengths in the visible - near infrared spectrum where most biological imaging experiments are conducted. 

Table \ref{table2} shows the results of sensitivity improvements with optimized Raman amplifiers at wavelengths from 500 nm to 1000 nm. It can be seen that at all these wavelengths, the sensitivity of a photodiode with a Raman amplifier, $P_2$, is about two orders of magnitude better than the sensitivity of a photodiode without a Raman amplifier, $P_1$. Note that in the amplifier-enhanced detection system, the degree of the sensitivity improvement improves as the Stokes wavelength increases. Also, the minimum detectable power decreases at longer Stokes wavelengths. Due to the large Raman gain coefficients at shorter wavelengths, pump lasers with much smaller powers are sufficient to reach the optimum sensitivity improvements. 

\begin{figure}[t]
\centering\includegraphics[width=13cm]{fig5.eps}
\caption{Detection enhancement by the Raman amplifier. \textit{Left}: Sensitivity improvements by the Raman amplifier for a photodiode at various Stokes wavelengths. \textit{Right}: Minimum detectable powers of the Stokes field (equivalent to the amplifier-enhanced detector sensitivity) at various Stokes wavelengths. Note that higher sensitivity improvements and lower minimum detectable powers are achievable at lower pump powers with the Stokes field at shorter wavelengths while better optimum sensitivity improvements and minimum detectable powers can be obtained at longer Stokes wavelengths.}
\label{fig5}
\end{figure}

As shown in Fig. \ref{fig5}, further analysis of the sensitivity improvement and minimum detectable power indicates that they reach the optimum levels at certain pump powers beyond which they cannot be improved any more. At these wavelengths, the maximum sensitivity improvement ranges from 17 to 21 dB. The same maximum sensitivity can be achieved with shorter fibers, but requires higher pump powers to maintain the same Raman gain. Note that higher sensitivity improvements and lower minimum detectable powers are achievable at lower pump powers with the Stokes field at shorter wavelengths while better optimum sensitivity improvements and minimum detectable powers can be obtained at longer Stokes wavelengths.

%----------------------------------------------------------------------
\section{Conclusion}
\label{conclusion}
In summary, we have theoretically shown that the use of an optical pre-amplifier based on SRS can improve the sensitivity of a high-speed photodetection system in the visible - near infrared spectral range. To analyze the sensitivity of the detection system with the Raman amplifier, we have derived and compared various noise photocurrents. More specifically, we have demonstrated a sensitivity improvement of about 20 dB in the amplifier-enhanced detection system at Stokes wavelengths from 500 to 1000 nm using a 100-MHz bandwidth which supports very high scan rates. This analysis is expected to be valuable for design of optical pre-amplifiers for high-speed detection in biomedical sensing and imaging applications such as Raman spectroscopy, laser-scanning confocal microscopy, two-photon microscopy, and fluorescence microscopy.

%----------------------------------------------------------------------
%\break

\end{document} 
