\chapter{Analysis of Phase Noise in Phase/Frequency Detectors}

The phase noise of phase/frequency detectors can significantly raise the in-band phase noise of frequency synthesizers, corrupting the modulated
signal. This chapter analyzes the phase noise mechanisms in CMOS phase/frequency detectors and applies the results to two different topologies. It is
shown that an octave increase in the input frequency raises the phase noise by 6 dB if flicker noise is dominant and by 3 dB if white noise is
dominant. An optimization methodology is also proposed that lowers the phase noise by 4 to 8 dB for a given power consumption. Simulation and
analytical results agree to within 3.1 dB for the two topologies at different frequencies.



\section{Introduction}

The phase noise of the phase/frequency detector (PFD) in a phase-locked loop (PLL) directly adds to that of the reference,
manifesting itself for a high frequency multiplication factor and/or a wide loop bandwidth.

This chapter investigates the phase noise mechanisms in PFDs and computes the phase noise spectral density due to both white noise and flicker
noise. The results are applied to two PFD topologies, one using static NAND gates and the other employing true single-phase clocking (TSPC). A PFD
phase noise simulation technique is also proposed. The objective is to enable the designer to predict the PFD phase noise, and more importantly,
design the PFD so as to make its contribution to the overall PLL phase noise negligible.

The chapter is organized as follows. Section 2.2 describes the background and motivation for this work. Section 2.3 builds the foundation by
calculating the jitter spectrum of an inverter and Section 2.4 extends the results to a NAND gate. Section 2.5 applies
these findings to the analysis of two PFD topologies. Section 2.6 discusses the optimization of phase noise for the two PFDs and Section 2.7
presents simulation results. Section 2.8 explains the effect of pulse position modulation of the Up and Down signals and Section 2.9 calculates 
the phase noise of square wave with uncorrelated jitter on rising and falling edges. Section 2.10 proves that the spectrum of shaped and sampled 
white noise is white under certain conditions and Section 2.11 concludes chapter 2.


\section{Background}
\subsection{Motivation}
The in-band multiplication of a PFD's phase noise can create difficulties in RF synthesizer design \cite{Tsutsumi}-\cite{Wilson}. Consider, as an example, a
5-GHz synthesizer targeting IEEE802.11a applications. To negligibly corrupt the 64QAM signal constellation, the synthesizer
must achieve an integrated phase noise of roughly $0.5^\circ$ rms \cite{Chen}.\footnote{We assume the transmit and
receive synthesizers contribute equal but uncorrelated amounts of phase noise.}
\begin{figure}[htb!]
\centering
\includegraphics[scale=1]{FIGS/CH2/fig1a.ps}
\includegraphics[scale=0.6]{FIGS/CH2/fig1b.eps}
\includegraphics[scale=1]{FIGS/b.ps}
\caption{(a) NOR-based PFD, and (b) output phase noise of a 5-GHz PLL due to PFD.}
\label{fig:pfdnor}
\end{figure}
Now, suppose the standard NOR PFD shown in Fig. \ref{fig:pfdnor}(a) is employed at the input of such a synthesizer with an input
frequency of 20 MHz and a loop bandwidth of about 2 MHz. Plotted in Fig. \ref{fig:pfdnor}(b) is the simulated output phase noise of
the synthesizer including only the PFD contribution. Here, the PFD incorporates $(W/L)_{PMOS}=0.3\ \mu$m/60 nm and
$(W/L)_{NMOS}=0.2\ \mu$m/60 nm. The area under this plot from 10 kHz to 10 MHz yields an rms jitter of $0.3^\circ$, severely tightening the
contribution allowed for the voltage-controlled oscillator (VCO).

As another example, consider a 60-GHz transceiver operating with QPSK signals. A synthesizer multiplying the above PFD phase
noise to 60 GHz would exhibit an rms jitter of $3.5^\circ$. On the other hand, for negligible corruption of QPSK signals, the rms jitter
must be less than about $2.1^\circ$ \cite{Chen}.

The above examples underscore the need for a detailed treatment of phase noise mechanisms in PFDs. Of course, the
charge pump may also contribute significant phase noise and merits its own analysis.

\subsection{Observations}
Consider the generic PLL shown in Fig. \ref{fig:pll}. The PFD generates the Up and Down pulses in response to the rising edges on $A$ and $B$. The
noise in the PFD devices modulates the width and edges of the output pulses, creating a random component in the current produced by the charge
pump (CP). We neglect the phase noise of all other building blocks and denote the input frequency by $f_{in}$. 

\begin{figure}[htb!]
\centering
\includegraphics[scale=1]{FIGS/CH2/fig2.ps}
\caption{A PFD in an integer-N PLL.}
\label{fig:pll}
\end{figure}

The phase noise in Up and Down translates to random modulation of the time during which $I_1$ or $I_2$ is injected into the loop filter.
We consider three possible cases. As shown in Fig. \ref{fig:3cases}(a), the phase noise may modulate the widths of Up and Down by the same amount,
in which case the CP produces no net output. In the second case [Fig. \ref{fig:3cases}(b)], the phase noise modulates only the position of Up with
respect to Down. As explained in Section 2.8, this effect is negligible. Lastly, the phase noise may modulate the widths of Up and Down
pulses {\em differently} [Fig. \ref{fig:3cases}(c)], and it is this case that matters most.

\begin{figure}[htb!]
\centering
\includegraphics[scale=1]{FIGS/CH2/fig3.ps}
\caption{Modulation of Up and Down (a) width by the same amount, (b) position, and (c) width differently.}
\label{fig:3cases}
\end{figure}

The above observations also reveal that, contrary to a designer's first guess, the PFD phase noise of interest is {\em not} equal to
the phase noise of the Up or Down signals themselves. After all, if the widths of Up and Down pulses vary randomly but exactly in unison, then the
net current produced by the CP contains no random component. This point raises the question of how exactly the PFD noise must be
simulated, which we address in Section 2.6.

The foregoing points suggest that the phase noise arising from a PFD in fact relates to the random pulsewidth {\em difference} between the Up and
Down signals, $\Delta T_{UD}$. Moreover, four edges, namely, the rising and falling edges of both Up and Down signals, contribute to $\Delta
T_{UD}$. Some of the PFD internal transitions displace Up and Down by the same amount and should be
ignored (Section 2.5). 

The analysis of PFD phase noise in \cite{Brennan}, \cite{Thompson} relates the phase noise to the timing jitter, $\Delta t$, as $\Delta \phi _{in} =
2 \pi f_s \Delta t$, where $f_s$ denotes the operating frequency, but expresses $\Delta t$ in terms of the (thermal) noise factor and input
resistance of the PFD.
By contrast, our approach begins with the gates comprising the PFD and determines the jitter in the Up and Down pulsewidth difference, taking into
account both flicker and thermal noise. {The mismatch between Up and Down currents is neglected here.}\footnote{Simulations show 0.2 dB
higher phase noise due to a 10\% mismatch between the Up and Down currents.}


\section{Phase Noise of CMOS Inverter}
A good understanding of the phase noise mechanisms in CMOS inverters proves beneficial in the analysis of PFDs as well. Consider the CMOS inverter
and its waveforms shown in Fig. \ref{fig:noisepercycle}. We wish to study the time envelope of the noise produced by $M_1$ and $M_2$. These transistors inject
thermal and flicker noise to the output node as they turn on. At the end of the transition, however, the on transistor carries no current and
produces no flicker noise. Thus, the thermal noise envelope of each transistor lasts about half of the input cycle, $T_{in}$, whereas its flicker
noise envelope pulsates only during transitions [Fig. \ref{fig:noisepercycle}(b)]. Note that in typical PLLs, the transition times within a PFD are much shorter than the input
period.

\begin{figure}[htb!]
\centering
\includegraphics[scale=1]{FIGS/CH2/fig4.ps}
\caption{(a) CMOS inverter, and (b) thermal and flicker noise envelopes of $M_1$.}
\label{fig:noisepercycle}
\end{figure}

In the analysis that follows, we make numerous approximations based on our intuitive understanding of the circuit's behavior. The soundness of
these approximations is ultimately put to test in Section 2.7, where two completely different PFD realizations are simulated and the results are
compared with hand calculations.

It is convenient to view the noise injection of $M_1$ and $M_2$ as follows: the transistor that is turning on injects thermal and flicker noise
during the transition, and the transistor that is turning off (coming out of the deep triode region) deposits kT/C noise at the output.

\subsection{Noise of Transistor Turning On}
In order to formulate the noise contribution by the transistors in Fig. \ref{fig:noisepercycle}, we must examine the circuits' waveforms more
closely. As depicted in Fig. \ref{fig:noiseenvelope} for a rising transition at the input and for an inverter with a fanout of about 2, the output begins to fall only
after $V_{in}$ is relatively close to $V_{DD}$. Transistor $M_1$ turns on as $V_{in}$ exceeds its threshold, $V_{THN}$, at $t=t_1$, and injects increasingly
larger flicker and thermal noise as $V_{in}$ rises. The noise envelope reaches a maximum before the transistor enters the triode region, around
$t=t_2$. Thereafter, the flicker noise injection subsides, falling to zero at $t=t_3$. The thermal noise current, on the other hand, goes from
$4kT\gamma g_m$ to a slightly lower value, $4kT$/$R_{on}$, where $R_{on}$ denotes the channel resistance of $M_1$ with $V_{GS}=V_{DD}$.

\begin{figure}[htb!]
\centering
\includegraphics[scale=1]{FIGS/CH2/fig5.ps}
\caption{Detailed view of thermal and flicker noise envelopes during input and output transitions.}
\label{fig:noiseenvelope}
\end{figure}

Our next simplifying assumption is that the output phase noise of interest manifests itself while $V_{out}$ in Fig. \ref{fig:noiseenvelope} crosses
approximately $V_{DD}$/$2$ and the noise injected by the transistors after this point is unimportant \cite{Abidi}. Thus, in the waveforms of Fig.
\ref{fig:noiseenvelope}, we consider the area under the envelopes for only up to $t=t_{mid}$. 

We now wish to approximate the area under the noise envelopes by a simple function. As shown in Fig. \ref{fig:deltat}, the flicker noise envelope is
approximated by a rectangular waveform of the same height, $h$, but lasting from the time the actual envelope reaches half of its height,
$t_{h/2}$, to the time $V_{out}$ reaches $V_{DD}$/$2$, $t_{mid}$.
We expect that the sum of the gray areas is roughly equal to the cross-hatched area. Transient noise simulations in Cadence's Spectre indicate an
error of about 4\% in this approximation. We apply the same concept to the thermal noise envelope as well. Note that \cite{Abidi} uses a
rectangle from the time $V_{out}$ begins to fall ($t_p$ in Fig. \ref{fig:noiseenvelope}) to $t_{mid}$, which, according to simulations, underestimates
the integrated noise power by 2 to 3 dB.

\begin{figure}[htb!]
\centering
\includegraphics[scale=1]{FIGS/CH2/fig6.ps}
\caption{Rectangular approximation of noise envelope.}
\label{fig:deltat}
\end{figure}

Another simplifying assumption can be derived from the waveforms in Fig. \ref{fig:noiseenvelope}: at the peak of the noise
envelope, one transistor is nearly off. Thus, we consider only the noise of $M_1$ on the falling edges at the output and only the noise of $M_2$ on
the rising edges.

Based on the foregoing approximations and utilizing the rectangular function, $w(t)$, in Fig. \ref{fig:deltat}, we now outline the inverter phase noise
analysis as follows. As shown in Fig. \ref{fig:vni}(a), the noise current of each transistor, $i_n(t)$ is equivalently multiplied by shifted versions of
$w(t)$. Each product is integrated for a duration of $\Delta T=t_{mid}-t_{h/2}$ and divided by the load capacitance, $C_L$, yielding the noise
voltage [Fig. \ref{fig:vni}(b)].
\begin{figure}[htb!]
\centering
\includegraphics[scale=1]{FIGS/CH2/fig7.ps}
\caption{(a) Equivalent operation of inverter on noise of one transistor, and (b) conversion of noise current to noise voltage.}
\label{fig:vni}
\end{figure}
These voltages are then divided by the slew rate, $r_{edge}$ {(Fig. \ref{fig:deltat})}, to give the time displacement (jitter), sampled, and summed. We write the noise
voltage, $v_{n,1}$, after the first window as 
\begin{eqnarray}
v_{n,1}&=&\frac{1}{C_L}\int_{0}^{\Delta T}\!{i_n(t)}\, dt \nonumber  \\ 
&=&\frac{1}{C_L}\int_{-\infty}^{+\infty}\!i_n(t)w(t)\, dt .
\label{vn1}
\end{eqnarray}
Note that the load capacitance is assumed constant and equal to its value at $V_{out}=V_{DD}/2$. Also, the
integration tacitly neglects the effect of the inverter's output resistance, $r_O$. This approximation is justified because the time constant,
$r_O C_L$, at the inverter output is much greater than $\Delta T$. Similarly,
\beq
v_{n,m}=\frac{1}{C_L}\int_{-\infty}^{+\infty}\!i_n(t)w(t-mT_{in})\, dt .
\label{vnm}
\eeq
The particular shape of $w(t)$ allows this equation to be rewritten as
\beq
v_{n,m}=\frac{1}{C_L}\int_{-\infty}^{+\infty}\!i_n(t)w(mT_{in}-t)\, dt ,
\label{vnm2}
\eeq
which is the convolution integral \cite{Abidi}. The noise voltage spectrum is therefore given by 
\beq
S_{Vn}(f)=\frac{1}{C_L^2}|W(f)|^2 S_{In}(f),
\label{svn}
\eeq
where $W(f)$ denotes the Fourier transform of $w(t)$ and $S_{In}(f)$ the spectrum of $i_n(t)$. 
As shown in Section 2.9, the phase noise spectrum\footnote{Throughout this chapter, all the spectra are two-sided, and the phase noise is denoted by
$S_{\Phi}(f)$.} due to noise of NMOS transistor on the falling edges is equal to
\beq
S_{\Phi}(f)=\frac{\pi^2}{r_{edge}^2 T_{in}^2}\sum_{m=-\infty}^{m=+\infty} S_{Vn}(f-\frac{m}{T_{in}}).
\label{sphi}
\eeq

It is important to recognize two differences between the above analysis and that in \cite{Abidi}: (1) as mentioned earlier, our window definition
(from $t_{h/2}$ to $t_{mid}$) more accurately predicts the injected noise power, and (2) the sampling phenomenon reveals aliasing even for flicker
noise if the $1/f$ corner, $f_{cor}$, is comparable with the operation frequency, which may be the case for PFDs.

We now simplify Eq. (\ref{sphi}) if $\overline{I_n^2}$ is white. As shown in Section 2.10, $S_{\Phi}(f)$ is also white and equal to 
\begin{eqnarray}
S_{\Phi}(f)&=&\frac{\pi^2}{r_{edge}^2 T_{in}^2}\frac{1}{C_L^2}\frac{\Delta T}{f_{in}}S_I(f) \nonumber \\ &=&\frac{\pi^2}{r_{edge}^2 C_L^2} \frac{\Delta T}{T_{in}} S_I(f).
\label{sphiw}
\end{eqnarray}
In this expression, the load capacitance appears in both $r_{edge}$ ($=I_D/C_L$, where $I_D$ is the drain current of the on transistor as $V_{out}$
crosses $V_{DD}/2$) and in $\Delta T$. Thus, $S_{\Phi}(f)$ is {\em directly} proportional to $C_L$ and $f_{in}$. The output phase noise due to white
noise therefore rises by 3 dB for each doubling of the operation frequency.

The flicker noise behavior of the inverter can also be deduced from Eq. (\ref{sphi}). If $f_{in}$ is well above the flicker noise corner
frequency, no aliasing occurs and (\ref{sphi}) is simplified by choosing $m=0$ :
\beq
S_{\Phi}(f)=\frac{\pi^2}{r_{edge}^2 T_{in}^2}S_{Vn}(f).
\label{sphina}
\eeq
Since $\Delta T$ is much less than $1/f_{cor}$, we can assume $W(f)={\Delta T}^2 {sinc}^2(\pi f \Delta T)$ is relatively constant for the frequency
range of interest and equal to ${\Delta T}^2$. It follows that
\beq
S_{\Phi}(f)=\frac{\pi^2}{r_{edge}^2 T_{in}^2}\frac{{\Delta T}^2}{C_L^2}S_{1/f}(f),
\label{sphif}
\eeq
where $S_{1/f}(f)$ denotes the noise current spectral density of the on transistor due to its $1/f$ noise. In this case, the phase noise rises by 6
dB for each doubling of $f_{in}$. It also exhibits a stronger dependence upon $\Delta T$. As mentioned earlier, (\ref{sphiw}) and (\ref{sphif}) are
evaluated for $M_1$ on the falling edge at the output and for $M_2$ on the rising edge.
Note that \cite{Abidi} does not analyze the effect of flicker noise in CMOS inverters.

\subsection{Noise of Transistor Turning Off}
As illustrated in Fig. \ref{fig:noiseenvelope}, when the noise envelope reaches its peak, one transistor is near the edge of the triode region and the other is almost
off. Before turning off, however, this transistor has acted as a resistor, producing noise across $C_L$. {Turning off once every $T_{in}$ seconds,
the NMOS transistor deposits a noise voltage whose spectral density is given by $(kT/C_L)/f_{in}$. As
shown in Section 2.9, the falling edges
exhibit a phase noise equal to 
\beq
S_{1}(f)=\frac{\pi^2}{T_{in}^2}\frac{1}{r_{edge}^2}\frac{kT}{C_L f_{in}}.
\label{sphi1}
\eeq
Taking the PMOS contribution into account, we obtain the total kT/C-induced phase noise as
\beq
S_{\Phi}(f)=\frac{2 \pi^2}{T_{in}^2}\frac{1}{r_{edge}^2}\frac{kT}{C_L f_{in}}.
\label{sphioff}
\eeq
}

\subsection{Total Phase Noise}
The total phase noise is given by the sum of five terms: Eqs. (\ref{sphiw}) and (\ref{sphif}) evaluated for both NMOS and PMOS transistors, and
Eq. (\ref{sphioff}):

\begin{eqnarray}
S_{\Phi}(f)&=& \left\{\frac{\pi^2}{r_{edge}^2 C_L^2}[\frac{{\Delta T}}{T_{in}}S_{I}(f)+\frac{{\Delta T}^2}{T_{in}^2}S_{1/f}(f)]\right\}_{NMOS} \nonumber  \\ 
&+&\left\{\frac{\pi^2}{r_{edge}^2 C_L^2}[\frac{{\Delta T}}{T_{in}}S_{I}(f)+\frac{{\Delta T}^2}{T_{in}^2}S_{1/f}(f)]\right\}_{PMOS} \nonumber  \\ 
&+&\frac{2 \pi^2 f_{in}}{r_{edge}^2}\frac{kT}{C_L}.
\label{sphitot}
\end{eqnarray}




\section{Phase Noise of CMOS NAND Gate}
The inverter phase noise analysis can be readily extended to other CMOS gates as well. We briefly consider here the noise behavior of a static NAND
gate and use the results in Section 2.5 to study a NAND-based PFD.

Since in a PFD environment, the two inputs do not change simultaneously, we can reduce the gate to an inverter for each transition. Such an inverter
incurs an additional capacitance at the output due to the second PMOS transistor, and its output falling edge is produced by the series combination
of two NMOS transistors (Fig. \ref{fig:nand}).

\begin{figure}[htb!]
\centering
\includegraphics[scale=1]{FIGS/CH2/fig8.ps}
\caption{NAND gate with one input changing.}
\label{fig:nand}
\end{figure}

In our PFD design, $M_1$ and $M_3$ have the same width and minimum length; thus, they can be replaced with one NMOS device having twice their
length.\footnote{The drain and source capacitance at node X introduce a negligible error in this equivalency.}
In other words, Eq. (\ref{sphitot}) holds if $r_{edge}$, $\Delta T$, $C_L$ and $S_{In}(f)$ are modified to reflect the equivalent values
in the NAND circuit.




\section{PFD Phase Noise Analysis}
\subsection{NAND PFD}
As suggested by the factors ${\Delta T}$ in (\ref{sphiw}) and $\Delta T^2$ in (\ref{sphif}), the phase noise rises in proportion to
the turn-on time of the transistors in each gate. A worthy effort in PFD design, therefore, is to minimize the rise and
fall times. We thus modify the standard NOR-based PFD to the NAND-based topology shown in Fig. \ref{fig:pfdnand}(a). Note that this circuit
responds to the {\em falling} edges of $A$ and $B$, and its Up and Down outputs are {\em low} when asserted.

We must now examine the propagation of the edges through the PFD circuit, seeking those whose jitter modulates the
pulsewidth {\em difference} between the Up and Down pulses. To this end, we draw a detailed timing diagram, mark with a certain shade or
pattern the
jitter contributed by each gate to each transition, carry the jitters on to the final Up and Down
pulses, and omit those that are in common.

\begin{figure}[htb!]
\centering
\includegraphics[scale=1]{FIGS/CH2/fig9.ps}
\caption{(a) NAND-based PFD, (b) jitter contributions to falling edges of outputs, and (c) jitter contributions to rising edges of outputs.}
\label{fig:pfdnand}
\end{figure}

Figure \ref{fig:pfdnand}(b) shows the timing diagram, assuming input $A$ falls earlier than input $B$. NAND 1 adds jitter to the
falling edge of $A$, producing a rising edge on $\overline{Up}$. This edge experiences additional jitter in NAND 2 and
generates the falling edge of Up. That is, each falling edge of Up is corrupted by only the jitters of NANDs 1 and 2.
Similarly, when a falling edge of $B$ follows, $\overline{ Down}$ rises with NAND 5's jitter and Down falls with both
NAND 5's and NAND 6's jitters.

We must also follow the $\overline{ Up}$ and $\overline{ Down}$ rising edges through the reset path. As illustrated in Fig.
\ref{fig:pfdnand}(c), after $\overline{ Down}$ goes up, Reset falls, inheriting the jitters of NAND 5 and NAND 9. In response, $E$
and $F$ rise, incurring additional jitter from NAND 4 and NAND 8, respectively. Subsequently, $C$ falls with the jitter
of NAND 3 and $D$ with that of NAND 7. Finally, Up and Down rise with the jitters of NAND 2 and NAND 6,
respectively.

The Up and Down waveforms in Fig. \ref{fig:pfdnand}(c) merit two remarks. First, NAND 2 contributes jitter to both the rising
and falling edges of Up, but the two jitters are uncorrelated because the former is due to a PMOS device and the latter
due to an NMOS device (the series combination of $M_1$ and $M_3$ in Fig. \ref{fig:nand}). A similar observation applies to NAND 6
contributions to Down. Second, the jitter produced by NAND 9 appears on the rising edges of both Up and
Down pulses and hence is immaterial.
As seen from Fig. \ref{fig:pfdnand}(c), NANDs 1-8 make a total of 10 contributions to the pulsewidth
difference between Up and Down. The phase noise spectral densities of these contributions
are summed to obtain the overall PFD phase noise. 

{In response to the jitter components in the Up and Down pulses (except for those that are in common), the charge pump in Fig.
\ref{fig:pll} produces an error current, $\Delta I$.
Adding up the powers of uncommon jitters, $T_m$, $m=1,...,10$, in the Up and Down pulses, we have
\beq
\overline{\Delta I^2}=\frac {I_p^2}{T_{in}^2}\sum_{m=1}^{10}T_m^2.
\label{deltai2}
\eeq
It can be shown that the transfer function from this current injection to the PLL output within the loop bandwidth is equal to $\Phi_{out,
PLL}/\Delta I =(2\pi/I_p)M$. It follows that 
\beq
S_{\Phi,PLL}(f)=\frac {4\pi^2}{T_{in}^2}M^2\sum_{m=1}^{10}S_{Tm}(f),
\label{sphipll0}
\eeq
where $S_{Tm}(f)$ denotes the spectral density of jitter component $T_m$ and is equal to $S_{Vn}(f)/r_{edge}^2$.
}
For roughly similar gates and rise and fall times, the in-band phase noise observed at the PLL output is given by
\begin{eqnarray}
S_{\Phi,PLL}\!\!\!\!&\cong&\!\!\!\! 10 M^2 [\frac{4\pi^2{\Delta T}}{r_{edge}^2 C_L^2 T_{in}}S_{I}(f) + \frac{4\pi^2{\Delta T}^2}{r_{edge}^2 C_L^2 T_{in}^2}S_{1/f}(f)\nonumber  \\ 
&+&\!\!\!\!\frac{4 \pi^2 f_{in}}{r_{edge}^2}\frac{kT}{C_L}].
\label{sphipll}
\end{eqnarray}
As explained in Section 2.6, however, an optimum design may incorporate different sizings for the gates. 


An important point emerging from our analysis is that, to reduce the flicker noise of a PFD, the channel length
of its constituent transistors must {\em not} be increased. This is because longer-channel devices inevitably raise
$\Delta T$ in (\ref{sphipll}). Instead, the channel area of the transistors can be increased by choosing {\em
wider} devices.

\subsection{TSPC PFD}
The foregoing analysis can be applied to other PFD topologies as well. In this section, we study the phase noise of a TSPC
implementation \cite{tspcpfd} as it can potentially achieve a higher speed and proves useful in cascaded PLLs. Depicted in Fig.
\ref{fig:pfdtspc}(a), the circuit operates as follows. A rising edge on $A$ turns on $M_5$, discharging the Up output. Similarly,
a rising edge on $B$ discharges the Down output. Once both Up and Down are low, Reset rises, discharging nodes
$C$ and $D$ and forcing Up and Down to go high.

\begin{figure}[htb!]
\centering
\includegraphics[scale=1]{FIGS/CH2/fig10.ps}
\caption{(a) TSPC PFD, and (b) jitter contributions to the outputs.}
\label{fig:pfdtspc}
\end{figure}

In a manner similar to the analysis of the NAND PFD, we follow the transitions through the circuit and mark the jitter
contributed by each stage. As illustrated in Fig. \ref{fig:pfdtspc}(b), the falling edges of Up and Down are corrupted by the
noise of the series combinations $M_5$-$M_6$ and $M_{11}$-$M_{12}$, respectively. Next, Reset experiences the jitter
due to $M_{11}$-$M_{12}$ and the NOR gate. The falling transitions at $C$ and $D$ inherit the jitter of Reset and incur
additional noise due to $M_3$ and $M_9$, respectively. Finally, these edges are corrupted by the noise of $M_4$ and
$M_{10}$.

Let us draw several conclusions. First, the jitter of the NOR gate modulates the widths of Up and Down equally and hence
is ignored. Second, the overall TSPC PFD phase noise arises from six transitions and can be potentially smaller than that
of the NAND PFD. Third, the noise injection mechanisms in each stage are similar to those of the inverter and NAND gates
studied earlier. For example, when $M_5$ turns on, its corresponding stage acts approximately like a NAND gate (except
that $M_4$ has been off well before this transition). Also, when node $C$ falls, the series combination 
$M_5$-$M_6$ deposits $kT/C$ noise at the output while $M_4$ turns on as in an inverter and injects both thermal and
flicker noise. Thus, Eq. (\ref{sphipll}) applies here as well if the factor of 10 is replaced by 6 and the gates and rise and fall times are assumed
similar.


\section{Design Optimization}
With the insights developed above into PFD phase noise mechanisms, we now seek to optimize each design for minimum phase
noise. Of course, one can simply enlarge the widths of all of the PFD transistors by a factor of $\alpha$ so as to
reduce the phase noise by the same factor, but at the cost of proportionally higher power consumption. A more methodical approach,
however, is to assume a certain power budget and determine the best sizing of the transistors
that yields minimum phase noise. This optimization can still be followed by the above scaling technique to trade
power for phase noise. We consider $1/f$ noise here as it dominates for offsets as high as 10 MHz, but optimization for
thermal noise is similar.

Since the PFD power dissipation is proportional to the total transistor width in the signal path, $W_{tot}$, we must
determine how a given $W_{tot}$ is apportioned among the transistors so as to minimize the phase noise. Our general
procedure is to favor transistors that define the transition time of critical edges. We also make four approximations: (1)
The capacitance at a given node is proportional to the width of the ``driver'' transistor, $W_a$, and the width of the
``driven'' transistor, $W_b$: $ C_L \propto \eta W_a + W_b$. The first term on the right accounts for the drain junction
capacitance and the Miller multiplication of the gate-drain overlap capacitance at the output node (about a factor of 2).
(2) The drain $1/f$ noise current spectrum is given by $S_{1/f}(f)=g_m^2 K_f/(W_a L_a C_{ox} f)$, where $g_m \approx I_D /
(V_{GS}-V_{TH})$ and $V_{GS}=V_{DD}$.\footnote{This $g_m$ equation assumes heavy velocity saturation. For long-channel
devices, $g_m \approx 2 I_D /(V_{GS}-V_{TH})$. This distinction is not critical in our analysis.} (3) At the point of
interest, namely, $V_{GS} \approx V_{DD}$ and $V_{out} \approx V_{DD}/2$, we have $I_D \propto W_a$ regardless of the
transistor (short-channel) characteristics. Thus, the slew rate in Eq. (\ref{sphif}), $r_{edge} \propto I_D / C_L \propto W_a /
C_L$. (4) The window width, $\Delta T$, is proportional to $V_{DD}/r_{edge} \approx V_{DD}C_L/I_D$. Equation (\ref{sphif}) is
now rewritten as
\beq
S_{\Phi}(f) \propto \frac{{f_{in}}^2 {V_{DD}}^2 {C_L}^2}{{W_a}^3}\frac{1}{f}.
\label{sphifsimple}
\eeq
For given values of $f_{in}$, $V_{DD}$, and $f$,
\beq
S_{\Phi}(f) \propto \frac{(\eta W_a + W_b)^2}{{W_a}^3}.
\label{sphifprop}
\eeq
The power consumed to charge and discharge such a node once per cycle is approximately equal to $P = f_{in} C_L
{V_{DD}}^2$. We now apply these results to the optimization of the NAND and TSPC PFDs.

\subsection{NAND PFD Optimization}
As evident from Figs. \ref{fig:pfdnand}(b) and (c), the NAND PFD phase noise arises from five transistors: the PMOS device in NAND 1,
the NMOS device in NAND 2, the PMOS device in NAND 4, the NMOS device in NAND 3, and the PMOS device in NAND 2. Denoting
the widths of PMOS and NMOS transistors in NAND $j$ by $W_{Pj}$ and $W_{Nj}$, respectively, we use Eq. (\ref{sphifprop}) to express the
first PMOS contribution as:
\beq
S_{\Phi _1}(f) \propto \frac{[\eta (2 W_{P1} + W_{N1}) + W_2 + W_3 + W_9]^2}{{W_{P1}}^3}.
\label{sphin1}
\eeq
Here, the factor of 2 accounts for the two PMOS devices tied to the output and $W_j = W_{Pj} + W_{Nj}$. The sum $W_2 + W_3
+ W_9$ represents the load due to the three NANDs driven by NAND 1. The other four contributions can be expressed in a
similar manner, e.g., for the NMOS device in NAND 2:
\beq
S_{\Phi _2}(f) \propto \frac{[\eta (2 W_{P2} + W_{N2}) + W_{P1}+ W_{N1}]^2}{{W_{N2}}^3}.
\label{sphin2}
\eeq
Note that the proportionality factors relating the right-hand sides of (\ref{sphin1}) and (\ref{sphin2}) to their left-hand side are
different as they include the mobility and flicker noise coefficient of PMOS and NMOS devices, respectively. The total power consumption
satisfies the relation:
\beq
P \propto f_{in} {V_{DD}}^2 [W_{P9} + W_{N9} + 2 \sum_{j=1}^{4} (W_{Pj} + W_{Nj})].
\label{powern}
\eeq

As explained in Section 2.5, the jitter of some of the edges does not enter the overall PFD phase noise. The transistors
causing these edges can therefore have nearly minimum widths so long as they respond fast enough to avoid circuit failure. The devices falling into this
category are the NFETs in NANDs 1, 4, and 9 and the PFETs in NANDs 3 and 9. 
The sum of the five phase noise contributions described above must be minimized subject to the power budget imposed by
(\ref{powern}). This is accomplished using the ``fmincon'' function in MATLAB. For example, a total width of 162 $\mu$m {(corresponding to
0.24 mW at 1 GHz)} for the transistors yields
$W_{P1} = 11$, $W_{N1} = 0.12$, $W_{P2} = 9.1$, $W_{N2} = 5.9$, $W_{P3} = 0.12$, $W_{N3} = 6.22$, $W_{P4} = 7.8$, $W_{N4} = 0.12$,
$W_{P9} = 0.12$, $W_{N9} = 0.12$, all in microns. Using transient circuit simulations, we adjust some of the noncritical transistors widths
so to minimize crowbar currents and speed up the critical transitions, obtaining 
$W_{P1} = 10.6$, $W_{N1} = 0.5$, $W_{P2} = 8.5$, $W_{N2} = 5.5$, $W_{P3} = 0.6$, $W_{N3} = 5.84$, $W_{P4} = 7.4$, $W_{N4} = 0.5$,
$W_{P9} = 0.12$, $W_{N9} = 2$, all in microns. It is interesting that such a range of widths would not be obvious if we attempted to manually
optimize the PFD transistors by trial and error. As shown in Section 2.7, this optimization lowers the phase noise by 4 to 6 dB.

\subsection{TSPC PFD Optimization}
The foregoing procedure can be applied to the TSPC PFD of Fig. \ref{fig:pfdtspc}(a) as well. Here the phase noise has three contributions arising from
$1/f$ noise:
\beq
S_{\Phi _1}(f) \propto \frac{[\eta (W_4 + W_5) + W_{P,NOR}  + W_{N,NOR}]^2}{{W_5}^3},
\label{sphit1}
\eeq
where $W_j$ refers to the width of $M_j$ and $W_{P,NOR}$ and $W_{N,NOR}$ are the PMOS and NMOS widths in the NOR gate, respectively. The
power consumption satisfies the relation:
\beq
P \propto f_{in} {V_{DD}}^2 (2 \sum_{j=1}^{6} W_j + W_{P,NOR}  + W_{N,NOR}).
\label{powert}
\eeq
For simplicity, we assume equal widths for the transistors within each cascode structure. Also, $M_1$-$M_2$ and $M_7$-$M_8$ in Fig.
\ref{fig:pfdtspc}(a) contribute no
jitter to the PFD and hence can have small widths. For example, a total width of 162 $\mu$m {(corresponding to
0.2 mW at 1 GHz)} is apportioned as follows:
$W_1 = 0.12, W_2 = 0.12, W_3 = 28,
 W_4 = 25, W_5 = 13.72, W_6 = 13.72, 
 W_{P,NOR} = 0.12, W_{N,NOR} = 0.12$,
all in microns. Manual adjustment to improve transition times in the simulations yields
$W_1 = 1.4, W_2 = 1.4, W_3 = 12, 
 W_4 = 24, W_5 = 10, W_6 = 10,
 W_{P,NOR} = 10, W_{N,NOR} = 0.12$, all in microns. As discussed in Section 2.7, this optimization reduces the phase noise by 5 to 8 dB.

\subsection{Dependence on Operation frequency}
Equation (\ref{sphipll}) reveals that the phase noise of PFDs rises in proportion to $f_{in}$ in the thermal regime and
${f_{in}}^2$ in the flicker noise regime. This dependence imposes certain bounds on the in-band phase noise of PLLs. For a feedback divide
ratio of $M$, the first term in Eq. (\ref{sphipll}) yields an output phase noise of
\begin{eqnarray}
S_{\Phi, out}(f) &\propto& f_{in} M^2 S_{I}(f) \nonumber \\ &\propto& \frac{{f_{out}}^2} {f_{in}} S_{I}(f).
\label{sphiwfreq}
\end{eqnarray}
That is, to minimize the phase noise due to the PFD thermal noise, $f_{in}$ must be maximized. For PFD flicker noise, on the other hand, 
\begin{eqnarray}
S_{\Phi, out}(f) &\propto& {f_{in}}^2 M^2 S_{1/f}(f) \nonumber \\ &\propto& {{f_{out}}^2} S_{1/f}(f).
\label{sphiffreq}
\end{eqnarray}
Interestingly, this PFD contribution is independent of the input frequency so long as flicker noise does not experience aliasing.




\section{Simulation Results}
This section presents simulation results in 65-nm CMOS technology for the circuits studied in this chapter and compares them with our
analytical derivations. The objective is threefold: (a) validate the trends
predicted by our analysis, e.g., the dependence of phase noise upon the input frequency and node capacitance, (b) check the absolute accuracy
of the analytical results, and (c) examine the soundness of our optimization procedure.

A few remarks with respect to the hand calculations are warranted. First, the transistor capacitances, drain bias currents, and drain (1/f and
thermal) noise currents are obtained from ac and transient simulations for various values of $V_{GS}$ and $V_{DS}$. These simulations also reveal the peak
noise current and the gate-source voltage, $V_{GS,half}$, at which the noise current is equal to half of its peak. Second, the window width,
$\Delta T$, in Eqs. (\ref{sphiw}), (\ref{sphif}), (\ref{sphitot}) and (\ref{sphipll}) is derived from transient simulations of the stage of interest by finding the time at
which the gate-source voltage reaches $V_{GS,half}$.

\subsection{Inverter and NAND Simulations}
Figure \ref{fig:inverterchain} plots the phase noise of a chain of eight inverters with $W_P=6\ \mu$m and $W_N=3\ \mu$m at an input frequency of 1 GHz. (As
explained in Section 2.7.2, scaling to other frequencies is straightforward.) In
order to investigate the robustness of our analytical approach, the chain is also studied with an additional node capacitance of 20
fF. In each case, the results of Cadence pnoise simulations are compared with those of hand calculations. Figure \ref{fig:nandchain} repeats these
experiments for a chain of eight NAND gates with one input tied to $V_{DD}$ and $W_P=W_N=6\ \mu$m. We observe
that in all cases, the hand calculations incur an error of less than 2 dB.

\begin{figure}[htb!]
\centering
\includegraphics[scale=0.6]{FIGS/CH2/fig11.eps}
\caption{Phase noise of a chain of eight inverters running at 1 GHz.}
\label{fig:inverterchain}
\end{figure} 

\begin{figure}[htb!]
\centering
\includegraphics[scale=0.6]{FIGS/CH2/fig12.eps}
\caption{Phase noise of a chain of eight NANDs running at 1 GHz (with one input tied to $V_{DD}$).}
\label{fig:nandchain}
\end{figure} 


\subsection{PFD Simulations}
As argued in Section 2.2, the PFD phase noise cannot be simulated by examining only the Up or Down pulses. For this reason, we embed the
PFD within an otherwise ideal PLL, run a pss and pnoise analysis, allow the PLL to settle, and compute the output phase noise
of the PLL in the steady state. If the PLL bandwidth is large enough, the PFD phase noise up to the offset frequencies of interest passes to
the output unattenuated. Such a simulation takes a long time but is necessary here to demonstrate the validity of our approach.
The PLL comprises behavioral descriptions of the VCO, frequency divider, and charge pump. The loop filter employs a noiseless resistor. To ensure
that the PLL does not attenuate the PFD phase noise for offset frequencies as high as 100 MHz, the reference frequency, $f_{ref}$, is chosen equal to or
greater than 1 GHz. 
{Such a high value is chosen so as to readily observe and validate the effect of flicker noise. 
For much lower input frequencies, the aliasing of white noise tends to mask the effect of flicker noise, making it difficult to correlate the
simulations with the analytical results. For example, if $f_{ref}$ is reduced to 20 MHz, then the effect of flicker noise rises by $10\log(50)=17$
dB and that of white noise by $20\log(50)=34$ dB, masking the former.} 




Figure \ref{fig:pfdnandsim} plots the simulated and calculated phase noise of the NAND PFD for different input frequencies. (Each simulation
incorporates a different set of PLL parameters\footnote{For example, $R_1=600~\Omega$, $C_1=200$ pF, $C_2=100$ fF, $I_p=1$ mA,
$M=1$,
and $K_{VCO}=2\pi(1.5\times 10^9)$ rad/s.} commensurate with the reference frequency.) As predicted in
Section 2.3, doubling $f_{ref}$ raises
the phase noise by 6 dB in the 1/f noise regime and by 3 dB in the white noise regime. The error in the analytical calculations is 3.1 dB. 
{The effect of white noise is overestimated possibly due to assuming that all of the high-frequency noise components experience only a
$ sinc^2$ envelope before folding, whereas in the actual circuit, these components are also attenuated by the finite bandwidth and hence do not
extend to infinity.
}

Figure \ref{fig:pfdtspcsim} plots similar results for the TSPC PFD. The maximum error in this case is 2.8 dB. Designed for the same power
consumption as the NAND PFD, the TSPC topology exhibits about 6 dB lower phase noise.

Illustrated in Fig. \ref{fig:pfdoptim} are the results of the optimization procedure described in
Section 2.6. For a given power consumption, the phase noise is
reduced by 4 to 8 dB for the two PFDs.

\begin{figure}[htb!]
\centering
\includegraphics[scale=0.6]{FIGS/CH2/fig13.eps}
\caption{Phase noise of NAND PFD at various input frequencies.}
\label{fig:pfdnandsim}
\end{figure} 
\begin{figure}[htb!]
\centering
\includegraphics[scale=0.6]{FIGS/CH2/fig14.eps}
\caption{Phase noise of TSPC PFD at various input frequencies.}
\label{fig:pfdtspcsim}
\end{figure} 
\begin{figure}[htb!]
\centering
\includegraphics[scale=0.6]{FIGS/CH2/fig15.eps}
\caption{Phase noise of NAND and TSPC PFDs before and after optimization.}
\label{fig:pfdoptim}
\end{figure} 




\section{Effect of Pulse Position Modulation}
In this section, we show that if noise modulates only the position of the Up or Down pulses, the resulting phase noise is negligible. Consider the waveforms
depicted in Fig. \ref{fig:ppmcompare}(a), where Up and Down have a pulsewidth of $T_{RST}$ and a random skew of $T_{skew}$. Assuming an ideal charge pump, we note that the
disturbance on the oscillator control voltage is in the form of a pulse with a mean width of $T_{RST}$. By contrast, as shown in Fig. \ref{fig:ppmcompare}(b), a pulsewidth
difference of $T_D$ between Up and Down manifests itself as a step on the control voltage, producing a much larger phase disturbance. 
\begin{figure}[htb!]
\centering
\includegraphics[scale=1]{FIGS/CH2/fig16.ps}
\caption{Modulation of (a) position, and (b) pulsewidth of Up and Down signals.}
\label{fig:ppmcompare}
\end{figure}

\section{Phase Noise of Square Wave with Uncorrelated Jitters on Rising and Falling Edges}
It is usually assumed that an edge displacement of $\Delta T$ translates to a phase change of $2\pi \Delta T / T_{in}$, where $T_{in}=1/f_{in}$ denotes the period. Of
course, if {\em all} of the edges of a square wave are displaced by $\Delta T$, this amount of phase change arises. However, jitter affects the consecutive edges
differently, requiring a closer look at the resulting phase noise.

Let us first suppose a sinusoidal jitter, $T_m \cos \omega_m t$, is applied to only the rising edges of an ideal square wave, $p(t)$. As shown in Fig.
\ref{fig:squaretonepn}(a), the rising edge at $k T_{in}$ is displaced by an amount equal to $T_m \cos (\omega_m k T_{in})$.
\begin{figure}[htb!]
\centering
\includegraphics[scale=1]{FIGS/CH2/fig17.ps}
\caption{(a) Square wave with modulated rising edges, (b) decomposition into two waveforms, and (c) resulting magnitude of Fourier transform.}
\label{fig:squaretonepn}
\end{figure}
This jittery waveform can be expressed as the sum of $p(t)$ and a train of pulses that occur at $k T_{in}$ with a width of $T_m \cos (\omega_m k T_{in})$
[Fig. \ref{fig:squaretonepn}(b)]. If $T_m \ll T_{in}$, the latter can be approximated by a train of impulses and expressed as
\beq
q(t) \approx T_m  \sum_{k=-\infty}^{k=+\infty} \cos (\omega_m k T_{in})\;\delta (t-k{T_{in}}).
\label{qt}
\eeq
Adding the Fourier transforms of $p(t)$ and $q(t)$, we obtain the result shown in Fig. \ref{fig:squaretonepn}(c), where each harmonic of the square wave is
surrounded by two impulses of area $T_m / (2T_{in})$ at frequency offsets of $\pm f_m= \pm \omega _m / (2 \pi)$. It can be shown that these sidebands
generate only phase modulation (PM).

We thus observe that a jitter spectrum consisting of two impulses having an area of $T_m/2$ produces two PM sidebands around $f_{in}$ whose normalized
magnitude is equal to $\pi T_m / (2 T_{in})$. That is, a jitter of $T_m/2$ yields a phase disturbance of $(\pi /T_{in})(T_m /2)$ rather than $(2 \pi
/T_{in})(T_m /2)$ in this case. One may expect this result because only the rising edges have been displaced.

We now generalize the foregoing observation to random jitter, while still assuming jitter on only the rising edges. If the jitter itself in the time
domain is denoted by $\sigma (t)$, then Eq. (\ref{qt}) is rewritten as 
\beq
q(t) \approx  \sigma (t) \sum_{k=-\infty}^{k=+\infty} \delta (t-k{T_{in}}).
\label{qtrandom}
\eeq
Adding the power spectral densities of $p(t)$ and $q(t)$, we obtain the overall spectrum shown in Fig. \ref{fig:psdrising}. Thus, the jitter spectrum,
$S_\sigma(f)$, is shifted to $\pm f_{in}$, $\pm 2f_{in}$, etc., scaled by a factor of $1/{T_{in}}^2$, and normalized to a carrier power of $1/\pi
^2$, yielding $(\pi^2/{T_{in}}^2)S_\sigma(f \pm f_{in})$, etc., for the phase noise.\footnote{Using Rice's approximation of random noise by a sum of
sinusoids \cite{Rice}, it can be proved that the spectra at $\pm f_{in}$ produce only phase modulation.} MATLAB simulations confirm this result.
\begin{figure}[htb!]
\centering
\includegraphics[scale=1]{FIGS/CH2/fig18.ps}
\caption{Spectrum of jittery square wave.}
\label{fig:psdrising}
\end{figure}

Since the jitters on the rising and falling edges of a CMOS inverter's output are generated by different transistors and are hence uncorrelated, we
write the overall phase noise of the square wave as
\beq
S_\Phi(f) = \frac{\pi^2}{{T_{in}}^2}\sum_{k=-\infty}^{k=+\infty} [S_{\sigma p}(f \pm k f_{in}) + S_{\sigma n}(f \pm k f_{in})],
\label{sphiuncorrelated}
\eeq
where $S_{\sigma p}$ and $S_{\sigma n}$ denote the spectra of the jitters produced by the PMOS and NMOS transistors, respectively. Note that $S_\sigma$ and
$S_{Vn}$ are simply related by a factor of $r_{edge}^2$.


\section{Spectrum of Shaped and Sampled White Noise}
In this section, we examine the phase noise spectrum due to white noise:
\beq
S_{\Phi}(f)=\frac{\pi^2}{r_{edge}^2 T_{in}^2}\sum_{m=-\infty}^{m=+\infty} S_{Vn}(f-\frac{m}{T_{in}}).
\label{sphia1}
\eeq
Since the Fourier transform of the rectangular window, $w(t)$, is given by $\Delta T\times$ $\sin(\pi f \Delta T)$$/(\pi f \Delta T)$, we have from (\ref{svn})
\beq
S_{Vn}(f)=\frac{1}{C_L^2}{\Delta T}^2 \frac{\sin^2(\pi f \Delta T)}{(\pi f \Delta T)^2}S_{In}(f).
\label{svna}
\eeq
If $S_{In}(f)$ is white, then $S_{Vn}(f)$ has a $ sinc^2$ shape; i.e., $S_{\Phi}(f)$ consists of $ sinc^2$ functions centered at $m f_{in}=m/T_{in}$.
We now prove that the sum of these $ sinc^2$ functions is a flat line under a certain condition.

\begin{figure}[htb!]
\centering
\includegraphics[scale=1]{FIGS/CH2/fig19.ps}
\caption{Inverse Fourier transform of (a) $ sinc^2$ function, and (b) shifted $ sinc^2$ functions.}
\label{fig:sinc2}
\end{figure}

Considering only the $ sinc^2$ shape itself, we recognize that the inverse Fourier transform of ${\Delta T}^2 { sinc^2}(\pi f \Delta T)$ is a triangle,
$g(t)$, with a time duration of $-\Delta T$ to $+\Delta T$ and a height of ${\Delta T}$ [Fig. \ref{fig:sinc2}(a)]. As a result of shifts of $ sinc^2$ by $m
f_{in}$ in the frequency domain, $g(t)$ is multiplied by ${ exp}(j2\pi m f_{in}t)$ in the time domain:
\beq
g(t)\sum_{m=-\infty}^{m=+\infty} e^{j2\pi m f_{in}t} \leftrightarrow \sum_{m=-\infty}^{m=+\infty}{\Delta T}^2 \frac {\sin^2[\pi \Delta
T(f-mf_{in})]}{[\pi \Delta T(f-mf_{in})]^2}.
\label{ga}
\eeq
We also note that
\beq
\sum_{m=-\infty}^{m=+\infty} e^{j2\pi m f_{in}t} = \frac{1}{f_{in}} \sum_{m=-\infty}^{m=+\infty} \delta (t-mT_{in}).
\label{suma}
\eeq
In other words, $g(t)$ is multiplied by a train of impulses centered at $m T_{in}$ [Fig. \ref{fig:sinc2}(b)]. Thus, if the duration of $g(t)$ is short enough
to enclose only the impulse at $t=0$, we have 
\beq
g(t)\sum_{m=-\infty}^{m=+\infty} e^{j2\pi m f_{in}t} = \Delta T \frac{1}{f_{in}} \delta (t).
\label{gsuma}
\eeq
The Fourier transform of this result is equal to $\Delta T / f_{in}$ and hence:
\beq
\sum_{m=-\infty}^{m=+\infty}{\Delta T}^2 \frac {\sin^2[\pi \Delta T(f-mf_{in})]}{[\pi \Delta T(f-mf_{in})]^2} = \frac{\Delta T}{f_{in}},
\label{suma2}
\eeq
which is a flat line.

In summary, if the sampling period, $T_{in}$, is greater than the rectangular window width, $\Delta T $, then the window-integrated and sampled white noise
still has a white spectrum. Note that this result is valid for any shape chosen for $w(t)$ so long as the inverse Fourier transform of $|W(f)|^2$
has a total time duration less than $2 T_{in}$, or more generally, so long as the inverse Fourier transform of $|W(f)|^2$ crosses zero at $t=m
T_{in}$ except for $t=0$.




\section{Conclusion}
The phase noise of PFDs can manifest itself within the bandwidth of PLLs, corrupting the transmitted and received signal constellations. This chapter
analyzes the phase noise of two PFD topologies based on the approximations made for a CMOS inverter. It is also shown that the PFD phase noise is
not merely that of the Up and Down pulses. Simulations using each PFD in a PLL reveal good agreement with analytical predictions, indicating, most
notably, the dependence of the phase noise on the frequency of operation.

