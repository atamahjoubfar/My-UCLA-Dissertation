\chapter{Introduction} 

The design of RF integrated circuits continues to challenge engineers and researchers, demanding new circuit topologies and transceiver architectures. New ideas often require new analysis techniques as well, so that the designer can insightfully quantify the underlying
principles. In addition, novel analysis of the existing circuits helps the designer to optimize them and sometimes suggests modifications for better performance.

\section{Organization} 

This dissertation proposes three novel analyses in RF circuits. The first analysis which is covered in chapter 2, is on the phase noise in phase/frequency detectors (PFDs), an essential component in RF synthesizers. This is the only published work on the subject that
derives equations. Using our compact equations, we have optimized two PFDs for minimum phase noise. Chapter 3 is the first analysis of the relation between the phase noise of delay lines and ring oscillators. Knowing the relation, compact equations are derived for the
phase noise of ring oscillators which show a factor of 2 correction for the flicker-noise-induced phase noise. In addition, it dispels the commonly-accepted premise that symmetric rise and fall times in a ring oscillator suppress the upconversion of flicker noise.
Chapter 4 deals with the design of a new low-power RF CMOS receiver for IEEE 802.11a. The most interesting part is the replacement of the LNA with a transformer. How do the transformer and mixers provide proper input matching? Why the frequency response is flat across
the channel while the loads of the mixers are capacitors? Why the noise figure is not too high? These are all important questions that would not have been answered without our novel analysis explained in chapter 4. Finally, chapter 5 summarizes the future work.
