\begin{center}
{\Large \bf A 5-GHz 11.6-mW CMOS Receiver for IEEE 802.11a Applications}                        
                \vskip.2in
{Aliakbar Homayoun and Behzad Razavi}\\
{Electrical Engineering Department}\\
\vspace{0pt}
{University of California, Los Angeles}\\
\end{center}

\noindent
\parbox{3.5in}{\begin{flushleft}
{\underline{Correspondence Address}:}\\
{Behzad Razavi}\\
{56-147D, EIV}\\
{Electrical Engineering Department}\\
{University of California}\\
{Los Angeles, CA 90095-1594}\\
{Phone:  (310) 206-1633} \\
{Fax: (310) 206-8495}\\
{e-mail:  razavi@ee.ucla.edu}\\
\end{flushleft}} \hfill

%\usepackage{color}
%\usepackage{graphicx}

\begin{center}
\vspace{0.3in}
{\large \bf Abstract}
\end{center}
{\bf 
Keywords: }


%\pagebreak




\section{Introduction}
% The very first letter is a 2 line initial drop letter followed
% by the rest of the first word in caps.
% 
% form to use if the first word consists of a single letter:
% \IEEEPARstart{A}{demo} file is ....
% 
% form to use if you need the single drop letter followed by
% normal text (unknown if ever used by IEEE):
% \IEEEPARstart{A}{}demo file is ....
% 
% Some journals put the first two words in caps:
% \IEEEPARstart{T}{his demo} file is ....
% 
% Here we have the typical use of a "T" for an initial drop letter
% and "HIS" in caps to complete the first word.

In today's mobile devices, the WiFi transceiver still consumes a relatively large amount of power. 
The receiver power is substantial due to the greater on-time of the receiver compared to the transmitter.
Figure \ref{genericRX} shows a generic receiver system for 802.11a. 
\begin{figure}[htb]
\vspace{1.3in}
\caption{Generic 802.11a Receiver.}
\special{psfile=FIGS/genericRX hoffset=-50 voffset=15 vscale=82 hscale=82}
\label{genericRX}
\end{figure}
While advances in the art have considerably reduced the power consumption of analog-to-digital converters and frequency synthesizers, the main
receiver (RX) chain draws disproportionately high power. For example, we can now realize each ADC in Fig. \ref{genericRX} with about 3 mW of power and
the frequency synthesizer with about 5 mW. On the other hand, the receive path itself draws more than 45 mW \cite{Kan}. It is therefore desirable to
develop low-power RX front ends and baseband filters for WiFi applications. 

This paper introduces a complete 5-GHz CMOS receiver that meets the 11a sensitivity, blocking, and filtering requirements while consuming 11.6 mW.
This fourfold reduction in power is achieved through the use of a transformer as a low-noise amplifier (LNA), passive mixers, and ``non-invasive''
baseband filtering \cite{Zolfaghari}.

Section II elaborates on the design of the transformer and Section III introduces the receiver architecture.
Section IV analyzes the current-driven passive mixers in a general form which is used in Section V to design the mixers connected to the transformer. 
Section VI describes the baseband channel-select filters and Section VII presents the experimental results.


 
\section{Transformers as LNAs}
In the generic receiver of Fig. \ref{genericRX}, the low-noise amplifier (LNA) provides voltage gain and proper input matching, but dissipates
considerable amount of power. In order to realize a low-power receiver, we wish to remove the LNA or at least its power consumption. This becomes
possible if a passive device with zero power consumption can serve as an LNA. A 1-to-N on-chip transformer provides voltage gain at the cost of power
loss and noise figure degradation. It can also provide input matching as will be explained later. Thus, it seems that a transformer is a viable
substitute for the LNA. However, its design is not straightforward. The principal point is to achieve a high voltage gain with low power loss at the
frequency of interest. In particular, we would like to obtain a high coupling factor between the primary and the secondary. 
\begin{figure}[htb]
\vspace{4.6in}
\caption{Various transformer topologies.}
\special{psfile=FIGS/Transformer_topologies hoffset=-50 voffset=15 vscale=82 hscale=82}
\label{Transformer_topologies}
\end{figure}
Consider the stacked structure \cite{Zolfaghari2} shown in Fig. \ref{Transformer_topologies}(a) where spirals in high and low metal layers form the
secondary and one spiral in middle metal layer forms the primary. This structure has a high coupling factor but, the high capacitance between the
layers and the loss of the lower metal layers limit the performance. Note that the capacitance here sustains a larger voltage difference than in a
planar structure \cite{Long} shown in Fig. \ref{Transformer_topologies}(b). The self-resonance frequency and conductivity of the planar topology is
much higher, but the coupling between the primary and the inner turns of the secondary is quite low. The optimum topology is shown in Fig.
\ref{Transformer_topologies}(c) where a one-turn primary is stacked in the middle of the secondary. It is expected to have reasonable coupling, low
loss, and small capacitance to be able to work at 5-GHz band. To reduce the loss, an octagonal shape has been used as shown in Fig.
\ref{transformer}, with a one-turn primary in metal 8 and a six-turn secondary in metal 9.
\begin{figure}[htb]
\vspace{2.3in}
\special{psfile=FIGS/Transformer2D.eps hoffset=30 voffset=10 vscale=52 hscale=52}
\caption{Transformer geometry.}
\special{psfile=FIGS/Transformertext hoffset=-50 voffset=15 vscale=82 hscale=82}
\label{transformer}
\end{figure}
The outer diameters of the primary and the secondary are 146 $\mu$m and 170 $\mu$m, respectively. The two thick metal layers have been used to
minimize the loss. Since the secondary has already much more parasitics than the primary, it uses the top metal to minimize the capacitance to the
substrate. The number of turns, diameter, and metal widths are chosen to provide maximum voltage gain with acceptable loss at the desired
frequency. If we increase the number of secondary turns, the voltage gain grows slowly because the new turns are farther from the primary, but the
capacitance and loss increase significantly. Also, a higher number of turns reduces the transformer input resistance.  According to HFSS
simulations, the above transformer has an insertion loss of 2.4 dB at 5.5 GHz, an unloaded voltage gain of 13.4 dB, and an unloaded input resistance
of 87 $\Omega$. With a matched load connected to the secondary, the voltage gain and input resistance drop to 12 dB and 44 $\Omega$, respectively. We then
need to understand how the transformer performs when it is connected to the following block, namely mixers. 



\section{Receiver Architecture}
Figure \ref{architecture} shows the overall receiver architecture. 
\begin{figure}[htb]
\vspace{3.5in}
\caption{Receiver architecture.}
\special{psfile=FIGS/Architecture_corrected hoffset=-50 voffset=15 vscale=82 hscale=82}
\label{architecture}
\end{figure}
The transformer described in the previous section serves as the LNA and single-ended to differential converter at the input and drives two sets of
passive mixers with 25\% duty cycle LOs. The downconverted baseband signals are then applied to 4-th order elliptic filters for channel selection. A
side-benefit is that the transformer also provides ESD protection. The LO frequency arrives at twice the carrier frequency and is divided by two to
generate quadrature phases. We then have some logic to produce the 25\% LO waveforms.

We should highlight two advantages of our approach over the LNA-less receiver in \cite{Nauta}. First, the input matching inherent in our receiver
provides a robust interface with the antenna in the presence of long external traces. Second, our front end has much less power consumption.

There are three important questions that we need to address. First, how do the transformer and mixers provide input matching? Second, What is the
conversion gain of the mixers? It seems that the current is integrated in the load capacitor, $C_1$. In that case, the trans-impedance conversion gain
will not be flat across the channel bandwidth. Finally, how should we calculate the noise figure?
To answer these three important questions, a new analysis technique is required. We will study the input impedance, conversion gain and noise figure
and see why the trans-impedance conversion gain is flat across the channel.

By virtue of its high turns ratio, the transformer in Fig. \ref{architecture} exhibits a relatively high output impedance, approximating a current
source. The switches can therefore be viewed as current-driven mixers, thus contributing less noise than voltage-driven topologies \cite{Kaczman}.
Moreover, for the sake of analysis, since a capacitor is a good {\em keeper} of voltage, it is difficult to assume voltage excitations.




\section{Analysis of Current-Driven Passive Mixers} 

The 25\% duty cycle passive mixers have been used extensively since 2008 \cite{Blaakmeer}. But, their input impedance, conversion gain, and noise
figure have not been formulated accurately \cite{Mirzaei}-\cite{Andrews}. This section proposes an accurate analysis that helps the designer to insightfully
quantify the underlying principles. The method can be extended to other LO schemes as well.
\begin{figure}[htb]
\vspace{2.5in}
\special{psfile=FIGS/IQMixer hoffset=-50 voffset=0 vscale=82 hscale=82}
\caption{Current-driven passive mixer.}
\label{IQMixer}
\end{figure}
Figure \ref{IQMixer} shows the current-driven quadrature passive mixers with a source impedance $Z_s$. The circuit has
been redrawn in Fig. \ref{IQMixer2} for convenience where the LO waveforms controlling the switches, $S_1$--$S_4$, are also shown.
\begin{figure}[htb]
\vspace{3.2in}
\special{psfile=FIGS/IQMixer2 hoffset=-50 voffset=0 vscale=82 hscale=82}
\caption{Current-driven passive mixer and LO waveforms.}
\label{IQMixer2}
\end{figure}
For simplicity, let us assume that the switch resistance, $R_{sw}$, is zero. We will add its effect later. The in-phase output voltage can be
written as  
\beq
V_{out,I}(t)=\left[I_{mix} \times (S_1-S_3)\right] \ast 2 Z_L. 
\label{vouti} 
\eeq 
The mixer input current, $I_{mix}$, is chopped by $S_1$ and $S_3$ and convolved with the differential load impedance, $2 Z_L$. 
Similarly, the quadrature output voltage would be
\beq 
V_{out,Q}(t)=\left[I_{mix} \times (S_2-S_4)\right] \ast 2 Z_L. 
\label{voutq} 
\eeq 
The input voltage is equal to $V_{out,I}$ during $S_1$, -$V_{out,I}$ during $S_3$, $V_{out,Q}$ during $S_2$, and -$V_{out,Q}$ during $S_4$.
Thus,
\beq
V_{in}(t)= V_{out,I}(t) \times (S_1-S_3) + V_{out,Q}(t) \times (S_2-S_4). 
\label{vint1} 
\eeq 
Substituting (\ref{vouti}) and (\ref{voutq}) in (\ref{vint1}), yields
\ber
V_{in}(t) &=& \left\{[I_{mix} \times (S_1-S_3)] \ast 2 Z_L \right\} \times (S_1-S_3) \nonumber\\
&+& \left\{[I_{mix} \times (S_2-S_4)] \ast 2 Z_L \right\} \times (S_2-S_4).  
\label{vint2} 
\eer 
Taking the Fourier transform of (\ref{vint2}) leads to
\ber
V_{in}(f) &=& \left\{[I_{mix} \ast (S_1-S_3)] \times 2 Z_L \right\} \ast (S_1-S_3) \nonumber\\ 
&+& \left\{[I_{mix} \ast (S_2-S_4)] \times 2 Z_L \right\} \ast (S_2-S_4),  
\label{vinf1} 
\eer
which is the key equation to derive the input impedance. 
Figure \ref{S1S4} plots $(S_1-S_3)$ and $(S_2-S_4)$ in both time and frequency domains.
\begin{figure}[htb]
\vspace{3in}
\special{psfile=FIGS/S1S4 hoffset=-50 voffset=0 vscale=82 hscale=82}
\caption{$(S_1-S_3)$ and $(S_2-S_4)$ in time and frequency domains.}
\label{S1S4}
\end{figure}
Let us do a warm-up exercise and consider the case where the source impedance is infinite in Fig. \ref{IQMixer2}.

\subsection{Infinite Source Impedance}
Infinite source impedance, $Z_s$, yields $I_{mix}=I_{in}$. This simplifies the calculations as $I_{mix}$ would be single-tone.
\subsubsection{Input Impedance Calculation} 
To calculate the input impedance, we apply a tone at $f_{LO}+f_{IF}$ as shown in Fig. \ref{spectrumzmix} and would like to calculate the input voltage
at the same frequency. 
\begin{figure}[htb]
\vspace{1.9in}
\special{psfile=FIGS/spectrumzmix hoffset=-50 voffset=0 vscale=82 hscale=82}
\caption{A single-tone current applied to the mixer and the resultant input voltage spectrum.}
\label{spectrumzmix}
\end{figure}
Let us first focus on the positive-frequency impulse of
$I_{mix}$ and see how it will be shifted along the spectrum to finally reside at the same frequency. The only way is to shift up and down by the
same amount. For example, from the spectrum of $(S_1-S_3)$, $I_{mix}$ will be shifted up by $\sqrt 2 \delta (f-f_{LO})/\pi$ and down by  $\sqrt 2
\delta (f+f_{LO})/\pi$. Another way is to be shifted up by $\sqrt 2 \delta (f-f_{LO})/(3\pi)$ and down by $\sqrt 2 \delta (f+f_{LO})/(3\pi)$.
Also, the mixing mechanism can happen with first shifting down and then up. The same power would come from mixing with $(S_2-S_4)$. Note that
because the spectrum of $(S_2-S_4)$ is odd, one component would be multiplied by $+j$ and the other by $-j$, resulting a real positive value, the
same as that of mixing with $(S_1-S_3)$.
An interesting practice is to see if there would be any component at the image frequency, $f_{LO}-f_{IF}$ in the $V_{in}$ spectrum. The sum of the
two shifting frequencies has to be $\pm 2 f_{LO}$ in order to generate image. For example, the impulse at $f_{LO}+f_{IF}$ would be shifted down twice by
$\sqrt 2 \delta (f+f_{LO})/\pi$ to reside at $-f_{LO}+f_{IF}$. However, shifting down twice by  $+j \sqrt 2 \delta (f+f_{LO})/\pi$ generates the
same amplitude with a negative sign. Similarly, shifting down by $3 f_{LO}$ and up by $f_{LO}$ will not generate any image because the I and Q
branches cancel each other. It is instructive to see at what frequencies $V_{in}$ has components in response to a single tone that resides at
$f_{LO}+f_{IF}$ and $-f_{LO}-f_{IF}$. An interested reader can prove that those frequencies are $(f_{LO}+f_{IF}+4kf_{LO})$ and
$(-f_{LO}-f_{IF}+4kf_{LO})$, where $k$ varies from $-\infty$ to $+\infty$ (Fig. \ref{spectrumzmix}).

Knowing the frequency shifts that happen in (\ref{vinf1}) that contribute to the input voltage at $f_{LO}+f_{IF}$, we can write the mixer input
impedance, $Z_{mix}$, as
\beq
Z_{mix}(f)=\frac{8}{\pi^2}\left[Z_L(f \pm f_{LO})+\frac{1}{3^2}Z_L(f \pm 3f_{LO})+\frac{1}{5^2} Z_L(f \pm 5f_{LO}) + ... \right].
\label{zmix}
\eeq
If the load impedance is a capacitor, $1/(j 2 \pi C_L f)$, at the frequencies close to the LO frequency, $Z_L(f - f_{LO})$ dominates the summation in
(\ref{zmix}) and the input impedance would be approximately equal to
\beq
Z_{mix}|_{C_L}(f) \approx \frac{8}{\pi^2} Z_L(f - f_{LO}) = \frac{8}{\pi^2} \frac{1}{j 2 \pi C_L(f-f_{LO})}.
\label{zmixc}
\eeq
As another special case, if the load impedance is a resistor, $R_L$, the input impedance would be
\beq
Z_{mix}|_{R_L}(f)=\frac{16 R_L}{\pi^2}\left(1+\frac{1}{3^2}+\frac{1}{5^2}+...\right) = 2 R_L .
\label{zmixr}
\eeq
Since there is no memory in the system, the source would not recognize that the resistor is being switched and at any point of time the differential
input impedance is $2 R_L$.



\subsubsection{Conversion Gain Calculation} 
We are mostly interested in the trans-impedance conversion gain defined as the transfer function from the input current at $f_{LO}+f_{IF}$ to the
output voltage at $f_{IF}$. Since $I_{mix}$ is single-tone, only the fundamental of $(S_1-S_3)$ matters. Thus, from Eq. (\ref{vouti}) the
trans-impedance conversion gain, $A_R$, would be
\beq
A_R=\frac {2 \sqrt 2}{\pi} Z_L ~ \approx 0.9 Z_L.
\label{armix}
\eeq
Note that if $Z_L$ is a capacitor, $A_R$ is {\em not} constant across the channel bandwidth. 
Now, let us look at the voltage conversion gain, $A_V$, defined as the transfer function from the input voltage at $f_{LO}+f_{IF}$ to the output
voltage at $f_{IF}$. We have
\beq
A_V=\frac{V_{out,I}}{V_{in}}=\frac{V_{out,I}}{I_{in}} \times \frac{I_{in}}{V_{in}}=\frac{A_R}{Z_{mix}}.
\label{avar}
\eeq
Substituting (\ref{zmix}) and (\ref{armix}) in (\ref{avar}), we get
\beq
A_V=\frac{\frac {2 \sqrt 2}{\pi} Z_L}{\frac{8}{\pi^2}\left[Z_L(f \pm f_{LO})+\frac{1}{3^2}Z_L(f \pm 3f_{LO})+\frac{1}{5^2} Z_L(f \pm 5f_{LO}) + ... \right]}.
\label{avmix}
\eeq
We can simplify $A_V$ for the special cases of load capacitor and load resistor as
\beq
A_V|_{C_L} \approx \frac{\pi \sqrt 2}{4} \approx 0.9 \approx  0.9~{\rm dB}, 
\label{avmixc}
\eeq
\beq
A_V|_{R_L} = \frac{\sqrt 2}{\pi} \approx 0.45 \approx -6.9~{\rm dB}.
\label{avmixr}
\eeq
As expected, having a load capacitance is superior than a load resistance. Interestingly, $A_V|_{C_L}$ is greater than 1. It is not surprising,
because $V_{out}$ is the cause and $V_{in}$ is the effect here. The power of $V_{out}$ is only at $f_{IF}$. This power is then spread over the
specific harmonics at $V_{in}$. The same result can be achieved using Eq. (\ref{vint1}), where $V_{out}$ at $f_{IF}$ will be convolved with the
fundamental of $(S_1-S_3)$ and $(S_2-S_4)$ to form $V_{in}$ at $f_{LO}+f_{IF}$.


\subsubsection{Switch Resistance Effect}
\begin{figure}[htb]
\vspace{3in}
\special{psfile=FIGS/Rsw hoffset=-50 voffset=0 vscale=82 hscale=82}
\caption{Moving the switch resistance to the main path.}
\label{rsw}
\end{figure}
Because only one switch out of the four switches that share one side is on at a time, we can move the switch resistance to the main path as shown in
Fig. \ref{rsw}. 
Then, the switch resistance is in series with the input current source ($Z_s$ is infinite), and the input impedance would be
\beq
Z_{mix}(f)=2 R_{sw}+\frac{8}{\pi^2}\left[Z_L(f \pm f_{LO})+\frac{1}{3^2}Z_L(f \pm 3f_{LO})+\frac{1}{5^2} Z_L(f \pm 5f_{LO}) + ... \right] .
\label{zmixsw}
\eeq
The trans-impedance conversion gain does not change, but, the voltage conversion gain needs to be modified according to (\ref{zmixsw}) as
\beq
A_V=\frac{\frac {2 \sqrt 2}{\pi} Z_L}{2 R_{sw}+\frac{8}{\pi^2}\left[Z_L(f \pm f_{LO})+\frac{1}{3^2}Z_L(f \pm 3f_{LO})+\frac{1}{5^2} Z_L(f \pm 5f_{LO}) + ...
\right]}.
\label{avmixsw}
\eeq
Interested readers can simplify (\ref{zmixsw}) and (\ref{avmixsw}) for the special cases that we studied before.




\subsection{Finite Source Impedance}
With finite source impedance, $Z_s$, $I_{mix}$ is no longer equal to $I_{in}$. Instead,
\beq
I_{mix}=I_{in} - \frac{V_{in}}{Z_s}.
\label{imix}
\eeq
Although this might seem a small change to the case with infinite $Z_s$, the equations become much more complicated. Recall from section IV.A that
$V_{in}$ had components around odd harmonics. It can be shown that in the presence of $Z_s$, $V_{in}$ has the same frequency content but the
amplitudes needs to be modified. Note that (\ref{imix}) yields that $I_{mix}$ also has components around odd harmonics (Fig. \ref{spectrumzin}).
\begin{figure}[htb]
\vspace{3.3in}
\special{psfile=FIGS/spectrumzin hoffset=-50 voffset=0 vscale=82 hscale=82}
\caption{A single-tone current applied at the input and the resultant spectrum of the mixer input current and voltage.}
\label{spectrumzin}
\end{figure}

\subsubsection{Input Impedance Calculation} 
We can still use Eq. (\ref{vinf1}) to derive the input impedance. This time, the harmonics of $I_{mix}$ convolve with the harmonics of
$(S_1-S_3)$ and $(S_2-S_4)$, and fold to the baseband frequency. There are also other components around harmonics which we ignore after multiplication
by $Z_L$. Since the mixer is a downconverter, the load impedance, $Z_L$, would be small at high frequencies. In order to calculate the input
impedance, as shown in Fig. \ref{spectrumzin}, we assign $a_i$ and $b_i$ as the phasors of $I_{mix}$ and $V_{in}$ around $i$th harmonic and find their
values using (\ref{vinf1}) and (\ref{imix}). Since the amplitude of the tone in $I_{in}$ is unity, the input impedance is equal to $b_1$. Appendix I
finds the values of $a_i$ and $b_i$, yielding
\beq
Z_{in}(f)=\frac{\frac{8}{\pi^2} Z_L(f-f_{LO})}{1+\frac{8}{\pi^2} Z_L(f-f_{LO})\displaystyle{\SumAll k
\frac{1}{(4k+1)^2 Z_s(f+4k f_{LO})}}}.
\label{zin}
\eeq
If $Z_L$ is a capacitor, we can usually neglect $1$ in the denominator of (\ref{zin}) for reasonable values of $Z_s$. Thus,
\beq
\frac {1}{Z_{in}|_{C_L}(f)} \approx \SumAll k \frac{1}{(4k+1)^2 Z_s(f+4k f_{LO})}.
\label{zinc}
\eeq
The input impedance is the parallel combination of the scaled source impedance at certain frequencies and independent of the load impedance, $Z_L$.
Moreover, if $Z_s$ is a resistor, $R_s$, then 
\beq
Z_{in}|_{C_L,R_s}=\frac{8}{\pi^2}R_s \approx 0.81 R_s
\label{zincr}
\eeq


\subsubsection{Conversion Gain Calculation}
As mentioned earlier, we assume that $Z_L$ has a lowpass shape and ignore the harmonics at the output nodes.
The voltage conversion gain is the same as the case with infinite $Z_s$, while we do not include the switch resistance. Thus, for a capacitive load,
we have
\beq
A_V|_{C_L}=\frac{\pi \sqrt 2}{4} \approx 1.11 \approx 0.9~{\rm dB}.
\label{avinc}
\eeq
The trans-impedance conversion gain would be $A_V \times Z_{in}$ and equal to
\beq
A_R|_{C_L} = \frac{\pi \sqrt 2}{4} \div \SumAll k \frac{1}{(4k+1)^2 Z_s(f+4k f_{LO})},
\label{arinc}
\eeq
independent of $C_L$ and $f_{IF}$.


\subsubsection{Switch Resistance Effect}
If we add the switch resistance, the input voltage is equal to the one in Eq. (\ref{vinf1}) plus $2 R_{sw}I_{mix}$. Thus, 
\ber
V_{in}(f) &=& \left\{[I_{mix} \ast (S_1-S_3)] \times 2 Z_L \right\} \ast (S_1-S_3) \nonumber\\ 
&+& \left\{[I_{mix} \ast (S_2-S_4)] \times 2 Z_L \right\} \ast (S_2-S_4) + 2R_{sw}I_{mix}.  
\label{vinf1sw} 
\eer
Equation (\ref{imix}) is still valid and along with (\ref{vinf1sw}) derives the input impedance. Note that we could assign $a_{sw,i}$ and $b_{sw,i}$
as the phasors of $I_{mix}$ and $V_{in}$ around $i$th harmonic and find their values similar to Appendix I. However, finding the solution is more
difficult in this case. We wish to perform some transforms so that we can utilize the equations that we already have. This is done by the
Norton-Thevenin-Norton conversion shown in Fig. \ref{ntn}. 
\begin{figure}[htb]
\vspace{1.7in}
\special{psfile=FIGS/ntn hoffset=-20 voffset=0 vscale=82 hscale=82}
\caption{Norton-Thevenin-Norton Conversion.}
\label{ntn}
\end{figure}
The circuit in Fig. \ref{ntn}(c) is similar to the case without switch resistance, and we
can easily find $V_{mix}$ as the input current times the input impedance seen by the source, i. e., 
\beq
V_{mix}(f)=I_{in}\frac{Z_s}{Z_s+2R_{sw}} \times \frac{\frac{8}{\pi^2} Z_L(f-f_{LO})}{1+\frac{8}{\pi^2} Z_L(f-f_{LO})\displaystyle{\SumAll k
\frac{1}{(4k+1)^2 [Z_s(f+4k f_{LO})+2R_{sw}]}}}.
\label{vmix}
\eeq
Then from the original circuit in Fig. \ref{ntn}(a), $V_{in}=V_{mix}+2R_{sw}I_{mix}$. Replacing $I_{mix}$ with $I_{in}-V_{in}/Z_s$, and using
(\ref{vmix}), we find the input impedance as 
\ber
&&\!\!\!\!\!\!\!\!\!\!\!\!\!\!\!\!\!\!\!\!\!\!Z_{in}(f)=Z_s||2R_{sw}+ \nonumber\\
&& (\frac{Z_s}{Z_s+2R_{sw}})^2 \times \frac{\frac{8}{\pi^2} Z_L(f-f_{LO})}{1+\frac{8}{\pi^2} Z_L(f-f_{LO})\displaystyle{\SumAll k
\frac{1}{(4k+1)^2 [Z_s(f+4k f_{LO})+2R_{sw}]}}}.
\label{zinsw}
\eer
For a load capacitance and reasonable source impedance, we can usually ignore $1$ in the denominator of (\ref{zinsw}) and write
\beq
Z_{in}|_{C_L}(f)=Z_s||2R_{sw}+ (\frac{Z_s}{Z_s+2R_{sw}})^2 \div \displaystyle{\SumAll k \frac{1}{(4k+1)^2 [Z_s(f+4k f_{LO})+2R_{sw}]}}.
\label{zinswc}
\eeq
Again, we see that the input impedance is independent of the load impedance under certain conditions.
Now, we wish to find the voltage conversion gain, $A_V$. 
We can write
\beq
V_{in}(f)=\frac {2 \sqrt 2}{\pi} V_{out,I}(f-f_{LO}) + 2R_{sw}I_{mix}(f),
\label{vinfsw}
\eeq
where only low-frequency component of $V_{out,I}$ has been taken into account. Using Eq. (\ref{imix}) and the fact that $I_{in}=V_{in}/Z_{in}$, we get
\beq
I_{mix}=\frac {Z_s - Z_{in}}{Z_s Z_{in}} V_{in}
\label{imix2}
\eeq
Solving (\ref{vinfsw}) and (\ref{imix2}) yields
\beq
A_V|_{C_L}=\frac{\pi \sqrt 2}{4}\left(1-2R_{sw}\frac {Z_s - Z_{in}}{Z_s Z_{in}}\right)
\label{avincsw}
\eeq
The trans-impedance conversion gain, $A_R$, is equal to $A_V \times Z_{in}$, and can be found using (\ref{zinswc}) and (\ref{avincsw}).
Interestingly, $A_R$ is independent of $C_L$ and $f_{IF}$. The latter is very important because otherwise, we would not have flat response across the
channel.


\subsection{Noise Figure Calculation}
With infinite $Z_s$, the switch resistance and hence its noise is in series with the input current source. Thus, it does not contribute noise. 
Now, let us calculate the noise figure with finite purely resistive $Z_s$ equal to $R_s$. We include the switch resistance, but assume it noiseless
first. We add the effect of switch resistance noise later. We also assume that the load impedance is a capacitor and ignore the higher harmonics at the
output. If $I_{in}$ is the rms value of the input signal, the signal-to-noise ratio (SNR) at the input would
be
\beq
SNR_{in}=\frac{I_{in}^2}{\displaystyle{{4kT\frac{1}{R_s}}}}.
\label{snrin}
\eeq
The signal and noise around $f_{LO}$ will be downconverted at the output nodes with the same trans-impedance conversion gain, $A_R$. However, the noise
around $i$th harmonics of $f_{LO}$ will also fold on the signal with a conversion gain of $A_R/i$. Thus, the output SNR would be
\beq
SNR_{out}=\frac{I_{in}^2 A_R^2}{\displaystyle{{4kT\frac{1}{R_s}}}A_R^2 \left(1+\frac{1}{9}+\frac{1}{25}+...\right)},
\label{snrout}
\eeq
where we have assumed the signal is double-sideband (DSB).
By definition, the DSB noise figure is
\beq
NF=\frac{SNR_{in}}{SNR_{out}}=(1+\frac{1}{9}+\frac{1}{25}+...)=\frac{\pi^2}{8} \approx 0.9~{\rm dB}.
\label{nf}
\eeq 
Now, let us add the switch resistance noise. For noise analysis purpose, we remove the input signal source and notice that the switch resistance is in
series with $R_s$. Therefore, the noise figure will be degraded by $(1+2R_{sw}/R_s)$, i.e., 
\beq
NF=\frac{\pi^2}{8} (1+\frac{2R_{sw}}{R_s}).
\label{nftot}
\eeq 
Now, let us consider a general source impedance. We still assume that at the vicinity of $f_{LO}$, $Z_s=R_s$ and the noise current source is $4kT/R_s$.
We use the converted model in Fig. \ref{ntn}(c). 
The input signal power is $I_{in}^2Z_s^2/(Z_s+2R_{sw})^2$, and the input noise current would be
\beq
\overline{i_{n}^2}=\frac{4kT Z_s^2}{R_s(Z_s+2R_{sw})^2}+\frac{4kT 2R_{sw}}{(Z_s+2R_{sw})^2}.
\eeq
Then the output signal is simply
\beq
V_{out}^2=\frac{Z_s(f_{LO})^2}{[Z_s(f_{LO})+2R_{sw}]^2}I_{in}^2 A_R^2,
\eeq
but for the noise we have to consider the noise around harmonics as well. It folllows that
\ber
\overline{V_{n,out}^2}&=&\left\{\frac{4kT Z_s(f_{LO})^2}{R_s[Z_s(f_{LO})+2R_{sw}]^2}+\frac{4kT 2R_{sw}}{[Z_s(f_{LO})+2R_{sw}]^2}\right\}A_R^2 \nonumber\\
&+&\left\{\frac{4kT Z_s(3f_{LO})^2}{R_s[Z_s(3f_{LO})+2R_{sw}]^2}+\frac{4kT 2R_{sw}}{[Z_s(3f_{LO})+2R_{sw}]^2}\right\}\frac{A_R^2}{9} \nonumber\\
&+&\left\{\frac{4kT Z_s(5f_{LO})^2}{R_s[Z_s(5f_{LO})+2R_{sw}]^2}+\frac{4kT 2R_{sw}}{[Z_s(5f_{LO})+2R_{sw}]^2}\right\}\frac{A_R^2}{25} + ...~.
\eer
Thus, we have
\beq
\frac{1}{SNR_{out}}=\frac{\displaystyle{{4kT\frac{1}{R_s}}}}{I_{in}^2} \frac{[Z_s(f_{LO})+2R_{sw}]^2}{Z_s(f_{LO})^2}
\sum_{k=0}^{+\infty} \frac{1}{(2k+1)^2} \left[ \frac {Z_s[(2k+1)f_{LO}]^2 + 2R_{sw}R_s} {\{Z_s[(2k+1)f_{LO}]+2R_{sw}\}^2} \right].
\label{1snrout}
\eeq
Using (\ref{snrin}) and (\ref{1snrout}), we can write the noise figure as
\beq
NF=\frac{[Z_s(f_{LO})+2R_{sw}]^2}{Z_s(f_{LO})^2}
\sum_{k=0}^{+\infty} \frac{1}{(2k+1)^2} \left[ \frac {Z_s[(2k+1)f_{LO}]^2+2R_{sw}R_s} {\{Z_s[(2k+1)f_{LO}]+2R_{sw}\}^2} \right].
\label{nfgeneral}
\eeq
Note that (\ref{nfgeneral}) simplifies to (\ref{nftot}) if $Z_s=R_s$. If $Z_s$ is an RLC tank resonating at $f_{LO}$ with a parallel resistor $R_s$, we
can neglect the tank impedance at the harmonic frequencies with respect to $R_{sw}$. Thus, (\ref{nfgeneral}) simplifies to 
\beq
NF|_{RLC}=1+\frac{2R_{sw}}{R_s}+(\frac{\pi^2}{8}-1)\frac{(R_S+2R_{sw})^2}{2R_{sw}R_s}.
\label{nfrlc}
\eeq
Note that if we do not neglect the tank impedance at the harmonics, the noise figure would be better than the one in (\ref{nfrlc}).
An interesting point is that reducing $R_{sw}$ does not necessarily reduce the noise figure. In fact, with an RLC tank as the source resistance, the
optimum switch resistance is
\beq
R_{sw, opt}=\frac{\sqrt{\pi^2-8}}{2\pi}R_s \approx 0.218R_s.
\label{rswopt}
\eeq
Plugging this optimum value of $R_{sw}$ in (\ref{nfrlc}) derives the minimum noise figure of the mixer as $2.54 \approx 4.05$ dB.



\section{Transformer-Mixer Design}
Now that we know how to analyze the 25\% duty cycle mixers, we get back to the design of our receiver front end. The transformer is designed and
optimized first to have maximum voltage gain and low power loss at the 5-GHz band. Then, looking through the secondary of the transformer, we build
the Norton equivalent circuit of the antenna-transformer cascade over a wide bandwidth as shown in Fig. \ref{mixer}.
\begin{figure}[htb]
\vspace{3.7in}
\special{psfile=FIGS/Mixer hoffset=-50 voffset=5 vscale=82 hscale=82}
\caption{Transformer-mixer interface.}
\label{mixer}
\end{figure}
Our transformer guarantees that if a load impedance of 800 $\Omega$ is attached to its secondary, then the input impedance seen from the primary is about
50 $\Omega$.  In other words, since $Z_s$ is equal to 800 $\Omega$ at the carrier frequency, for proper matching the input impedance of the mixer needs to be
800 $\Omega$ too. However, we know that for current-driven mixers, the input impedance depends on the source impedance itself. To avoid confusion, we
refer to the impedance seen by $I_{in}$ denoted by $Z_{in}$. Since $I_{in}$ is ideal, this resistance is unique and independent of $I_{in}$. Thus, the
combination of antenna-transformer and the mixers must have a secondary-referred input impedance of 400 $\Omega$. This composite input impedance is
calculated in (\ref{zinswc}) and repeated here as
\beq
Z_{in}(f)=Z_s||2R_{sw}+ (\frac{Z_s}{Z_s+2R_{sw}})^2 \div \displaystyle{\SumAll k \frac{1}{(4k+1)^2 [Z_s(f+4k f_{LO})+2R_{sw}]}}.
\label{zinswcrepeat}
\eeq
Due to the bandpass nature of $Z_s$, the summation on the right-hand side must be carried out for about 14 terms. Ideally, in the range of 5 to 6 GHz,
we must have $Re\{Z_{in}(f)\} \approx Z_s(f)/2 \approx 400$ $\Omega$ and $Im\{Z_{in}(f)\} \approx 0$. Thus, Eq. (\ref{zinswcrepeat})
yields $R_{sw}=57~\Omega$ corresponding to  $W/L=$10 $\mu$m/ 60 nm for the switches. The LO buffers driving eight such switches draw a total power of
$fCV_{DD}^2 \approx$ 0.4 mW at 6 GHz.
If $R_{sw} \ll Z_{in}(f)$, Eq. (\ref{zinswcrepeat}) can further be simplified to
\beq
\frac{1}{Z_{in}(f)}=\SumAll k \frac{1}{(4k+1)^2[Z_s(f+4kf_{LO})+R_{sw}]}.
\label{zinsimple}
\eeq
Equation \ref{zinsimple} reveals that the source impedance, $Z_s$, around integer multiples of $4 f_{LO}$ in series with the switch resistance
is scaled and put in parallel to form the composite input impedance, $Z_{in}$. In other words, since the IF port of the mixer has high impedance, the
impedance matching is achieved by folding of the source impedance itself.



Simulations indicate that the ``zero-power'' RF front end consisting of the transformer and the mixers exhibits a noise figure of 4.5 dB, a voltage
gain of 12 dB, an input $P_{1dB}$ of $-$5.2 dBm, and an $S_{11}$ of better than $-$12 dB across the 5 GHz band. For a target receiver NF of less than
6 dB, all of the subsequent stages must contribute no more than 1.5 dB, demanding additional circuit techniques to build low-noise yet linear baseband
filters with low power dissipation.

\section{Filter Design}
In the 11a standard, for the lowest data rate of 6 Mb/s, the adjacent and alternate adjacent channels can be higher than the desired channel by 16 dB
and 32 dB, respectively. For the highest data rate of 54 Mb/s, the maximum interferer power levels are reduced by 17 dB, relaxing the filtering
requirements. However, we design the baseband filters for the worst case which requires a sharp roll-off to reduce these channels to well below the
desired signal level $-$ unless the baseband ADCs offer a dynamic range wide enough and a sampling rate high enough to handle partially-attenuated
blockers. 
Because in OFDM, each subcarrier has a narrow bandwidth, the phase response can be assumed linear across each sub-channel. Thus, the phase response
of the filter is not critical suggesting that an elliptic filter implementation is acceptable. A fifth-order elliptic filter is sufficient for our
purpose and can be realized by cascading two biquadratic transfer functions and a single RC pole. Let us focus on designing a biquad section. 

\begin{figure}[htb]
\vspace{3.6in}
\special{psfile=FIGS/Biquad hoffset=-50 voffset=0 vscale=82 hscale=82}
\caption{(a) Conventional implementation of a biquad, (b) basic idea of noninvasive filtering, and (c) noninvasive implementation of a biquad.}
\label{biquad}
\end{figure}
Figure \ref{biquad}(a) shows a conventional implementation of such a transfer function using three $G_m$ stages.  
In order to study the noise behavior, we assume that each $G_m$ has an input-referred noise voltage and calculate its transfer function to the output.
The transfer function for the noise of $G_{m1}$ and $G_{m2}$ is proportional to $(s^2+\omega_z^2)/(s^2+as+b)$ and $1/(s^2+as+b)$, respectively, which
are both low-pass. Therefore, their noise is not attenuated in the signal band.
The noise of $G_{m3}$, however, is attenuated through the bandpass shaping function, $s/(s^2+as+b)$. 
The circuit also experiences nonlinearity because the signal and interferer are both amplified by $G_{m1}R_1$. As a result, $G_{m1}$ compresses at its
output and $G_{m2}$ compresses at its input. Another issue is that this architecture needs 4-input $G_m$ cells for fully-differential implementation.
All of these issues are mitigated in the noninvasive filtering architecture. 
The basic idea is illustrated in Fig. \ref{biquad}(b), where a notch impedance similar to an LC trap is placed at the output of a $G_m$ stage
\cite{Zolfaghari}. The notch impedance, Z, is high in the signal band and low at the interferer frequency. The filter transfer function follows the
shape of Z and shunts the interferer to ground. 
The actual implementation is shown in Fig. \ref{biquad}(c) where $G_{m3}$ and $G_{m4}$ form a gyrator that transforms $C_3$ to an emulated inductor,
$L_e$. $L_e$ resonates with $C_2$ and generates a low-impedance path to ground at the interferer frequency. 
Let us study the noise behavior. The transfer function for the noise of $G_{m1}$ is proportional to $(s^2+\omega_z^2)/(s^2+as+b)$ and does not
attenuate it in the signal band. But, the noise of $G_{m2}$ and $G_{m4}$ is highpass-filtered by $s^2/(s^2+as+b)$ and can be made negligible. The
noise of $G_{m3}$ is also attenuated by the bandpass shaping function of $s/(s^2+as+b)$. 
In summary, compared to the previous case which had two unattenuated noise sources, this case has only one, and we expect to see lower noise.  
This topology is also more linear since the signal is amplified by $G_{m1}R_1$, but the interferer is attenuated. As a result, $G_{m1}$ does not
compress at its output. Note that the interferer is amplified at the output of $G_{m2}$. This can cause $G_{m2}$ and $G_{m4}$ to compress at their
outputs and $G_{m3}$ at its input. The important point here is that the compression of $G_{m2}$, $G_{m3}$ and $G_{m4}$ only shifts the notch frequency
but it does not affect the transfer function in the signal band.\footnote{There is a second-order effect that slightly changes the gain in the signal
band and is verified by the measurement results in Section VII.} Because all the $G_m$ cells have one grounded input, this architecture readily lends
itself to differential implementation.




\begin{figure}[htb]
\vspace{4in}
\special{psfile=FIGS/Filter hoffset=-50 voffset=0 vscale=82 hscale=82}
\caption{(a) Fourth-order elliptic low-pass filter, and (b) $G_m$ implementations.}
\label{filter}
\end{figure}
Figure \ref{filter}(a) shows the realization of the fourth-order elliptic filter.  A first-order RC filter is then added on the PCB to reach the
fifth order. In this design, $G_{m1}$ and $G_{m2}$ are made variable to provide gain control to accommodate the wide range of input powers specified
by 11a. Figure \ref{filter}(b) summarizes the $G_m$ implementations. $G_{m1}$ has a PMOS input for low flicker noise and to work
properly with an input common-mode level of 400 mV. Gain control is achieved by changing the source degeneration resistance of the input pair. Each
branch dissipates 720 uA from a 1-V supply.  $G_{m2}$ has an NMOS input and consumes less power as its noise is not as important as that of
$G_{m1}$. $G_{m3}$ to $G_{m8}$ are identical simple differential pairs.  The fourth-order filter exhibits an input-referred noise voltage of 2
n$\rm{V}/\sqrt{\rm{Hz}}$ at 5 MHz, an in-channel $IIP_3$ of 193 m${\rm{V}}_{rms}$ and a voltage gain of 39 dB while consuming 4.3 mW. The filter
voltage gain is programmable in steps of 2 to 3 dB for a total range of 43 dB.


\section{Experimental Results}
The receiver of Fig. \ref{architecture} has been fabricated in 65-nm digital CMOS technology. Figure \ref{die} shows the die photograph.
\begin{figure}[htb]
\vspace{1.025in}
\special{psfile=FIGS/die.eps hoffset=2 voffset=0 vscale=72 hscale=72}
\caption{Die photograph.}
\label{die}
\end{figure}
The RF section occupies 350 $\mu$m $\times$ 240 $\mu$m and the baseband section 450 $\mu$m $\times$ 220 $\mu$m.\footnote{Due to limited silicon area,
the receiver layout is decomposed and placed within other unrelated circuits, but all of the connections are present on the chip.}
The die is bonded to a printed-circuit board and uses a 1-V supply for the main circuits, a 1.2-V source for the open-drain output drivers, and a
400-mV supply connected to the center tap of the transformer.



Figure \ref{nf} plots the measured noise figure of the complete receiver as a function of the baseband frequency. The average noise figure is about
5.3 dB.
\begin{figure}[htb]
\vspace{2.4507in}
\special{psfile=SIMS/nf.eps hoffset=-50 voffset=-140 vscale=58 hscale=58}
\caption{Measured noise figure.}
\label{nf}
\end{figure}
The sensitivity of the receiver is measured with the aid of Agilent's N5182 MXG vector signal generator and N9020A MXA signal analyzer, which
respectively apply a 64-QAM 802.11a signal and sense the baseband outputs to construct the signal constellation. Figure \ref{constel} shows the
results for a $-65$-dBm 5.7-GHz input at 54 MB/s.
\begin{figure}[htb]
\vspace{2.6in}
\special{psfile=SIMS/65dBm3_0002.eps hoffset=0 voffset=0 vscale=28 hscale=28}
\caption{Measured EVM at $P_{in}=-65$ dBm.}
\label{constel}
\end{figure}
The error vector magnitude (EVM) is equal to $-28$ dB, exceeding the 11a specification by 5 dB, suggesting that the receiver sensitivity would be 5 dB
better. As expected, the sensitivity was measured to be $-$70 dBm with an EVM of $-$23.4 dB.

Figure \ref{s11} plots the $S_{11}$ from 5 to 6 GHz, measured at each input frequency, while the mixers switch at the corresponding LO frequency.
\begin{figure}[htb]
\vspace{2.6in}
\special{psfile=SIMS/s11_2.eps hoffset=-50 voffset=-140 vscale=58 hscale=58}
\caption{Measured input return loss.}
\label{s11}
\end{figure}
It is expected that a slightly larger transformer or adding more capacitance can yield $S_{11}<-10$ dB across the band.
\begin{figure}[htb]
\vspace{2.6in}
\special{psfile=SIMS/sensitivity.eps hoffset=-50 voffset=-140 vscale=58 hscale=58}
\caption{Measured sensitivity.}
\label{sensitivity}
\end{figure}
Figure \ref{sensitivity} shows the sensitivity of the receiver from 5 to 6 GHz. The sensitivity degrades a little at lower frequencies again because the
transformer is mistuned. 
\begin{figure}[htb]
\vspace{2.6in}
\special{psfile=SIMS/gainvsf.eps hoffset=-50 voffset=-140 vscale=58 hscale=58}
\caption{Measured passband gain.}
\label{gainvsf}
\end{figure}
The receiver gain across the band in Fig. \ref{gainvsf} also confirms the transformer mistuning.

\begin{figure}[htb]
\vspace{2.55in}
\special{psfile=SIMS/gain.eps hoffset=-50 voffset=-140 vscale=58 hscale=58}
\caption{Measured receiver transfer function.}
\label{tf}
\end{figure}
Figure \ref{tf} plots the measured receiver transfer function, revealing a passband peaking of 1 dB and a rejection of 22 dB at 20 MHz and 43 dB at
40 MHz.\footnote{In this measurement a first-order RC section follows each output on the PCB.}
Owing to the finite output resistance of the $G_m$ cells, the filter does not exhibit the deep notches that are characteristic of elliptic transfer
functions.
The performance of the baseband filter is ultimately tested when a large blocker accompanies a small desired signal. In such a case, the filter must
remain sufficiently selective and linear so that the desired signal does not experience compression. Figure \ref{gainblocker} plots the measured
passband gain as a function of the power of an RF blocker in the adjacent or alternate adjacent channel.
\begin{figure}[htb]
\vspace{2.65in}
\special{psfile=SIMS/gainblocker.eps hoffset=-50 voffset=-140 vscale=58 hscale=58}
\caption{Measured passband gain in the presence of a blocker.}
\vspace{0.1in}
\label{gainblocker}
\end{figure}
Note that this measurement is done with the maximum receiver gain which corresponds to the signal levels around the sensitivity, e.g. $-$60 dBm. The
maximum adjacent channel interferer would be 1 dB lower or $-$61 dBm for which the variation of gain is negligible. Even with the maximum 11a power of
$-$30 dBm, the gain variation is only about 1 dB. The peaking of gain with large adjacent-channel interferer is also observed in simulations and is
due to the compression in the notch-impedance $G_m$ cells. For the alternate-adjacent-channel interferer the gain variation is much less and totally
negligible even upto $-$30 dBm. 

The filter nonlinearity resulting from a blocker may also corrupt the 11a 64-QAM OFDM signal by creating cross modulation among the sub-channels.
This effect is characterized by setting the RF input signal level 3 dB above the sensitivity, applying a blocker, and raising its level until the EVM
falls to $-$23 dB. Figure \ref{evmint} plots the relative blocker level as a function of the frequency offset with respect to the desired signal
center frequency.
\begin{figure}[htb]
\vspace{2.65in}
\special{psfile=SIMS/interferer_rejection.eps hoffset=-50 voffset=-140 vscale=58 hscale=58}
\caption{Measured interferer rejection.}
\vspace{0.1in}
\label{evmint}
\end{figure}

Figure \ref{evmpin} shows the measured EVM and the corresponding passband gain versus input power. As the input signal power increases, a lower gain
is chosen for the variable-gain $G_m$ cells. For small input powers EVM is noise-limited and improves with the input power, while at large input powers
EVM is nonlinearity-limited and will eventually go up. The variable gain range is such that even with maximum 11a signal of $-$30 dBm, the EVM is not
degraded due to nonlinearity.
\begin{figure}[htb]
\vspace{2.65in}
\special{psfile=SIMS/evm_AGC.eps hoffset=-50 voffset=-140 vscale=58 hscale=58}
\caption{Measured EVM and passband gain versus input power.}
\vspace{0.1in}
\label{evmpin}
\end{figure}

Table 1 summarizes the receiver performance and compares it to that of prior art. This work has reduced the power consumption by about a factor of 4,
while demonstrating a sensitivity of $-$70 dBm. 
\begin{table}[htb]
\caption{Comparison with state-of-the-art.}
\vspace{3.05in}
\special{psfile=FIGS/table hoffset=-50 voffset=0 vscale=82 hscale=82}
\label{table}
\end{table}

\section{Conclusion}
This paper suggests the use of transformers in place of active LNAs to save power and provide ESD protection. The ``zero-power'' front end consisting
of a transformer followed by passive mixers has reasonable performance and combined with noninvasive filtering, exceeds the 11a requirements while
consuming only 11.6 mW.
The proposed analysis of current-driven passive mixers provides insight into their properties.


\begin{center} {\large Acknowledgment}
\end{center}
The authors wish to thank TSMC's University Shuttle Program for chip fabrication.
This work was supported by Realtek Semiconductor.


\begin{thebibliography}{9}

\bibitem{Kan} L. L. L. Kan et al., ``A 1-V 86-mW RX 53-mW TX single-chip CMOS transceiver for WLAN IEEE 802.11a,'' {\em IEEE J. Solid-State Circuits,} 
vol. 42, no. 9, pp. 1986--1998, Sep. 2007.

\bibitem{Zolfaghari} A. Zolfaghari and B. Razavi, ``A low-power 2.4-GHz transmitter/receiver CMOS IC,'' {\em IEEE J. Solid-State Circuits,} vol. 38,
no. 2, pp. 176--183, Feb. 2003.

\bibitem{Zolfaghari2} A. Zolfaghari, A. Chan, and B. Razavi, ``Stacked inductors and transformers in CMOS technology,'' {\em IEEE J. Solid-State Circuits,} vol.
36, no. 4, pp. 620--628, Apr. 2001.

\bibitem{Long} J. R. Long, ``Monolithic transformers for silicon RF IC design,'' {\em IEEE J. Solid-State Circuits,} vol.
35, no. 9, pp. 1368--1383, Sept. 2000.

\bibitem{Nauta} M. Soer, et al., ``A 0.2-to-2.0GHz 65nm CMOS receiver without LNA achieving $>$11dBm IIP3 and $<$6.5 dB NF,'' {\em ISSCC Dig.
of Tech. Papers,} pp. 222--223, Feb. 2009.

\bibitem{Kaczman} D. Kaczman, et al., ``A single-chip 10-band WCDMA/HSDPA 4-band GSM/EDGE SAW-less CMOS receiver with DigRF 3G interface
and +90 dBm IIP2,'' {\em IEEE J. Solid-State Circuits,} vol. 44, no. 3, pp. 718--739, Mar. 2009.

\bibitem{Blaakmeer} S. C. Blaakmeer, et al., ``The BLIXER, a wideband balun-LNA-I/Q-mixer topology,'' {\em IEEE J. Solid-State Circuits,} vol. 43,
no. 12, pp. 2706--2715, Dec. 2008.

\bibitem{Mirzaei} A. Mirzaei and H. Darabi, ``Analysis of imperfections on performance of 4-phase passive-mixer-based high-Q bandpass filters in SAW-less
receivers,'' {\em IEEE Trans. Circuits Syst. I, Reg. Papers,} vol 58, no. 5, pp. 879--892, May 2011.

\bibitem{Andrews} C. Andrews and A. C. Molnar, ``A passive mixer-first receiver with digitally controlled and widely tunable RF interface,'' {\em
IEEE J. Solid-State Circuits,} vol 45, no. 12, pp. 2696--2708, Dec. 2010. 

\bibitem{Maeda} T. Maeda et al., ``Low-power-consumption direct-conversion CMOS transceiver for multi-standard 5-GHz wireless LAN systems with channel
bandwidths of 5-20 MHz,'' {\em IEEE J. Solid-State Circuits,} 
vol. 41, no. 2, pp. 375--383, Feb. 2006.

\bibitem{Lim} K. Lim et al., ``A 2x2 MIMO tri-band dual-mode direct-conversion CMOS transceiver for worldwide WiMAX/WLAN applications,'' {\em IEEE J. Solid-State
Circuits,} vol. 46, no. 7, pp. 1648--1658, Jul. 2011.

%\bibitem{Valla} M. Valla et al., ``A 72-mW CMOS 802.11a direct conversion front-end with 3.5-dB NF and 200-KHz 1/f noise corner,'' {\em IEEE J. Solid-State Circuits,} 
%vol. 40, no. 4, pp. 970--977, Apr. 2005.


\end{thebibliography}

\end{document}


















