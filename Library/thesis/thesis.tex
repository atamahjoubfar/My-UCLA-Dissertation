
\documentclass [PhD] {uclathes}

\usepackage{amsmath,epsfig,graphics,graphicx,verbatim}
\usepackage{amsmath,amssymb,array,graphicx,boxedminipage,mathdots,color,epsfig}
\usepackage{rotating,longtable}
\usepackage{amsmath,epsfig,graphics,graphicx,verbatim}
\usepackage{multirow}
\usepackage{subfigure}
\usepackage{graphicx}
\usepackage{units}
\usepackage{rotating}
\newcommand{\hvecf}{{\mathbf h}}
\newcommand{\bvecf}{{\mathbf b}}
\newcommand{\pvecf}{{\mathbf p}}
\newcommand{\kvecf}{{\mathbf k}}
\newcommand{\wvecf}{{\mathbf w}}
\newcommand{\nvecf}{{\mathbf n}}
\newcommand{\qvecf}{{\mathbf q}}
\newcommand{\yvecf}{{\mathbf y}}
\newcommand{\cvecf}{{\mathbf c}}
\newcommand{\zvecf}{{\mathbf z}}
\newcommand{\uvecf}{{\mathbf u}}
\newcommand{\vvecf}{{\mathbf v}}
\newcommand{\mvecf}{{\mathbf m}}
\newcommand{\avecf}{{\mathbf a}}
\newcommand{\lvecf}{{\mathbf l}}
\newcommand{\xvecf}{{\mathbf x}}
\newcommand{\Gvecf}{{\mathbf G}}
\newcommand{\dvecf}{{\mathbf d}}
\newcommand{\evecf}{{\mathbf e}}
\newcommand{\tvecf}{{\mathbf t}}
\newcommand{\svecf}{{\mathbf s}}
\newcommand{\rvecf}{{\mathbf r}}
\newcommand{\gvecf}{{\mathbf g}}
%\newcommand{\hvec}{{\mathrm h}}
%\newcommand{\bvec}{{\mathrm b}}
%\newcommand{\pvec}{{\mathrm p}}
%\newcommand{\kvec}{{\mathrm k}}
%\newcommand{\wvec}{{\mathrm w}}
%\newcommand{\nvec}{{\mathrm n}}
%\newcommand{\qvec}{{\mathrm q}}
%\newcommand{\yvec}{{\mathrm y}}
%\newcommand{\cvec}{{\mathrm c}}
%\newcommand{\zvec}{{\mathrm z}}
%\newcommand{\uvec}{{\mathrm u}}
%\newcommand{\vvec}{{\mathrm v}}
%\newcommand{\mvec}{{\mathrm m}}
%\newcommand{\avec}{{\mathrm a}}
%\newcommand{\lvec}{{\mathrm l}}
%\newcommand{\xvec}{{\mathrm x}}
%\newcommand{\dvec}{{\mathrm d}}
%\newcommand{\evec}{{\mathrm e}}
%\newcommand{\tvec}{{\mathrm t}}
%\newcommand{\svec}{{\mathrm s}}
%\newcommand{\rvec}{{\mathrm r}}
%\newcommand{\gvec}{{\mathrm g}}

\newcommand{\hvec}{{h}}
\newcommand{\bvec}{{b}}
\newcommand{\pvec}{{p}}
\newcommand{\kvec}{{k}}
\newcommand{\wvec}{{w}}
\newcommand{\nvec}{{n}}
\newcommand{\qvec}{{q}}
\newcommand{\yvec}{{y}}
\newcommand{\cvec}{{c}}
\newcommand{\zvec}{{z}}
\newcommand{\uvec}{{u}}
\newcommand{\vvec}{{v}}
\newcommand{\mvec}{{m}}
\newcommand{\avec}{{a}}
\newcommand{\lvec}{{l}}
\newcommand{\xvec}{{x}}
\newcommand{\dvec}{{d}}
\newcommand{\evec}{{e}}
\newcommand{\tvec}{{t}}
\newcommand{\svec}{{s}}
\newcommand{\rvec}{{r}}
\newcommand{\gvec}{{g}}
\newcommand{\Gvec}{{\mathrm G}}
\newcommand{\lambdavec}{$$\mbox{\boldmath$\lambda$}$$}
\newcommand{\Thetamat}{{\mathbf \Theta}}
\newcommand{\Phimat}{{\mathbf \Phi}}
\newcommand{\Deltamat}{{\mathbf \Delta}}
\newcommand{\Sigmamat}{{\mathbf \Sigma}}
\newcommand{\Omegamat}{{\mathbf \Omega}}
\newcommand{\Psimat}{{ \Psi}}
\newcommand{\Pimat}{{\mathbf \Pi}}
\newcommand{\Gammamat}{{\mathbf \Gamma }}
\newcommand{\Lambdamat}{{\mathbf \Lambda}}
\newcommand{\epsilonvec}{{\mathbf \epsilon}}
\newcommand{\etavec }{{\mathbf \eta }}
\newcommand{\zeromat }{{\mathbf 0 }}
\newcommand{\Zeromat }{{\mathbf 0 }}
%\newcommand{\log10}{{\log_{10}}}
\newcommand{\Gmat}{{G}}
\newcommand{\Omat}{{O}}
\newcommand{\Mmat}{{M}}
\newcommand{\Lmat}{{L}}
\newcommand{\Emat}{{E}}
\newcommand{\Vmat}{{V}}
\newcommand{\Kmat }{{K}}
\newcommand{\Fmat }{{F}}
\newcommand{\Bmat}{{B}}
\newcommand{\Cmat}{{C}}
\newcommand{\Hmat}{{H}}
\newcommand{\Imat}{{\mathbf I}}
\newcommand{\Dmat}{{D}}
\newcommand{\Qmat}{{Q}}
\newcommand{\Rmat}{{R}}
\newcommand{\Jmat}{{J}}
\newcommand{\Xmat}{{X}}
\newcommand{\Nmat}{{N}}
\newcommand{\Amat}{{A}}
\newcommand{\Ymat}{{Y}}
\newcommand{\Wmat}{{W}}
\newcommand{\Umat}{{U}}
\newcommand{\Tmat}{{T}}
\newcommand{\Smat}{{S}}
\newcommand{\Zmat}{{Z}}
\newcommand{\Pmat}{{P}}
%\newcommand{\Gmat}{{\mathrm G}}
%\newcommand{\Omat}{{\mathrm O}}
%\newcommand{\Mmat}{{\mathrm M}}
%\newcommand{\Lmat}{{\mathrm L}}
%\newcommand{\Emat}{{\mathrm E}}
%\newcommand{\Vmat}{{\mathrm V}}
%\newcommand{\Kmat }{{\mathrm K}}
%\newcommand{\Fmat }{{\mathrm F}}
%\newcommand{\Bmat}{{\mathrm B}}
%\newcommand{\Cmat}{{\mathrm C}}
%\newcommand{\Hmat}{{\mathrm H}}
%\newcommand{\Imat}{{\mathrm I}}
%\newcommand{\Dmat}{{\mathrm D}}
%\newcommand{\Qmat}{{\mathrm Q}}
%\newcommand{\Rmat}{{\mathrm R}}
%\newcommand{\Jmat}{{\mathrm J}}
%\newcommand{\Xmat}{{\mathrm X}}
%\newcommand{\Nmat}{{\mathrm N}}
%\newcommand{\Amat}{{\mathrm A}}
%\newcommand{\Ymat}{{\mathrm Y}}
%\newcommand{\Wmat}{{\mathrm W}}
%\newcommand{\Umat}{{\mathrm U}}
%\newcommand{\Tmat}{{\mathrm T}}
%\newcommand{\Smat}{{\mathrm S}}
%\newcommand{\Zmat}{{\mathrm Z}}
%\newcommand{\Pmat}{{\mathrm P}}
\newcommand{\Gmatf}{{\mathbf G}}
\newcommand{\Omatf}{{\mathbf O}}
\newcommand{\Mmatf}{{\mathbf M}}
\newcommand{\Lmatf}{{\mathbf L}}
\newcommand{\Ematf}{{\mathbf E}}
\newcommand{\Vmatf}{{\mathbf V}}
\newcommand{\Rmatf}{{\mathbf R}}
\newcommand{\Kmatf }{{\mathbf K}}
\newcommand{\Fmatf }{{\mathbf F}}
\newcommand{\Hmatf}{{\mathbf H}}
\newcommand{\Imatf}{{\mathbf I}}
\newcommand{\Dmatf}{{\mathbf D}}
\newcommand{\Qmatf}{{\mathbf Q}}
\newcommand{\Jmatf}{{\mathbf J}}
\newcommand{\Xmatf}{{\mathbf X}}
\newcommand{\Nmatf}{{\mathbf N}}
\newcommand{\Amatf}{{\mathbf A}}
\newcommand{\Ymatf}{{\mathbf Y}}
\newcommand{\Wmatf}{{\mathbf W}}
\newcommand{\Umatf}{{\mathbf U}}
\newcommand{\Tmatf}{{\mathbf T}}
\newcommand{\Smatf}{{\mathbf S}}
\newcommand{\Zmatf}{{\mathbf Z}}
\newcommand{\Pmatf}{{\mathbf P}}
\newcommand{\Bmatf}{{\mathbf B}}
\newcommand{\Cmatf}{{\mathbf C}}
\newcommand{\be }{\begin{equation}}
\newcommand{\ee }{\end{equation}}
\newcommand{\beqn }{\begin{eqnarray}}
\newcommand{\eeqn }{\end{eqnarray}}
           % personal LaTeX macros
% Global abbreviations for all documents.
% Louis Williams  -  27 July 1990

% -------------
% abbreviations
% -------------
\def\beq{\begin{equation}}      % equation
\def\eeq{\end{equation}}
\def\ber{\begin{eqnarray}}      % equation
\def\eer{\end{eqnarray}}
\def\bal{\begin{align}}      % equation
\def\eal{\end{align}}
\def\bd{\begin{displaymath}}        % unnumbered equation
\def\ed{\end{displaymath}}
\def\bfig{\begin{figure}}       % figure
\def\efig{\end{figure}}
\def\btab{\begin{table}}        % table
\def\etab{\end{table}}
\def\IntAll{\int_{-\infty}^{\infty}}
\def\SumAll#1{\sum_{#1 = -\infty}^{\infty} \!}
\def\Expect#1{{\cal E} \! \left\{ #1 \right\}}
\def\comment#1{}                        % comment
             %       same
\newtheorem{thm}{Theorem}[chapter]
\newtheorem{cor}[thm]{Corollary}
\newtheorem{lem}[thm]{Lemma}
\topmargin 0.5in
\graphicspath{{./}{./FIGS/}}
\renewcommand{\topfraction}{.9}
\renewcommand{\bottomfraction}{.9}
\renewcommand{\textfraction}{.1}



% \newcommand{\hvecf}{{\mathbf h}}
\newcommand{\bvecf}{{\mathbf b}}
\newcommand{\pvecf}{{\mathbf p}}
\newcommand{\kvecf}{{\mathbf k}}
\newcommand{\wvecf}{{\mathbf w}}
\newcommand{\nvecf}{{\mathbf n}}
\newcommand{\qvecf}{{\mathbf q}}
\newcommand{\yvecf}{{\mathbf y}}
\newcommand{\cvecf}{{\mathbf c}}
\newcommand{\zvecf}{{\mathbf z}}
\newcommand{\uvecf}{{\mathbf u}}
\newcommand{\vvecf}{{\mathbf v}}
\newcommand{\mvecf}{{\mathbf m}}
\newcommand{\avecf}{{\mathbf a}}
\newcommand{\lvecf}{{\mathbf l}}
\newcommand{\xvecf}{{\mathbf x}}
\newcommand{\Gvecf}{{\mathbf G}}
\newcommand{\dvecf}{{\mathbf d}}
\newcommand{\evecf}{{\mathbf e}}
\newcommand{\tvecf}{{\mathbf t}}
\newcommand{\svecf}{{\mathbf s}}
\newcommand{\rvecf}{{\mathbf r}}
\newcommand{\gvecf}{{\mathbf g}}
%\newcommand{\hvec}{{\mathrm h}}
%\newcommand{\bvec}{{\mathrm b}}
%\newcommand{\pvec}{{\mathrm p}}
%\newcommand{\kvec}{{\mathrm k}}
%\newcommand{\wvec}{{\mathrm w}}
%\newcommand{\nvec}{{\mathrm n}}
%\newcommand{\qvec}{{\mathrm q}}
%\newcommand{\yvec}{{\mathrm y}}
%\newcommand{\cvec}{{\mathrm c}}
%\newcommand{\zvec}{{\mathrm z}}
%\newcommand{\uvec}{{\mathrm u}}
%\newcommand{\vvec}{{\mathrm v}}
%\newcommand{\mvec}{{\mathrm m}}
%\newcommand{\avec}{{\mathrm a}}
%\newcommand{\lvec}{{\mathrm l}}
%\newcommand{\xvec}{{\mathrm x}}
%\newcommand{\dvec}{{\mathrm d}}
%\newcommand{\evec}{{\mathrm e}}
%\newcommand{\tvec}{{\mathrm t}}
%\newcommand{\svec}{{\mathrm s}}
%\newcommand{\rvec}{{\mathrm r}}
%\newcommand{\gvec}{{\mathrm g}}

\newcommand{\hvec}{{h}}
\newcommand{\bvec}{{b}}
\newcommand{\pvec}{{p}}
\newcommand{\kvec}{{k}}
\newcommand{\wvec}{{w}}
\newcommand{\nvec}{{n}}
\newcommand{\qvec}{{q}}
\newcommand{\yvec}{{y}}
\newcommand{\cvec}{{c}}
\newcommand{\zvec}{{z}}
\newcommand{\uvec}{{u}}
\newcommand{\vvec}{{v}}
\newcommand{\mvec}{{m}}
\newcommand{\avec}{{a}}
\newcommand{\lvec}{{l}}
\newcommand{\xvec}{{x}}
\newcommand{\dvec}{{d}}
\newcommand{\evec}{{e}}
\newcommand{\tvec}{{t}}
\newcommand{\svec}{{s}}
\newcommand{\rvec}{{r}}
\newcommand{\gvec}{{g}}
\newcommand{\Gvec}{{\mathrm G}}
\newcommand{\lambdavec}{$$\mbox{\boldmath$\lambda$}$$}
\newcommand{\Thetamat}{{\mathbf \Theta}}
\newcommand{\Phimat}{{\mathbf \Phi}}
\newcommand{\Deltamat}{{\mathbf \Delta}}
\newcommand{\Sigmamat}{{\mathbf \Sigma}}
\newcommand{\Omegamat}{{\mathbf \Omega}}
\newcommand{\Psimat}{{ \Psi}}
\newcommand{\Pimat}{{\mathbf \Pi}}
\newcommand{\Gammamat}{{\mathbf \Gamma }}
\newcommand{\Lambdamat}{{\mathbf \Lambda}}
\newcommand{\epsilonvec}{{\mathbf \epsilon}}
\newcommand{\etavec }{{\mathbf \eta }}
\newcommand{\zeromat }{{\mathbf 0 }}
\newcommand{\Zeromat }{{\mathbf 0 }}
%\newcommand{\log10}{{\log_{10}}}
\newcommand{\Gmat}{{G}}
\newcommand{\Omat}{{O}}
\newcommand{\Mmat}{{M}}
\newcommand{\Lmat}{{L}}
\newcommand{\Emat}{{E}}
\newcommand{\Vmat}{{V}}
\newcommand{\Kmat }{{K}}
\newcommand{\Fmat }{{F}}
\newcommand{\Bmat}{{B}}
\newcommand{\Cmat}{{C}}
\newcommand{\Hmat}{{H}}
\newcommand{\Imat}{{\mathbf I}}
\newcommand{\Dmat}{{D}}
\newcommand{\Qmat}{{Q}}
\newcommand{\Rmat}{{R}}
\newcommand{\Jmat}{{J}}
\newcommand{\Xmat}{{X}}
\newcommand{\Nmat}{{N}}
\newcommand{\Amat}{{A}}
\newcommand{\Ymat}{{Y}}
\newcommand{\Wmat}{{W}}
\newcommand{\Umat}{{U}}
\newcommand{\Tmat}{{T}}
\newcommand{\Smat}{{S}}
\newcommand{\Zmat}{{Z}}
\newcommand{\Pmat}{{P}}
%\newcommand{\Gmat}{{\mathrm G}}
%\newcommand{\Omat}{{\mathrm O}}
%\newcommand{\Mmat}{{\mathrm M}}
%\newcommand{\Lmat}{{\mathrm L}}
%\newcommand{\Emat}{{\mathrm E}}
%\newcommand{\Vmat}{{\mathrm V}}
%\newcommand{\Kmat }{{\mathrm K}}
%\newcommand{\Fmat }{{\mathrm F}}
%\newcommand{\Bmat}{{\mathrm B}}
%\newcommand{\Cmat}{{\mathrm C}}
%\newcommand{\Hmat}{{\mathrm H}}
%\newcommand{\Imat}{{\mathrm I}}
%\newcommand{\Dmat}{{\mathrm D}}
%\newcommand{\Qmat}{{\mathrm Q}}
%\newcommand{\Rmat}{{\mathrm R}}
%\newcommand{\Jmat}{{\mathrm J}}
%\newcommand{\Xmat}{{\mathrm X}}
%\newcommand{\Nmat}{{\mathrm N}}
%\newcommand{\Amat}{{\mathrm A}}
%\newcommand{\Ymat}{{\mathrm Y}}
%\newcommand{\Wmat}{{\mathrm W}}
%\newcommand{\Umat}{{\mathrm U}}
%\newcommand{\Tmat}{{\mathrm T}}
%\newcommand{\Smat}{{\mathrm S}}
%\newcommand{\Zmat}{{\mathrm Z}}
%\newcommand{\Pmat}{{\mathrm P}}
\newcommand{\Gmatf}{{\mathbf G}}
\newcommand{\Omatf}{{\mathbf O}}
\newcommand{\Mmatf}{{\mathbf M}}
\newcommand{\Lmatf}{{\mathbf L}}
\newcommand{\Ematf}{{\mathbf E}}
\newcommand{\Vmatf}{{\mathbf V}}
\newcommand{\Rmatf}{{\mathbf R}}
\newcommand{\Kmatf }{{\mathbf K}}
\newcommand{\Fmatf }{{\mathbf F}}
\newcommand{\Hmatf}{{\mathbf H}}
\newcommand{\Imatf}{{\mathbf I}}
\newcommand{\Dmatf}{{\mathbf D}}
\newcommand{\Qmatf}{{\mathbf Q}}
\newcommand{\Jmatf}{{\mathbf J}}
\newcommand{\Xmatf}{{\mathbf X}}
\newcommand{\Nmatf}{{\mathbf N}}
\newcommand{\Amatf}{{\mathbf A}}
\newcommand{\Ymatf}{{\mathbf Y}}
\newcommand{\Wmatf}{{\mathbf W}}
\newcommand{\Umatf}{{\mathbf U}}
\newcommand{\Tmatf}{{\mathbf T}}
\newcommand{\Smatf}{{\mathbf S}}
\newcommand{\Zmatf}{{\mathbf Z}}
\newcommand{\Pmatf}{{\mathbf P}}
\newcommand{\Bmatf}{{\mathbf B}}
\newcommand{\Cmatf}{{\mathbf C}}
\newcommand{\be }{\begin{equation}}
\newcommand{\ee }{\end{equation}}
\newcommand{\beqn }{\begin{eqnarray}}
\newcommand{\eeqn }{\end{eqnarray}}
                         % personal LaTeX macros

%%%%%%%%%%%%%%%%%%%%%%%%%%%%%%%%%%%%%%%%%%%%%%%%%%%%%%%%%%%%%%%%%%%%%%
%
% Usually things live in separate flies.
%
% \input {prelim}                           % preliminary page info

%%%%%%%%%%%%%%%%%%%%%%%%%%%%%%%%%%%%%%%%%%%%%%%%%%%%%%%%%%%%%%%%%%%%%%%%
%                                                                      %
%                          PRELIMINARY PAGES                           %
%                                                                      %
%%%%%%%%%%%%%%%%%%%%%%%%%%%%%%%%%%%%%%%%%%%%%%%%%%%%%%%%%%%%%%%%%%%%%%%%

\title          {Novel RF Analysis and Design Techniques}
\author         {Aliakbar Homayoun}
\department     {Electrical Engineering}
% Note:  degreeyear should be optional, but as of  5-Feb-96
% it seems required or you get a year of ``2''.   -johnh
\degreeyear     {2013}

%%%%%%%%%%%%%%%%%%%%%%%%%%%%%%%%%%%%%%%%%%%%%%%%%%%%%%%%%%%%%%%%%%%%%%%%

\acknowledgments {  

I would like to express my gratitude to Professor Behzad Razavi, who is an exemplary advisor in my opinion. He always guided me to the right path in my research. He let me explore things
that I liked many times during my PhD without feeling pressured. But, when he noticed that I am diverging and will not reach any useful result, he asked me not to
think on that problem for a while and gave me another problem. He also sometimes pushed me to finish the projects at the right times which was a must to meet the
deadlines. Other than the technical guidance, he also gave me valuable advice for my life.  
It is an honor for me to get my PhD under his supervision and words cannot say how grateful I am for that. 

I was fortunate to be in a friendly research group that everybody was willing to help others. I think most of this spirit comes from Professor Razavi himself who wanted us to be so.
I would like to thank all the members of our group that I had overlap with, namely, Babak, Sameh, ChuanKang, Bibhue, Brian, Sedigheh, Jun, Joung, Wood, Jithin, Marco, Steven, Hegong, Joseph, Long and Abishek. 
My special thanks go to Marco who helped me a lot to understand various techniques used in frequency synthesizers. He also helped me a lot with Linux.
I also had useful discussions with Sedigheh, Jun, and Joung during my PhD. 

I had also valuable discussions with some people from other research groups in UCLA, namely, Arash, Alborz, Majid, Farid, Mostafa, Mohsen, and Ata.

I would like to thank Professor Carlo Zaniolo, Professor Frank Chang, and Professor Danijela Cabric for serving on my committee. 

Finally, I would like to thank my family, relatives, and friends for their love, support, and encouragement, no matter how far I am from them. 

}

%%%%%%%%%%%%%%%%%%%%%%%%%%%%%%%%%%%%%%%%%%%%%%%%%%%%%%%%%%%%%%%%%%%%%%%%

\chair          {Behzad\ Razavi}
\member         {Danijela\ Cabric}
\member         {Mau-Chung Frank\ Chang}
\member         {Carlo\ Zaniolo}


%%%%%%%%%%%%%%%%%%%%%%%%%%%%%%%%%%%%%%%%%%%%%%%%%%%%%%%%%%%%%%%%%%%%%%%%

\dedication     {\textsl{To my parents \ldots }}


%%%%%%%%%%%%%%%%%%%%%%%%%%%%%%%%%%%%%%%%%%%%%%%%%%%%%%%%%%%%%%%%%%%%%%%%



%%%%%%%%%%%%%%%%%%%%%%%%%%%%%%%%%%%%%%%%%%%%%%%%%%%%%%%%%%%%%%%%%%%%%%%%

\vitaitem   {1984}
                {Born, Shiraz, Iran.}
\vitaitem   {2006}
                {B.S.~(Electrical Engineering), Sharif University of Technology, Tehran, Iran.}
\vitaitem   {2009}
                {M.S.~(Electrical Engineering), Sharif University of Technology, Tehran, Iran.}
\vitaitem   {2013}
                {Ph.D.~(Electrical Engineering), University of California, Los Angeles, USA.}		

%%%%%%%%%%%%%%%%%%%%%%%%%%%%%%%%%%%%%%%%%%%%%%%%%%%%%%%%%%%%%%%%%%%%%%%%

\publication    {A. Homayoun and B. Razavi, ``Analysis of phase noise in phase/frequency detectors,'' {\em IEEE Trans. Circuits Syst. I, Reg. Papers,} vol 60, no. 3,
pp. 529--539, Mar. 2013.

\noindent A. Homayoun and B. Razavi, ``Relation between delay line phase noise and ring oscillator phase noise,'' {\em IEEE J. Solid-State Circuits,} vol.
49, no. 2, Feb. 2014.

\noindent A. Homayoun and B. Razavi, ``A 5-GHz 11.6-mW CMOS receiver for IEEE 802.11a applications,'' presented at the {\em IEEE CICC}, Sep. 2013
}

%%%%%%%%%%%%%%%%%%%%%%%%%%%%%%%%%%%%%%%%%%%%%%%%%%%%%%%%%%%%%%%%%%%%%%%%

\abstract       {

The design of RF integrated circuits continues to challenge engineers and researchers, demanding new circuit topologies and transceiver
architectures. New ideas often require new analysis techniques as well, so that the designer can insightfully quantify the underlying
principles.

This research addresses three problems in RF circuits: (1) analysis of phase noise in phase/frequency detectors (PFDs), an essential component
in RF synthesizers; (2) analysis of the relation between the phase noise of delay lines and ring oscillators; and (3) design of a new
low-power RF CMOS receiver for IEEE 802.11a. The first analysis derives equations for the phase noise and shows that an octave increase in the
input frequency raises the phase noise by 6 dB if flicker noise is dominant and by 3 dB if white noise is dominant. The second analysis
reveals a simple shaping function and also dispels the commonly-accepted premise that symmetric rise and fall times in a ring oscillator
suppress the upconversion of flicker noise. The third part deals with the design of a low-power 5-GHz receiver. While advances in the art have
considerably reduced the power consumption of RF oscillators, frequency dividers, and analog-to-digital converters, the main receiver chain in
5-GHz systems draws a disproportionately high power, about 46 mW. It is therefore desirable to develop low-power RX front ends and baseband
filters for WiFi applications. This work introduces a complete 5-GHz CMOS receiver that meets the 11a sensitivity, blocking, and filtering
requirements while consuming 11.6 mW. This fourfold reduction in power is achieved through the use of a transformer as a low-noise amplifier,
passive mixers, and non-invasive baseband filtering. A new analysis of passive current-driven mixers sheds light on their properties.}

%%%%%%%%%%%%%%%%%%%%%%%%%%%%%%%%%%%%%%%%%%%%%%%%%%%%%%%%%%%%%%%%%%%%%%%%


\begin {document}
\makeintropages

%%%%%%%%%%%%%%%%%%%%%%%%%%%%%%%%%%%%%%%%%%%%%%%%%%%%%%%%%%%%%%%%%%%%%%
%
% Ordinarily each chapter (at least) is in a separate file.
%
\chapter{Introduction} 

The design of RF integrated circuits continues to challenge engineers and researchers, demanding new circuit topologies and transceiver architectures. New ideas often require new analysis techniques as well, so that the designer can insightfully quantify the underlying
principles. In addition, novel analysis of the existing circuits helps the designer to optimize them and sometimes suggests modifications for better performance.

\section{Organization} 

This dissertation proposes three novel analyses in RF circuits. The first analysis which is covered in chapter 2, is on the phase noise in phase/frequency detectors (PFDs), an essential component in RF synthesizers. This is the only published work on the subject that
derives equations. Using our compact equations, we have optimized two PFDs for minimum phase noise. Chapter 3 is the first analysis of the relation between the phase noise of delay lines and ring oscillators. Knowing the relation, compact equations are derived for the
phase noise of ring oscillators which show a factor of 2 correction for the flicker-noise-induced phase noise. In addition, it dispels the commonly-accepted premise that symmetric rise and fall times in a ring oscillator suppress the upconversion of flicker noise.
Chapter 4 deals with the design of a new low-power RF CMOS receiver for IEEE 802.11a. The most interesting part is the replacement of the LNA with a transformer. How do the transformer and mixers provide proper input matching? Why the frequency response is flat across
the channel while the loads of the mixers are capacitors? Why the noise figure is not too high? These are all important questions that would not have been answered without our novel analysis explained in chapter 4. Finally, chapter 5 summarizes the future work.
                         % Chapter 1 of dissertation
\chapter{Analysis of Phase Noise in Phase/Frequency Detectors}

The phase noise of phase/frequency detectors can significantly raise the in-band phase noise of frequency synthesizers, corrupting the modulated
signal. This chapter analyzes the phase noise mechanisms in CMOS phase/frequency detectors and applies the results to two different topologies. It is
shown that an octave increase in the input frequency raises the phase noise by 6 dB if flicker noise is dominant and by 3 dB if white noise is
dominant. An optimization methodology is also proposed that lowers the phase noise by 4 to 8 dB for a given power consumption. Simulation and
analytical results agree to within 3.1 dB for the two topologies at different frequencies.



\section{Introduction}

The phase noise of the phase/frequency detector (PFD) in a phase-locked loop (PLL) directly adds to that of the reference,
manifesting itself for a high frequency multiplication factor and/or a wide loop bandwidth.

This chapter investigates the phase noise mechanisms in PFDs and computes the phase noise spectral density due to both white noise and flicker
noise. The results are applied to two PFD topologies, one using static NAND gates and the other employing true single-phase clocking (TSPC). A PFD
phase noise simulation technique is also proposed. The objective is to enable the designer to predict the PFD phase noise, and more importantly,
design the PFD so as to make its contribution to the overall PLL phase noise negligible.

The chapter is organized as follows. Section 2.2 describes the background and motivation for this work. Section 2.3 builds the foundation by
calculating the jitter spectrum of an inverter and Section 2.4 extends the results to a NAND gate. Section 2.5 applies
these findings to the analysis of two PFD topologies. Section 2.6 discusses the optimization of phase noise for the two PFDs and Section 2.7
presents simulation results. Section 2.8 explains the effect of pulse position modulation of the Up and Down signals and Section 2.9 calculates 
the phase noise of square wave with uncorrelated jitter on rising and falling edges. Section 2.10 proves that the spectrum of shaped and sampled 
white noise is white under certain conditions and Section 2.11 concludes chapter 2.


\section{Background}
\subsection{Motivation}
The in-band multiplication of a PFD's phase noise can create difficulties in RF synthesizer design \cite{Tsutsumi}-\cite{Wilson}. Consider, as an example, a
5-GHz synthesizer targeting IEEE802.11a applications. To negligibly corrupt the 64QAM signal constellation, the synthesizer
must achieve an integrated phase noise of roughly $0.5^\circ$ rms \cite{Chen}.\footnote{We assume the transmit and
receive synthesizers contribute equal but uncorrelated amounts of phase noise.}
\begin{figure}[htb!]
\centering
\includegraphics[scale=1]{FIGS/CH2/fig1a.ps}
\includegraphics[scale=0.6]{FIGS/CH2/fig1b.eps}
\includegraphics[scale=1]{FIGS/b.ps}
\caption{(a) NOR-based PFD, and (b) output phase noise of a 5-GHz PLL due to PFD.}
\label{fig:pfdnor}
\end{figure}
Now, suppose the standard NOR PFD shown in Fig. \ref{fig:pfdnor}(a) is employed at the input of such a synthesizer with an input
frequency of 20 MHz and a loop bandwidth of about 2 MHz. Plotted in Fig. \ref{fig:pfdnor}(b) is the simulated output phase noise of
the synthesizer including only the PFD contribution. Here, the PFD incorporates $(W/L)_{PMOS}=0.3\ \mu$m/60 nm and
$(W/L)_{NMOS}=0.2\ \mu$m/60 nm. The area under this plot from 10 kHz to 10 MHz yields an rms jitter of $0.3^\circ$, severely tightening the
contribution allowed for the voltage-controlled oscillator (VCO).

As another example, consider a 60-GHz transceiver operating with QPSK signals. A synthesizer multiplying the above PFD phase
noise to 60 GHz would exhibit an rms jitter of $3.5^\circ$. On the other hand, for negligible corruption of QPSK signals, the rms jitter
must be less than about $2.1^\circ$ \cite{Chen}.

The above examples underscore the need for a detailed treatment of phase noise mechanisms in PFDs. Of course, the
charge pump may also contribute significant phase noise and merits its own analysis.

\subsection{Observations}
Consider the generic PLL shown in Fig. \ref{fig:pll}. The PFD generates the Up and Down pulses in response to the rising edges on $A$ and $B$. The
noise in the PFD devices modulates the width and edges of the output pulses, creating a random component in the current produced by the charge
pump (CP). We neglect the phase noise of all other building blocks and denote the input frequency by $f_{in}$. 

\begin{figure}[htb!]
\centering
\includegraphics[scale=1]{FIGS/CH2/fig2.ps}
\caption{A PFD in an integer-N PLL.}
\label{fig:pll}
\end{figure}

The phase noise in Up and Down translates to random modulation of the time during which $I_1$ or $I_2$ is injected into the loop filter.
We consider three possible cases. As shown in Fig. \ref{fig:3cases}(a), the phase noise may modulate the widths of Up and Down by the same amount,
in which case the CP produces no net output. In the second case [Fig. \ref{fig:3cases}(b)], the phase noise modulates only the position of Up with
respect to Down. As explained in Section 2.8, this effect is negligible. Lastly, the phase noise may modulate the widths of Up and Down
pulses {\em differently} [Fig. \ref{fig:3cases}(c)], and it is this case that matters most.

\begin{figure}[htb!]
\centering
\includegraphics[scale=1]{FIGS/CH2/fig3.ps}
\caption{Modulation of Up and Down (a) width by the same amount, (b) position, and (c) width differently.}
\label{fig:3cases}
\end{figure}

The above observations also reveal that, contrary to a designer's first guess, the PFD phase noise of interest is {\em not} equal to
the phase noise of the Up or Down signals themselves. After all, if the widths of Up and Down pulses vary randomly but exactly in unison, then the
net current produced by the CP contains no random component. This point raises the question of how exactly the PFD noise must be
simulated, which we address in Section 2.6.

The foregoing points suggest that the phase noise arising from a PFD in fact relates to the random pulsewidth {\em difference} between the Up and
Down signals, $\Delta T_{UD}$. Moreover, four edges, namely, the rising and falling edges of both Up and Down signals, contribute to $\Delta
T_{UD}$. Some of the PFD internal transitions displace Up and Down by the same amount and should be
ignored (Section 2.5). 

The analysis of PFD phase noise in \cite{Brennan}, \cite{Thompson} relates the phase noise to the timing jitter, $\Delta t$, as $\Delta \phi _{in} =
2 \pi f_s \Delta t$, where $f_s$ denotes the operating frequency, but expresses $\Delta t$ in terms of the (thermal) noise factor and input
resistance of the PFD.
By contrast, our approach begins with the gates comprising the PFD and determines the jitter in the Up and Down pulsewidth difference, taking into
account both flicker and thermal noise. {The mismatch between Up and Down currents is neglected here.}\footnote{Simulations show 0.2 dB
higher phase noise due to a 10\% mismatch between the Up and Down currents.}


\section{Phase Noise of CMOS Inverter}
A good understanding of the phase noise mechanisms in CMOS inverters proves beneficial in the analysis of PFDs as well. Consider the CMOS inverter
and its waveforms shown in Fig. \ref{fig:noisepercycle}. We wish to study the time envelope of the noise produced by $M_1$ and $M_2$. These transistors inject
thermal and flicker noise to the output node as they turn on. At the end of the transition, however, the on transistor carries no current and
produces no flicker noise. Thus, the thermal noise envelope of each transistor lasts about half of the input cycle, $T_{in}$, whereas its flicker
noise envelope pulsates only during transitions [Fig. \ref{fig:noisepercycle}(b)]. Note that in typical PLLs, the transition times within a PFD are much shorter than the input
period.

\begin{figure}[htb!]
\centering
\includegraphics[scale=1]{FIGS/CH2/fig4.ps}
\caption{(a) CMOS inverter, and (b) thermal and flicker noise envelopes of $M_1$.}
\label{fig:noisepercycle}
\end{figure}

In the analysis that follows, we make numerous approximations based on our intuitive understanding of the circuit's behavior. The soundness of
these approximations is ultimately put to test in Section 2.7, where two completely different PFD realizations are simulated and the results are
compared with hand calculations.

It is convenient to view the noise injection of $M_1$ and $M_2$ as follows: the transistor that is turning on injects thermal and flicker noise
during the transition, and the transistor that is turning off (coming out of the deep triode region) deposits kT/C noise at the output.

\subsection{Noise of Transistor Turning On}
In order to formulate the noise contribution by the transistors in Fig. \ref{fig:noisepercycle}, we must examine the circuits' waveforms more
closely. As depicted in Fig. \ref{fig:noiseenvelope} for a rising transition at the input and for an inverter with a fanout of about 2, the output begins to fall only
after $V_{in}$ is relatively close to $V_{DD}$. Transistor $M_1$ turns on as $V_{in}$ exceeds its threshold, $V_{THN}$, at $t=t_1$, and injects increasingly
larger flicker and thermal noise as $V_{in}$ rises. The noise envelope reaches a maximum before the transistor enters the triode region, around
$t=t_2$. Thereafter, the flicker noise injection subsides, falling to zero at $t=t_3$. The thermal noise current, on the other hand, goes from
$4kT\gamma g_m$ to a slightly lower value, $4kT$/$R_{on}$, where $R_{on}$ denotes the channel resistance of $M_1$ with $V_{GS}=V_{DD}$.

\begin{figure}[htb!]
\centering
\includegraphics[scale=1]{FIGS/CH2/fig5.ps}
\caption{Detailed view of thermal and flicker noise envelopes during input and output transitions.}
\label{fig:noiseenvelope}
\end{figure}

Our next simplifying assumption is that the output phase noise of interest manifests itself while $V_{out}$ in Fig. \ref{fig:noiseenvelope} crosses
approximately $V_{DD}$/$2$ and the noise injected by the transistors after this point is unimportant \cite{Abidi}. Thus, in the waveforms of Fig.
\ref{fig:noiseenvelope}, we consider the area under the envelopes for only up to $t=t_{mid}$. 

We now wish to approximate the area under the noise envelopes by a simple function. As shown in Fig. \ref{fig:deltat}, the flicker noise envelope is
approximated by a rectangular waveform of the same height, $h$, but lasting from the time the actual envelope reaches half of its height,
$t_{h/2}$, to the time $V_{out}$ reaches $V_{DD}$/$2$, $t_{mid}$.
We expect that the sum of the gray areas is roughly equal to the cross-hatched area. Transient noise simulations in Cadence's Spectre indicate an
error of about 4\% in this approximation. We apply the same concept to the thermal noise envelope as well. Note that \cite{Abidi} uses a
rectangle from the time $V_{out}$ begins to fall ($t_p$ in Fig. \ref{fig:noiseenvelope}) to $t_{mid}$, which, according to simulations, underestimates
the integrated noise power by 2 to 3 dB.

\begin{figure}[htb!]
\centering
\includegraphics[scale=1]{FIGS/CH2/fig6.ps}
\caption{Rectangular approximation of noise envelope.}
\label{fig:deltat}
\end{figure}

Another simplifying assumption can be derived from the waveforms in Fig. \ref{fig:noiseenvelope}: at the peak of the noise
envelope, one transistor is nearly off. Thus, we consider only the noise of $M_1$ on the falling edges at the output and only the noise of $M_2$ on
the rising edges.

Based on the foregoing approximations and utilizing the rectangular function, $w(t)$, in Fig. \ref{fig:deltat}, we now outline the inverter phase noise
analysis as follows. As shown in Fig. \ref{fig:vni}(a), the noise current of each transistor, $i_n(t)$ is equivalently multiplied by shifted versions of
$w(t)$. Each product is integrated for a duration of $\Delta T=t_{mid}-t_{h/2}$ and divided by the load capacitance, $C_L$, yielding the noise
voltage [Fig. \ref{fig:vni}(b)].
\begin{figure}[htb!]
\centering
\includegraphics[scale=1]{FIGS/CH2/fig7.ps}
\caption{(a) Equivalent operation of inverter on noise of one transistor, and (b) conversion of noise current to noise voltage.}
\label{fig:vni}
\end{figure}
These voltages are then divided by the slew rate, $r_{edge}$ {(Fig. \ref{fig:deltat})}, to give the time displacement (jitter), sampled, and summed. We write the noise
voltage, $v_{n,1}$, after the first window as 
\begin{eqnarray}
v_{n,1}&=&\frac{1}{C_L}\int_{0}^{\Delta T}\!{i_n(t)}\, dt \nonumber  \\ 
&=&\frac{1}{C_L}\int_{-\infty}^{+\infty}\!i_n(t)w(t)\, dt .
\label{vn1}
\end{eqnarray}
Note that the load capacitance is assumed constant and equal to its value at $V_{out}=V_{DD}/2$. Also, the
integration tacitly neglects the effect of the inverter's output resistance, $r_O$. This approximation is justified because the time constant,
$r_O C_L$, at the inverter output is much greater than $\Delta T$. Similarly,
\beq
v_{n,m}=\frac{1}{C_L}\int_{-\infty}^{+\infty}\!i_n(t)w(t-mT_{in})\, dt .
\label{vnm}
\eeq
The particular shape of $w(t)$ allows this equation to be rewritten as
\beq
v_{n,m}=\frac{1}{C_L}\int_{-\infty}^{+\infty}\!i_n(t)w(mT_{in}-t)\, dt ,
\label{vnm2}
\eeq
which is the convolution integral \cite{Abidi}. The noise voltage spectrum is therefore given by 
\beq
S_{Vn}(f)=\frac{1}{C_L^2}|W(f)|^2 S_{In}(f),
\label{svn}
\eeq
where $W(f)$ denotes the Fourier transform of $w(t)$ and $S_{In}(f)$ the spectrum of $i_n(t)$. 
As shown in Section 2.9, the phase noise spectrum\footnote{Throughout this chapter, all the spectra are two-sided, and the phase noise is denoted by
$S_{\Phi}(f)$.} due to noise of NMOS transistor on the falling edges is equal to
\beq
S_{\Phi}(f)=\frac{\pi^2}{r_{edge}^2 T_{in}^2}\sum_{m=-\infty}^{m=+\infty} S_{Vn}(f-\frac{m}{T_{in}}).
\label{sphi}
\eeq

It is important to recognize two differences between the above analysis and that in \cite{Abidi}: (1) as mentioned earlier, our window definition
(from $t_{h/2}$ to $t_{mid}$) more accurately predicts the injected noise power, and (2) the sampling phenomenon reveals aliasing even for flicker
noise if the $1/f$ corner, $f_{cor}$, is comparable with the operation frequency, which may be the case for PFDs.

We now simplify Eq. (\ref{sphi}) if $\overline{I_n^2}$ is white. As shown in Section 2.10, $S_{\Phi}(f)$ is also white and equal to 
\begin{eqnarray}
S_{\Phi}(f)&=&\frac{\pi^2}{r_{edge}^2 T_{in}^2}\frac{1}{C_L^2}\frac{\Delta T}{f_{in}}S_I(f) \nonumber \\ &=&\frac{\pi^2}{r_{edge}^2 C_L^2} \frac{\Delta T}{T_{in}} S_I(f).
\label{sphiw}
\end{eqnarray}
In this expression, the load capacitance appears in both $r_{edge}$ ($=I_D/C_L$, where $I_D$ is the drain current of the on transistor as $V_{out}$
crosses $V_{DD}/2$) and in $\Delta T$. Thus, $S_{\Phi}(f)$ is {\em directly} proportional to $C_L$ and $f_{in}$. The output phase noise due to white
noise therefore rises by 3 dB for each doubling of the operation frequency.

The flicker noise behavior of the inverter can also be deduced from Eq. (\ref{sphi}). If $f_{in}$ is well above the flicker noise corner
frequency, no aliasing occurs and (\ref{sphi}) is simplified by choosing $m=0$ :
\beq
S_{\Phi}(f)=\frac{\pi^2}{r_{edge}^2 T_{in}^2}S_{Vn}(f).
\label{sphina}
\eeq
Since $\Delta T$ is much less than $1/f_{cor}$, we can assume $W(f)={\Delta T}^2 {sinc}^2(\pi f \Delta T)$ is relatively constant for the frequency
range of interest and equal to ${\Delta T}^2$. It follows that
\beq
S_{\Phi}(f)=\frac{\pi^2}{r_{edge}^2 T_{in}^2}\frac{{\Delta T}^2}{C_L^2}S_{1/f}(f),
\label{sphif}
\eeq
where $S_{1/f}(f)$ denotes the noise current spectral density of the on transistor due to its $1/f$ noise. In this case, the phase noise rises by 6
dB for each doubling of $f_{in}$. It also exhibits a stronger dependence upon $\Delta T$. As mentioned earlier, (\ref{sphiw}) and (\ref{sphif}) are
evaluated for $M_1$ on the falling edge at the output and for $M_2$ on the rising edge.
Note that \cite{Abidi} does not analyze the effect of flicker noise in CMOS inverters.

\subsection{Noise of Transistor Turning Off}
As illustrated in Fig. \ref{fig:noiseenvelope}, when the noise envelope reaches its peak, one transistor is near the edge of the triode region and the other is almost
off. Before turning off, however, this transistor has acted as a resistor, producing noise across $C_L$. {Turning off once every $T_{in}$ seconds,
the NMOS transistor deposits a noise voltage whose spectral density is given by $(kT/C_L)/f_{in}$. As
shown in Section 2.9, the falling edges
exhibit a phase noise equal to 
\beq
S_{1}(f)=\frac{\pi^2}{T_{in}^2}\frac{1}{r_{edge}^2}\frac{kT}{C_L f_{in}}.
\label{sphi1}
\eeq
Taking the PMOS contribution into account, we obtain the total kT/C-induced phase noise as
\beq
S_{\Phi}(f)=\frac{2 \pi^2}{T_{in}^2}\frac{1}{r_{edge}^2}\frac{kT}{C_L f_{in}}.
\label{sphioff}
\eeq
}

\subsection{Total Phase Noise}
The total phase noise is given by the sum of five terms: Eqs. (\ref{sphiw}) and (\ref{sphif}) evaluated for both NMOS and PMOS transistors, and
Eq. (\ref{sphioff}):

\begin{eqnarray}
S_{\Phi}(f)&=& \left\{\frac{\pi^2}{r_{edge}^2 C_L^2}[\frac{{\Delta T}}{T_{in}}S_{I}(f)+\frac{{\Delta T}^2}{T_{in}^2}S_{1/f}(f)]\right\}_{NMOS} \nonumber  \\ 
&+&\left\{\frac{\pi^2}{r_{edge}^2 C_L^2}[\frac{{\Delta T}}{T_{in}}S_{I}(f)+\frac{{\Delta T}^2}{T_{in}^2}S_{1/f}(f)]\right\}_{PMOS} \nonumber  \\ 
&+&\frac{2 \pi^2 f_{in}}{r_{edge}^2}\frac{kT}{C_L}.
\label{sphitot}
\end{eqnarray}




\section{Phase Noise of CMOS NAND Gate}
The inverter phase noise analysis can be readily extended to other CMOS gates as well. We briefly consider here the noise behavior of a static NAND
gate and use the results in Section 2.5 to study a NAND-based PFD.

Since in a PFD environment, the two inputs do not change simultaneously, we can reduce the gate to an inverter for each transition. Such an inverter
incurs an additional capacitance at the output due to the second PMOS transistor, and its output falling edge is produced by the series combination
of two NMOS transistors (Fig. \ref{fig:nand}).

\begin{figure}[htb!]
\centering
\includegraphics[scale=1]{FIGS/CH2/fig8.ps}
\caption{NAND gate with one input changing.}
\label{fig:nand}
\end{figure}

In our PFD design, $M_1$ and $M_3$ have the same width and minimum length; thus, they can be replaced with one NMOS device having twice their
length.\footnote{The drain and source capacitance at node X introduce a negligible error in this equivalency.}
In other words, Eq. (\ref{sphitot}) holds if $r_{edge}$, $\Delta T$, $C_L$ and $S_{In}(f)$ are modified to reflect the equivalent values
in the NAND circuit.




\section{PFD Phase Noise Analysis}
\subsection{NAND PFD}
As suggested by the factors ${\Delta T}$ in (\ref{sphiw}) and $\Delta T^2$ in (\ref{sphif}), the phase noise rises in proportion to
the turn-on time of the transistors in each gate. A worthy effort in PFD design, therefore, is to minimize the rise and
fall times. We thus modify the standard NOR-based PFD to the NAND-based topology shown in Fig. \ref{fig:pfdnand}(a). Note that this circuit
responds to the {\em falling} edges of $A$ and $B$, and its Up and Down outputs are {\em low} when asserted.

We must now examine the propagation of the edges through the PFD circuit, seeking those whose jitter modulates the
pulsewidth {\em difference} between the Up and Down pulses. To this end, we draw a detailed timing diagram, mark with a certain shade or
pattern the
jitter contributed by each gate to each transition, carry the jitters on to the final Up and Down
pulses, and omit those that are in common.

\begin{figure}[htb!]
\centering
\includegraphics[scale=1]{FIGS/CH2/fig9.ps}
\caption{(a) NAND-based PFD, (b) jitter contributions to falling edges of outputs, and (c) jitter contributions to rising edges of outputs.}
\label{fig:pfdnand}
\end{figure}

Figure \ref{fig:pfdnand}(b) shows the timing diagram, assuming input $A$ falls earlier than input $B$. NAND 1 adds jitter to the
falling edge of $A$, producing a rising edge on $\overline{Up}$. This edge experiences additional jitter in NAND 2 and
generates the falling edge of Up. That is, each falling edge of Up is corrupted by only the jitters of NANDs 1 and 2.
Similarly, when a falling edge of $B$ follows, $\overline{ Down}$ rises with NAND 5's jitter and Down falls with both
NAND 5's and NAND 6's jitters.

We must also follow the $\overline{ Up}$ and $\overline{ Down}$ rising edges through the reset path. As illustrated in Fig.
\ref{fig:pfdnand}(c), after $\overline{ Down}$ goes up, Reset falls, inheriting the jitters of NAND 5 and NAND 9. In response, $E$
and $F$ rise, incurring additional jitter from NAND 4 and NAND 8, respectively. Subsequently, $C$ falls with the jitter
of NAND 3 and $D$ with that of NAND 7. Finally, Up and Down rise with the jitters of NAND 2 and NAND 6,
respectively.

The Up and Down waveforms in Fig. \ref{fig:pfdnand}(c) merit two remarks. First, NAND 2 contributes jitter to both the rising
and falling edges of Up, but the two jitters are uncorrelated because the former is due to a PMOS device and the latter
due to an NMOS device (the series combination of $M_1$ and $M_3$ in Fig. \ref{fig:nand}). A similar observation applies to NAND 6
contributions to Down. Second, the jitter produced by NAND 9 appears on the rising edges of both Up and
Down pulses and hence is immaterial.
As seen from Fig. \ref{fig:pfdnand}(c), NANDs 1-8 make a total of 10 contributions to the pulsewidth
difference between Up and Down. The phase noise spectral densities of these contributions
are summed to obtain the overall PFD phase noise. 

{In response to the jitter components in the Up and Down pulses (except for those that are in common), the charge pump in Fig.
\ref{fig:pll} produces an error current, $\Delta I$.
Adding up the powers of uncommon jitters, $T_m$, $m=1,...,10$, in the Up and Down pulses, we have
\beq
\overline{\Delta I^2}=\frac {I_p^2}{T_{in}^2}\sum_{m=1}^{10}T_m^2.
\label{deltai2}
\eeq
It can be shown that the transfer function from this current injection to the PLL output within the loop bandwidth is equal to $\Phi_{out,
PLL}/\Delta I =(2\pi/I_p)M$. It follows that 
\beq
S_{\Phi,PLL}(f)=\frac {4\pi^2}{T_{in}^2}M^2\sum_{m=1}^{10}S_{Tm}(f),
\label{sphipll0}
\eeq
where $S_{Tm}(f)$ denotes the spectral density of jitter component $T_m$ and is equal to $S_{Vn}(f)/r_{edge}^2$.
}
For roughly similar gates and rise and fall times, the in-band phase noise observed at the PLL output is given by
\begin{eqnarray}
S_{\Phi,PLL}\!\!\!\!&\cong&\!\!\!\! 10 M^2 [\frac{4\pi^2{\Delta T}}{r_{edge}^2 C_L^2 T_{in}}S_{I}(f) + \frac{4\pi^2{\Delta T}^2}{r_{edge}^2 C_L^2 T_{in}^2}S_{1/f}(f)\nonumber  \\ 
&+&\!\!\!\!\frac{4 \pi^2 f_{in}}{r_{edge}^2}\frac{kT}{C_L}].
\label{sphipll}
\end{eqnarray}
As explained in Section 2.6, however, an optimum design may incorporate different sizings for the gates. 


An important point emerging from our analysis is that, to reduce the flicker noise of a PFD, the channel length
of its constituent transistors must {\em not} be increased. This is because longer-channel devices inevitably raise
$\Delta T$ in (\ref{sphipll}). Instead, the channel area of the transistors can be increased by choosing {\em
wider} devices.

\subsection{TSPC PFD}
The foregoing analysis can be applied to other PFD topologies as well. In this section, we study the phase noise of a TSPC
implementation \cite{tspcpfd} as it can potentially achieve a higher speed and proves useful in cascaded PLLs. Depicted in Fig.
\ref{fig:pfdtspc}(a), the circuit operates as follows. A rising edge on $A$ turns on $M_5$, discharging the Up output. Similarly,
a rising edge on $B$ discharges the Down output. Once both Up and Down are low, Reset rises, discharging nodes
$C$ and $D$ and forcing Up and Down to go high.

\begin{figure}[htb!]
\centering
\includegraphics[scale=1]{FIGS/CH2/fig10.ps}
\caption{(a) TSPC PFD, and (b) jitter contributions to the outputs.}
\label{fig:pfdtspc}
\end{figure}

In a manner similar to the analysis of the NAND PFD, we follow the transitions through the circuit and mark the jitter
contributed by each stage. As illustrated in Fig. \ref{fig:pfdtspc}(b), the falling edges of Up and Down are corrupted by the
noise of the series combinations $M_5$-$M_6$ and $M_{11}$-$M_{12}$, respectively. Next, Reset experiences the jitter
due to $M_{11}$-$M_{12}$ and the NOR gate. The falling transitions at $C$ and $D$ inherit the jitter of Reset and incur
additional noise due to $M_3$ and $M_9$, respectively. Finally, these edges are corrupted by the noise of $M_4$ and
$M_{10}$.

Let us draw several conclusions. First, the jitter of the NOR gate modulates the widths of Up and Down equally and hence
is ignored. Second, the overall TSPC PFD phase noise arises from six transitions and can be potentially smaller than that
of the NAND PFD. Third, the noise injection mechanisms in each stage are similar to those of the inverter and NAND gates
studied earlier. For example, when $M_5$ turns on, its corresponding stage acts approximately like a NAND gate (except
that $M_4$ has been off well before this transition). Also, when node $C$ falls, the series combination 
$M_5$-$M_6$ deposits $kT/C$ noise at the output while $M_4$ turns on as in an inverter and injects both thermal and
flicker noise. Thus, Eq. (\ref{sphipll}) applies here as well if the factor of 10 is replaced by 6 and the gates and rise and fall times are assumed
similar.


\section{Design Optimization}
With the insights developed above into PFD phase noise mechanisms, we now seek to optimize each design for minimum phase
noise. Of course, one can simply enlarge the widths of all of the PFD transistors by a factor of $\alpha$ so as to
reduce the phase noise by the same factor, but at the cost of proportionally higher power consumption. A more methodical approach,
however, is to assume a certain power budget and determine the best sizing of the transistors
that yields minimum phase noise. This optimization can still be followed by the above scaling technique to trade
power for phase noise. We consider $1/f$ noise here as it dominates for offsets as high as 10 MHz, but optimization for
thermal noise is similar.

Since the PFD power dissipation is proportional to the total transistor width in the signal path, $W_{tot}$, we must
determine how a given $W_{tot}$ is apportioned among the transistors so as to minimize the phase noise. Our general
procedure is to favor transistors that define the transition time of critical edges. We also make four approximations: (1)
The capacitance at a given node is proportional to the width of the ``driver'' transistor, $W_a$, and the width of the
``driven'' transistor, $W_b$: $ C_L \propto \eta W_a + W_b$. The first term on the right accounts for the drain junction
capacitance and the Miller multiplication of the gate-drain overlap capacitance at the output node (about a factor of 2).
(2) The drain $1/f$ noise current spectrum is given by $S_{1/f}(f)=g_m^2 K_f/(W_a L_a C_{ox} f)$, where $g_m \approx I_D /
(V_{GS}-V_{TH})$ and $V_{GS}=V_{DD}$.\footnote{This $g_m$ equation assumes heavy velocity saturation. For long-channel
devices, $g_m \approx 2 I_D /(V_{GS}-V_{TH})$. This distinction is not critical in our analysis.} (3) At the point of
interest, namely, $V_{GS} \approx V_{DD}$ and $V_{out} \approx V_{DD}/2$, we have $I_D \propto W_a$ regardless of the
transistor (short-channel) characteristics. Thus, the slew rate in Eq. (\ref{sphif}), $r_{edge} \propto I_D / C_L \propto W_a /
C_L$. (4) The window width, $\Delta T$, is proportional to $V_{DD}/r_{edge} \approx V_{DD}C_L/I_D$. Equation (\ref{sphif}) is
now rewritten as
\beq
S_{\Phi}(f) \propto \frac{{f_{in}}^2 {V_{DD}}^2 {C_L}^2}{{W_a}^3}\frac{1}{f}.
\label{sphifsimple}
\eeq
For given values of $f_{in}$, $V_{DD}$, and $f$,
\beq
S_{\Phi}(f) \propto \frac{(\eta W_a + W_b)^2}{{W_a}^3}.
\label{sphifprop}
\eeq
The power consumed to charge and discharge such a node once per cycle is approximately equal to $P = f_{in} C_L
{V_{DD}}^2$. We now apply these results to the optimization of the NAND and TSPC PFDs.

\subsection{NAND PFD Optimization}
As evident from Figs. \ref{fig:pfdnand}(b) and (c), the NAND PFD phase noise arises from five transistors: the PMOS device in NAND 1,
the NMOS device in NAND 2, the PMOS device in NAND 4, the NMOS device in NAND 3, and the PMOS device in NAND 2. Denoting
the widths of PMOS and NMOS transistors in NAND $j$ by $W_{Pj}$ and $W_{Nj}$, respectively, we use Eq. (\ref{sphifprop}) to express the
first PMOS contribution as:
\beq
S_{\Phi _1}(f) \propto \frac{[\eta (2 W_{P1} + W_{N1}) + W_2 + W_3 + W_9]^2}{{W_{P1}}^3}.
\label{sphin1}
\eeq
Here, the factor of 2 accounts for the two PMOS devices tied to the output and $W_j = W_{Pj} + W_{Nj}$. The sum $W_2 + W_3
+ W_9$ represents the load due to the three NANDs driven by NAND 1. The other four contributions can be expressed in a
similar manner, e.g., for the NMOS device in NAND 2:
\beq
S_{\Phi _2}(f) \propto \frac{[\eta (2 W_{P2} + W_{N2}) + W_{P1}+ W_{N1}]^2}{{W_{N2}}^3}.
\label{sphin2}
\eeq
Note that the proportionality factors relating the right-hand sides of (\ref{sphin1}) and (\ref{sphin2}) to their left-hand side are
different as they include the mobility and flicker noise coefficient of PMOS and NMOS devices, respectively. The total power consumption
satisfies the relation:
\beq
P \propto f_{in} {V_{DD}}^2 [W_{P9} + W_{N9} + 2 \sum_{j=1}^{4} (W_{Pj} + W_{Nj})].
\label{powern}
\eeq

As explained in Section 2.5, the jitter of some of the edges does not enter the overall PFD phase noise. The transistors
causing these edges can therefore have nearly minimum widths so long as they respond fast enough to avoid circuit failure. The devices falling into this
category are the NFETs in NANDs 1, 4, and 9 and the PFETs in NANDs 3 and 9. 
The sum of the five phase noise contributions described above must be minimized subject to the power budget imposed by
(\ref{powern}). This is accomplished using the ``fmincon'' function in MATLAB. For example, a total width of 162 $\mu$m {(corresponding to
0.24 mW at 1 GHz)} for the transistors yields
$W_{P1} = 11$, $W_{N1} = 0.12$, $W_{P2} = 9.1$, $W_{N2} = 5.9$, $W_{P3} = 0.12$, $W_{N3} = 6.22$, $W_{P4} = 7.8$, $W_{N4} = 0.12$,
$W_{P9} = 0.12$, $W_{N9} = 0.12$, all in microns. Using transient circuit simulations, we adjust some of the noncritical transistors widths
so to minimize crowbar currents and speed up the critical transitions, obtaining 
$W_{P1} = 10.6$, $W_{N1} = 0.5$, $W_{P2} = 8.5$, $W_{N2} = 5.5$, $W_{P3} = 0.6$, $W_{N3} = 5.84$, $W_{P4} = 7.4$, $W_{N4} = 0.5$,
$W_{P9} = 0.12$, $W_{N9} = 2$, all in microns. It is interesting that such a range of widths would not be obvious if we attempted to manually
optimize the PFD transistors by trial and error. As shown in Section 2.7, this optimization lowers the phase noise by 4 to 6 dB.

\subsection{TSPC PFD Optimization}
The foregoing procedure can be applied to the TSPC PFD of Fig. \ref{fig:pfdtspc}(a) as well. Here the phase noise has three contributions arising from
$1/f$ noise:
\beq
S_{\Phi _1}(f) \propto \frac{[\eta (W_4 + W_5) + W_{P,NOR}  + W_{N,NOR}]^2}{{W_5}^3},
\label{sphit1}
\eeq
where $W_j$ refers to the width of $M_j$ and $W_{P,NOR}$ and $W_{N,NOR}$ are the PMOS and NMOS widths in the NOR gate, respectively. The
power consumption satisfies the relation:
\beq
P \propto f_{in} {V_{DD}}^2 (2 \sum_{j=1}^{6} W_j + W_{P,NOR}  + W_{N,NOR}).
\label{powert}
\eeq
For simplicity, we assume equal widths for the transistors within each cascode structure. Also, $M_1$-$M_2$ and $M_7$-$M_8$ in Fig.
\ref{fig:pfdtspc}(a) contribute no
jitter to the PFD and hence can have small widths. For example, a total width of 162 $\mu$m {(corresponding to
0.2 mW at 1 GHz)} is apportioned as follows:
$W_1 = 0.12, W_2 = 0.12, W_3 = 28,
 W_4 = 25, W_5 = 13.72, W_6 = 13.72, 
 W_{P,NOR} = 0.12, W_{N,NOR} = 0.12$,
all in microns. Manual adjustment to improve transition times in the simulations yields
$W_1 = 1.4, W_2 = 1.4, W_3 = 12, 
 W_4 = 24, W_5 = 10, W_6 = 10,
 W_{P,NOR} = 10, W_{N,NOR} = 0.12$, all in microns. As discussed in Section 2.7, this optimization reduces the phase noise by 5 to 8 dB.

\subsection{Dependence on Operation frequency}
Equation (\ref{sphipll}) reveals that the phase noise of PFDs rises in proportion to $f_{in}$ in the thermal regime and
${f_{in}}^2$ in the flicker noise regime. This dependence imposes certain bounds on the in-band phase noise of PLLs. For a feedback divide
ratio of $M$, the first term in Eq. (\ref{sphipll}) yields an output phase noise of
\begin{eqnarray}
S_{\Phi, out}(f) &\propto& f_{in} M^2 S_{I}(f) \nonumber \\ &\propto& \frac{{f_{out}}^2} {f_{in}} S_{I}(f).
\label{sphiwfreq}
\end{eqnarray}
That is, to minimize the phase noise due to the PFD thermal noise, $f_{in}$ must be maximized. For PFD flicker noise, on the other hand, 
\begin{eqnarray}
S_{\Phi, out}(f) &\propto& {f_{in}}^2 M^2 S_{1/f}(f) \nonumber \\ &\propto& {{f_{out}}^2} S_{1/f}(f).
\label{sphiffreq}
\end{eqnarray}
Interestingly, this PFD contribution is independent of the input frequency so long as flicker noise does not experience aliasing.




\section{Simulation Results}
This section presents simulation results in 65-nm CMOS technology for the circuits studied in this chapter and compares them with our
analytical derivations. The objective is threefold: (a) validate the trends
predicted by our analysis, e.g., the dependence of phase noise upon the input frequency and node capacitance, (b) check the absolute accuracy
of the analytical results, and (c) examine the soundness of our optimization procedure.

A few remarks with respect to the hand calculations are warranted. First, the transistor capacitances, drain bias currents, and drain (1/f and
thermal) noise currents are obtained from ac and transient simulations for various values of $V_{GS}$ and $V_{DS}$. These simulations also reveal the peak
noise current and the gate-source voltage, $V_{GS,half}$, at which the noise current is equal to half of its peak. Second, the window width,
$\Delta T$, in Eqs. (\ref{sphiw}), (\ref{sphif}), (\ref{sphitot}) and (\ref{sphipll}) is derived from transient simulations of the stage of interest by finding the time at
which the gate-source voltage reaches $V_{GS,half}$.

\subsection{Inverter and NAND Simulations}
Figure \ref{fig:inverterchain} plots the phase noise of a chain of eight inverters with $W_P=6\ \mu$m and $W_N=3\ \mu$m at an input frequency of 1 GHz. (As
explained in Section 2.7.2, scaling to other frequencies is straightforward.) In
order to investigate the robustness of our analytical approach, the chain is also studied with an additional node capacitance of 20
fF. In each case, the results of Cadence pnoise simulations are compared with those of hand calculations. Figure \ref{fig:nandchain} repeats these
experiments for a chain of eight NAND gates with one input tied to $V_{DD}$ and $W_P=W_N=6\ \mu$m. We observe
that in all cases, the hand calculations incur an error of less than 2 dB.

\begin{figure}[htb!]
\centering
\includegraphics[scale=0.6]{FIGS/CH2/fig11.eps}
\caption{Phase noise of a chain of eight inverters running at 1 GHz.}
\label{fig:inverterchain}
\end{figure} 

\begin{figure}[htb!]
\centering
\includegraphics[scale=0.6]{FIGS/CH2/fig12.eps}
\caption{Phase noise of a chain of eight NANDs running at 1 GHz (with one input tied to $V_{DD}$).}
\label{fig:nandchain}
\end{figure} 


\subsection{PFD Simulations}
As argued in Section 2.2, the PFD phase noise cannot be simulated by examining only the Up or Down pulses. For this reason, we embed the
PFD within an otherwise ideal PLL, run a pss and pnoise analysis, allow the PLL to settle, and compute the output phase noise
of the PLL in the steady state. If the PLL bandwidth is large enough, the PFD phase noise up to the offset frequencies of interest passes to
the output unattenuated. Such a simulation takes a long time but is necessary here to demonstrate the validity of our approach.
The PLL comprises behavioral descriptions of the VCO, frequency divider, and charge pump. The loop filter employs a noiseless resistor. To ensure
that the PLL does not attenuate the PFD phase noise for offset frequencies as high as 100 MHz, the reference frequency, $f_{ref}$, is chosen equal to or
greater than 1 GHz. 
{Such a high value is chosen so as to readily observe and validate the effect of flicker noise. 
For much lower input frequencies, the aliasing of white noise tends to mask the effect of flicker noise, making it difficult to correlate the
simulations with the analytical results. For example, if $f_{ref}$ is reduced to 20 MHz, then the effect of flicker noise rises by $10\log(50)=17$
dB and that of white noise by $20\log(50)=34$ dB, masking the former.} 




Figure \ref{fig:pfdnandsim} plots the simulated and calculated phase noise of the NAND PFD for different input frequencies. (Each simulation
incorporates a different set of PLL parameters\footnote{For example, $R_1=600~\Omega$, $C_1=200$ pF, $C_2=100$ fF, $I_p=1$ mA,
$M=1$,
and $K_{VCO}=2\pi(1.5\times 10^9)$ rad/s.} commensurate with the reference frequency.) As predicted in
Section 2.3, doubling $f_{ref}$ raises
the phase noise by 6 dB in the 1/f noise regime and by 3 dB in the white noise regime. The error in the analytical calculations is 3.1 dB. 
{The effect of white noise is overestimated possibly due to assuming that all of the high-frequency noise components experience only a
$ sinc^2$ envelope before folding, whereas in the actual circuit, these components are also attenuated by the finite bandwidth and hence do not
extend to infinity.
}

Figure \ref{fig:pfdtspcsim} plots similar results for the TSPC PFD. The maximum error in this case is 2.8 dB. Designed for the same power
consumption as the NAND PFD, the TSPC topology exhibits about 6 dB lower phase noise.

Illustrated in Fig. \ref{fig:pfdoptim} are the results of the optimization procedure described in
Section 2.6. For a given power consumption, the phase noise is
reduced by 4 to 8 dB for the two PFDs.

\begin{figure}[htb!]
\centering
\includegraphics[scale=0.6]{FIGS/CH2/fig13.eps}
\caption{Phase noise of NAND PFD at various input frequencies.}
\label{fig:pfdnandsim}
\end{figure} 
\begin{figure}[htb!]
\centering
\includegraphics[scale=0.6]{FIGS/CH2/fig14.eps}
\caption{Phase noise of TSPC PFD at various input frequencies.}
\label{fig:pfdtspcsim}
\end{figure} 
\begin{figure}[htb!]
\centering
\includegraphics[scale=0.6]{FIGS/CH2/fig15.eps}
\caption{Phase noise of NAND and TSPC PFDs before and after optimization.}
\label{fig:pfdoptim}
\end{figure} 




\section{Effect of Pulse Position Modulation}
In this section, we show that if noise modulates only the position of the Up or Down pulses, the resulting phase noise is negligible. Consider the waveforms
depicted in Fig. \ref{fig:ppmcompare}(a), where Up and Down have a pulsewidth of $T_{RST}$ and a random skew of $T_{skew}$. Assuming an ideal charge pump, we note that the
disturbance on the oscillator control voltage is in the form of a pulse with a mean width of $T_{RST}$. By contrast, as shown in Fig. \ref{fig:ppmcompare}(b), a pulsewidth
difference of $T_D$ between Up and Down manifests itself as a step on the control voltage, producing a much larger phase disturbance. 
\begin{figure}[htb!]
\centering
\includegraphics[scale=1]{FIGS/CH2/fig16.ps}
\caption{Modulation of (a) position, and (b) pulsewidth of Up and Down signals.}
\label{fig:ppmcompare}
\end{figure}

\section{Phase Noise of Square Wave with Uncorrelated Jitters on Rising and Falling Edges}
It is usually assumed that an edge displacement of $\Delta T$ translates to a phase change of $2\pi \Delta T / T_{in}$, where $T_{in}=1/f_{in}$ denotes the period. Of
course, if {\em all} of the edges of a square wave are displaced by $\Delta T$, this amount of phase change arises. However, jitter affects the consecutive edges
differently, requiring a closer look at the resulting phase noise.

Let us first suppose a sinusoidal jitter, $T_m \cos \omega_m t$, is applied to only the rising edges of an ideal square wave, $p(t)$. As shown in Fig.
\ref{fig:squaretonepn}(a), the rising edge at $k T_{in}$ is displaced by an amount equal to $T_m \cos (\omega_m k T_{in})$.
\begin{figure}[htb!]
\centering
\includegraphics[scale=1]{FIGS/CH2/fig17.ps}
\caption{(a) Square wave with modulated rising edges, (b) decomposition into two waveforms, and (c) resulting magnitude of Fourier transform.}
\label{fig:squaretonepn}
\end{figure}
This jittery waveform can be expressed as the sum of $p(t)$ and a train of pulses that occur at $k T_{in}$ with a width of $T_m \cos (\omega_m k T_{in})$
[Fig. \ref{fig:squaretonepn}(b)]. If $T_m \ll T_{in}$, the latter can be approximated by a train of impulses and expressed as
\beq
q(t) \approx T_m  \sum_{k=-\infty}^{k=+\infty} \cos (\omega_m k T_{in})\;\delta (t-k{T_{in}}).
\label{qt}
\eeq
Adding the Fourier transforms of $p(t)$ and $q(t)$, we obtain the result shown in Fig. \ref{fig:squaretonepn}(c), where each harmonic of the square wave is
surrounded by two impulses of area $T_m / (2T_{in})$ at frequency offsets of $\pm f_m= \pm \omega _m / (2 \pi)$. It can be shown that these sidebands
generate only phase modulation (PM).

We thus observe that a jitter spectrum consisting of two impulses having an area of $T_m/2$ produces two PM sidebands around $f_{in}$ whose normalized
magnitude is equal to $\pi T_m / (2 T_{in})$. That is, a jitter of $T_m/2$ yields a phase disturbance of $(\pi /T_{in})(T_m /2)$ rather than $(2 \pi
/T_{in})(T_m /2)$ in this case. One may expect this result because only the rising edges have been displaced.

We now generalize the foregoing observation to random jitter, while still assuming jitter on only the rising edges. If the jitter itself in the time
domain is denoted by $\sigma (t)$, then Eq. (\ref{qt}) is rewritten as 
\beq
q(t) \approx  \sigma (t) \sum_{k=-\infty}^{k=+\infty} \delta (t-k{T_{in}}).
\label{qtrandom}
\eeq
Adding the power spectral densities of $p(t)$ and $q(t)$, we obtain the overall spectrum shown in Fig. \ref{fig:psdrising}. Thus, the jitter spectrum,
$S_\sigma(f)$, is shifted to $\pm f_{in}$, $\pm 2f_{in}$, etc., scaled by a factor of $1/{T_{in}}^2$, and normalized to a carrier power of $1/\pi
^2$, yielding $(\pi^2/{T_{in}}^2)S_\sigma(f \pm f_{in})$, etc., for the phase noise.\footnote{Using Rice's approximation of random noise by a sum of
sinusoids \cite{Rice}, it can be proved that the spectra at $\pm f_{in}$ produce only phase modulation.} MATLAB simulations confirm this result.
\begin{figure}[htb!]
\centering
\includegraphics[scale=1]{FIGS/CH2/fig18.ps}
\caption{Spectrum of jittery square wave.}
\label{fig:psdrising}
\end{figure}

Since the jitters on the rising and falling edges of a CMOS inverter's output are generated by different transistors and are hence uncorrelated, we
write the overall phase noise of the square wave as
\beq
S_\Phi(f) = \frac{\pi^2}{{T_{in}}^2}\sum_{k=-\infty}^{k=+\infty} [S_{\sigma p}(f \pm k f_{in}) + S_{\sigma n}(f \pm k f_{in})],
\label{sphiuncorrelated}
\eeq
where $S_{\sigma p}$ and $S_{\sigma n}$ denote the spectra of the jitters produced by the PMOS and NMOS transistors, respectively. Note that $S_\sigma$ and
$S_{Vn}$ are simply related by a factor of $r_{edge}^2$.


\section{Spectrum of Shaped and Sampled White Noise}
In this section, we examine the phase noise spectrum due to white noise:
\beq
S_{\Phi}(f)=\frac{\pi^2}{r_{edge}^2 T_{in}^2}\sum_{m=-\infty}^{m=+\infty} S_{Vn}(f-\frac{m}{T_{in}}).
\label{sphia1}
\eeq
Since the Fourier transform of the rectangular window, $w(t)$, is given by $\Delta T\times$ $\sin(\pi f \Delta T)$$/(\pi f \Delta T)$, we have from (\ref{svn})
\beq
S_{Vn}(f)=\frac{1}{C_L^2}{\Delta T}^2 \frac{\sin^2(\pi f \Delta T)}{(\pi f \Delta T)^2}S_{In}(f).
\label{svna}
\eeq
If $S_{In}(f)$ is white, then $S_{Vn}(f)$ has a $ sinc^2$ shape; i.e., $S_{\Phi}(f)$ consists of $ sinc^2$ functions centered at $m f_{in}=m/T_{in}$.
We now prove that the sum of these $ sinc^2$ functions is a flat line under a certain condition.

\begin{figure}[htb!]
\centering
\includegraphics[scale=1]{FIGS/CH2/fig19.ps}
\caption{Inverse Fourier transform of (a) $ sinc^2$ function, and (b) shifted $ sinc^2$ functions.}
\label{fig:sinc2}
\end{figure}

Considering only the $ sinc^2$ shape itself, we recognize that the inverse Fourier transform of ${\Delta T}^2 { sinc^2}(\pi f \Delta T)$ is a triangle,
$g(t)$, with a time duration of $-\Delta T$ to $+\Delta T$ and a height of ${\Delta T}$ [Fig. \ref{fig:sinc2}(a)]. As a result of shifts of $ sinc^2$ by $m
f_{in}$ in the frequency domain, $g(t)$ is multiplied by ${ exp}(j2\pi m f_{in}t)$ in the time domain:
\beq
g(t)\sum_{m=-\infty}^{m=+\infty} e^{j2\pi m f_{in}t} \leftrightarrow \sum_{m=-\infty}^{m=+\infty}{\Delta T}^2 \frac {\sin^2[\pi \Delta
T(f-mf_{in})]}{[\pi \Delta T(f-mf_{in})]^2}.
\label{ga}
\eeq
We also note that
\beq
\sum_{m=-\infty}^{m=+\infty} e^{j2\pi m f_{in}t} = \frac{1}{f_{in}} \sum_{m=-\infty}^{m=+\infty} \delta (t-mT_{in}).
\label{suma}
\eeq
In other words, $g(t)$ is multiplied by a train of impulses centered at $m T_{in}$ [Fig. \ref{fig:sinc2}(b)]. Thus, if the duration of $g(t)$ is short enough
to enclose only the impulse at $t=0$, we have 
\beq
g(t)\sum_{m=-\infty}^{m=+\infty} e^{j2\pi m f_{in}t} = \Delta T \frac{1}{f_{in}} \delta (t).
\label{gsuma}
\eeq
The Fourier transform of this result is equal to $\Delta T / f_{in}$ and hence:
\beq
\sum_{m=-\infty}^{m=+\infty}{\Delta T}^2 \frac {\sin^2[\pi \Delta T(f-mf_{in})]}{[\pi \Delta T(f-mf_{in})]^2} = \frac{\Delta T}{f_{in}},
\label{suma2}
\eeq
which is a flat line.

In summary, if the sampling period, $T_{in}$, is greater than the rectangular window width, $\Delta T $, then the window-integrated and sampled white noise
still has a white spectrum. Note that this result is valid for any shape chosen for $w(t)$ so long as the inverse Fourier transform of $|W(f)|^2$
has a total time duration less than $2 T_{in}$, or more generally, so long as the inverse Fourier transform of $|W(f)|^2$ crosses zero at $t=m
T_{in}$ except for $t=0$.




\section{Conclusion}
The phase noise of PFDs can manifest itself within the bandwidth of PLLs, corrupting the transmitted and received signal constellations. This chapter
analyzes the phase noise of two PFD topologies based on the approximations made for a CMOS inverter. It is also shown that the PFD phase noise is
not merely that of the Up and Down pulses. Simulations using each PFD in a PLL reveal good agreement with analytical predictions, indicating, most
notably, the dependence of the phase noise on the frequency of operation.

                         % Chapter 2
\chapter{Relation Between Delay Line Phase Noise and Ring Oscillator Phase Noise}                        


The phase noise of a ring oscillator can be obtained by multiplying its open-loop phase noise by a simple shaping function. The shaping
function is computed using first principles and is applicable to both flicker-noise-induced and white-noise-induced phase noise, leading to
compact equations for ring oscillators. It is also shown that flicker noise upconversion in ring oscillators is primarily a function of the total gate capacitance and inevitable
regardless of the risetime and falltime symmetry. Two oscillator prototypes fabricated in 65-nm CMOS technology verify the validity of the results.







\section{Introduction}

It has been recognized for more than two decades that delay lines exhibit less phase noise than ring oscillators do \cite{Sonntag}. This advantage
is intuitively explained by the lack of jitter accumulation in the former but has not been quantified analytically.

The phase noise in ring oscillators has been studied extensively \cite{Weigandt}-\cite{Grozing}.
In this chapter, we offer an analysis that leads to a direct relation between the phase noise of delay lines and that of ring oscillators, allowing comparison of
their performance for a given power dissipation and operation frequency. 
We begin with first principles and establish a unified relation for both white and $1/f$ noise sources.
As a byproduct, our analysis also shows that the flicker-noise-induced phase noise is inversely proportional to the total gate capacitance present in a ring oscillator
and relatively independent of the symmetry between rise and fall transitions.
The proposed relation is experimentally verified on 9-stage and 19-stage prototypes fabricated in 65-nm CMOS technology.




Section 3.2 deals with the phase noise of delay lines, expressing their jitter as two impulse trains. Section 3.3 analyzes jitter accumulation in a ring oscillator and
utilizes the results from Section 3.2 to arrive at the the proposed relation. Section 3.4 derives some useful results, including compact phase noise equations,
and Section 3.5 and 3.6 respectively present simulation and experimental confirmations of the equations. Section 3.7 concludes this chapter.
 
 
\section{Phase Noise of Delay Lines}

Let us consider the chain of inverters shown in Fig. \ref{fig:chain}(a) as a representative delay line, with the dummy load added to ensure uniform delays. Since the inverters
exhibit uncorrelated noise, the overall phase noise (as a power quantity) is equal to that of one multiplied by the number of stages (if they are identical). 
\begin{figure}[htb!]
\centering
\includegraphics[scale=1]{FIGS/CH3/3inverters.ps}
\caption{(a) Three-stage delay line with only one noisy inverter, (b) node voltages in response to a frequency equal to the oscillation frequency of a three-stage ring
oscillator, (c) decomposition of the output voltage to an ideal noiseless square wave and a noise waveform, and (d) approximation of the noise waveform in (c) to two
uncorrelated weighted impulse trains.}
\label{fig:chain}
\end{figure}

 
For our purposes, we tentatively assume that only the second inverter in Fig. \ref{fig:chain}(a) has noise. We also select the input frequency equal to the oscillation
frequency of this chain as if it were reconfigured to become a ring oscillator, i.e., $f_{in}=1/(6 T_d)$, where $T_d$ denotes the average gate delay. 
Thus, as $V_0$ propagates to $V_3$, it experiences three gate delays and the jitter of one inverter [Fig. \ref{fig:chain}(b)]. In other words, the falling edges of $V_3$ are
aligned with the falling edges of $V_0$ but modulated by the second inverter's jitter.
 
The output of the third inverter in Fig. \ref{fig:chain}(a) can be decomposed into an ideal square wave and a train of narrow pulses \cite{Perrott,Galton} that occur every $3 T_d=T_{in}/2$
seconds [Fig. \ref{fig:chain}(c)]. Since the jitters on the rising and falling edges arise from different noise sources and are
uncorrelated \cite{Homayoun}, we denote them by $\sigma_R(t)$ and $\sigma_F(t)$, respectively. Now, $V_{3,n}$ in Fig. \ref{fig:chain}(c) itself can be approximated as the
sum of a positive impulse train weighted by $\sigma_F(t)$ and a negative impulse train weighted by $\sigma_R(t)$ [Fig. \ref{fig:chain}(d)]: 
\ber
V_{3,n}(t) &=& V_{3,n,F}(t) + V_{3,n,R}(t) \nonumber\\
&=& \SumAll n \sigma_F [(2n+1)\frac{T_{in}}{2}] ~\delta [t-(2n+1)\frac{T_{in}}{2}] \nonumber\\
&-& \SumAll n \sigma_R (2n \frac{T_{in}}{2}) ~\delta (t-2n\frac{T_{in}}{2})
\label{v3n}
\eer
With the aid of Fig. \ref{fig:chain}(d), we recognize that 
the phase noise of the chain is equal to the sum of the power spectral densities of $V_{3,n,F}$ and $V_{3,n,R}$ normalized to the power of the first
harmonic of $V_{3,ideal}$ \cite{Homayoun}. We derive the phase noise expression in Section 3.4.


 
\section{Phase Noise of Ring Oscillators}

The perspective described above for the phase noise of delay lines proves useful in the phase noise analysis of ring oscillators as well. Suppose the delay line of
Fig. \ref{fig:chain}(a) is reconfigured to form a ring oscillator as shown in Fig. \ref{fig:ring}(a) (without the dummy load).  
\begin{figure}[htb!]
\centering
\includegraphics[scale=1]{FIGS/CH3/3stagering.ps}
\includegraphics[scale=0.75]{FIGS/CH3/3stage_ring.ps}
\caption{(a) Three-stage ring oscillator retimed at $t=0$ with only one noisy inverter, (b) jitter on all edges due to a single jitter event on $V_2$, (c) decomposition of $V_3$ in (b) to an ideal noiseless square
wave and a noise waveform, with $g(t)$ serving as a ``carrier,''
(d) jitter on edges when inverter \#2 adds jitter on every transition, (e) decomposition of $V_3$ in (d) to an ideal noiseless square wave and a noise waveform.}
\label{fig:ring}
\end{figure}
 
We perform a ``gedankenexperiment'' in which (1) the voltage source $V_{in}$ applies a noiseless rising edge to the input of the first inverter at $t=0^-$ and is
disconnected from the circuit at $t=0$, and (2) the second inverter produces jitter only once (i.e., a single time displacement) as this edge propagates through the chain and remains noiseless
thereafter. Thus, the input rising edge arrives at $V_3$ with a delay equal to $3T_d$ plus the jitter of the second inverter, $\Delta T_1$. As this edge circulates
around the ring, it experiences no more jitter; i.e., all of the subsequent edges are simply displaced by a constant equal to $\Delta T_1$. Figure \ref{fig:ring}(b)
illustrates this effect.

The output waveform obtained in the above experiment can be decomposed as shown in Fig. \ref{fig:ring}(c) and expressed as a single pulse of width $\Delta T_1$, convolved with an
alternating train of impulses, $g(t)$. Note that $g(t)=0$ for $t<0$. We can consider $g(t)$ as ``carrier'' for the time displacements.

We now repeat the above experiment while assuming that the second inverter is noisy at all times. The second time the oscillation edge passes through this inverter,
the jitter causes one additional displacement, $\Delta T_2$, as depicted by the dark shading in Fig. \ref{fig:ring}(d). The effect of this shift can be obtained by
convolving a pulse of width $\Delta T_2$ with $g(t)$ and adding the result to an ideal, noiseless waveform. Note that this calculation holds valid whether or not
$\Delta T_1$ and $\Delta T_2$ are correlated.

The foregoing observations suggest that the ring oscillator output can be decomposed into an ideal square waveform and a noise component [Fig. \ref{fig:ring}(e)] given by
\ber
V_{n,ring}(t) &=& \left\{\SumAll n \sigma_R [(2n+1)\frac{T_{in}}{2}] ~\delta [t-(2n+1)\frac{T_{in}}{2}]\right\}*g(t) \nonumber\\
&-& \left\{\SumAll n \sigma_F (2n \frac{T_{in}}{2}) ~\delta (t-2n\frac{T_{in}}{2})\right\}*g(t).
\label{vnring}
\eer
From (\ref{v3n}) and (\ref{vnring}), it follows that the delay line phase noise, $S_{\Phi,DL}(f)$, and the ring oscillator phase noise, $S_{\Phi,ring}(f)$, are related
as\footnote{Throughout this chapter, all the spectra are two-sided, and the phase noise is denoted by $S_{\Phi}$.}
\beq
S_{\Phi,ring}(f)=S_{\Phi,DL}(f)|G(f)|^2,
\label{keyrelation}
\eeq
where $|G(f)|^2$ denotes the spectrum of $g(t)$.

Equation (\ref{keyrelation}) is a general result and merits a few remarks. First, (\ref{keyrelation}) applies to the phase noise due to both white noise and flicker
noise. Second, (\ref{keyrelation}) holds for the phase noise arising from {\em all} of the devices in the delay line and the ring. Third, (\ref{keyrelation}) is not limited to CMOS
inverters and can be used for differential delay stages and rings as well.

To determine $|G(f)|^2$, we first write
\beq
g(t)=\left[ \SumAll n \delta(t-nT_{in}) - \SumAll n \delta(t-nT_{in}-\frac{T_{in}}{2}) \right] u(t),
\label{gt1}
\eeq
and hence
\beq
G(f)=\left[\frac{1}{T_{in}} ~ (1 - e^{-j \pi f T_{in}}) \SumAll n \delta (f-n f_{in})\right] * \left[ \frac{1}{j2\pi f}+\frac{1}{2}\delta(f) \right],
\label{gf1}
\eeq
which simplifies to 
\beq
G(f)=\left\{\frac{2}{T_{in}} \SumAll n \delta [f-(2n+1)f_{in}]\right\} * \left[ \frac{1}{j2\pi f}+\frac{1}{2}\delta
(f) \right].
\label{gf2}
\eeq
The unit step in (\ref{gt1}) ensures the causality of jitter accumulation, i.e., the jitter generated at any edge is present for only subsequent edges.
Figure \ref{fig:chain2ring} plots the magnitude of $G(f)$, revealing how the delay line phase noise is shaped to produce the ring oscillator phase noise.
\begin{figure}[htb!]
\centering
\includegraphics[scale=1]{FIGS/CH3/chain2ring.ps}
\caption{Fourier transform of $g(t)$.}
\label{fig:chain2ring}
\end{figure}

At an offset frequency of $\Delta f$ with respect to the fundamental frequency, $f_{in}=1/T_{in}$, we have
\ber
G(f_{in}+\Delta f)&=&\frac{2} {T_{in}} \SumAll n \frac{1}{j2\pi (\Delta f+ 2n f_{in})} \nonumber\\
&=&\frac{1}{j2\pi} \left\{ \frac{2 f_{in}}{\Delta f}+\sum_{n = 1}^{\infty} \frac{\Delta f / f_{in}}{[{\Delta f}/({2f_{in}})]^2-n^2} \right\}  \nonumber\\
&=&\frac{1}{j2\pi} \left[ \pi \cot(\frac{\pi \Delta f}{2f_{in}}) \right].
\label{gf3}
\eer
Thus, Eq. (\ref{keyrelation}) can be rewritten as 
\beq
S_{\Phi,ring}(\Delta f)=S_{\Phi,DL}(\Delta f)\frac{1}{4}\cot^2(\frac{\pi \Delta f}{2f_{in}}).
\label{exactrel}
\eeq
For offset frequencies much less than $f_{in}$, we have $\cot^2[\pi \Delta f/(2f_{in})] \approx [2f_{in}/(\pi \Delta f)]^2$. Changing our notation from $f_{in}$ to
$f_{osc}$, we write
\beq
S_{\Phi,ring}(\Delta f) = S_{\Phi,DL}(\Delta f) \left(\frac{f_{osc}}{\pi \Delta f} \right)^2.
\label{approxrel}
\eeq
This simple, fundamental relation holds for phase noise due to both $1/f$ and white noise.

\section{Useful Insights}

Equation (\ref{approxrel}) provides a multitude of interesting and useful insights into the phase noise behavior of ring oscillators.
Of course, it confirms that white noise and flicker noise lead to $1/\Delta f^2$ and $1/\Delta f^3$ phase noise profiles because the corresponding delay line phase
noise profiles are respectively flat and proportional to $1/\Delta f$ \cite{Homayoun}. This section presents some other insights that may benefit the circuit designer.

\subsection{Comparison of Delay Lines and Ring Oscillators}

Equation (\ref{approxrel}) indicates that conversion of a delay line to a ring oscillator shapes the phase noise by an $f_{osc}^2/(\pi \Delta f)^2$ function. 
Since $\Delta f$ is usually much less than $f_{osc}/\pi$, we observe that $S_{\Phi,ring}(\Delta f) \gg S_{\Phi,DL}(\Delta f)$
for a given power dissipation and fundamental frequency.
Why are low
noise frequencies scaled by a greater factor? 
\begin{figure}[htb!]
\centering
\includegraphics[scale=1]{FIGS/CH3/lowfreqamp.ps}
\caption{(a) Delay line and ring oscillator with one equivalent noise source, $V_{n1}$, and (b) $V_{n1}$ shown as a low-frequency component.}
\label{fig:lowfreqamp}
\end{figure}
Consider the scenario depicted in Fig. \ref{fig:lowfreqamp}(a), where one of the noise sources of the second inverter, $V_{n1}$, is
explicitly shown and placed in series with $V_{DD}$; for example, $V_{n1}$ represents the noise of the PMOS transistor in the inverter. Suppose $V_{n1}$ varies at a
rate much lower than the operation frequency, $f_{in}$ [Fig. \ref{fig:lowfreqamp}(b)]. We observe that the delay line simply experiences a relatively constant phase shift at $t=kT_{in}/2$,
$t=(k+1)T_{in}/2$, etc., so long as $V_{n1}$ changes negligibly. In the ring oscillator, on the other hand, the time displacements caused by $V_{n1}$ at
$t=kT_{in}/2$, $t=(k+1)T_{in}/2$, etc., continue to accumulate until $V_{n1}$ changes polarity. The lower the frequency of $V_{n1}$, the longer and larger this
accumulation is, producing the $1/\Delta f^2$ shaping function.



\subsection{Compact Phase Noise Equations}

The phase noise of an
inverter is derived in \cite{Homayoun} as
\beq
S_{\Phi,white}=\left\{\frac{\pi^2 \Delta T}{r_{edge}^2 C_L^2 T_{in}}S_{I}(f)\right\}_{NMOS} + \left\{\frac{\pi^2 \Delta T}{r_{edge}^2 C_L^2
T_{in}}S_{I}(f)\right\}_{PMOS} + \frac{2 \pi^2}{r_{edge}^2 T_{in}}\frac{kT}{C_L}.
\label{sphiwinverter}
\eeq
for white noise sources and as 
\beq
S_{\Phi,1/f}=\left\{\frac{\pi^2 {\Delta T}^2}{r_{edge}^2 C_L^2 T_{in}^2}S_{1/f}(f)\right\}_{NMOS} + \left\{\frac{\pi^2 {\Delta T}^2}{r_{edge}^2 C_L^2 T_{in}^2}S_{1/f}(f)\right\}_{PMOS}
\label{sphi1/finverter}
\eeq
for flicker noise sources, where $r_{edge}$ is the slew rate, $C_L$ the load capacitance, $T_{in}$ the input period, $S_I(f)$ the thermal noise
current, $S_{1/f}(f)$ the flicker noise current, $k$ the Boltzmann constant, $T$ the absolute temperature, and $\Delta T$ the equivalent ``on'' time for each
transistor \cite{Homayoun}.

In order to derive a compact expression for the delay line, we make three simplifying assumptions. 
(1) The equivalent on time, $\Delta T$, is approximately equal to the gate delay, $T_d$.
(2) The slew rate, $r_{edge}$, can be approximated as $I_D/C_L$, where $I_D$ denotes the drain current of the on transistor when its 
gate voltage is near the rail and its drain voltage around $V_{DD}/2$ \cite{Homayoun}.
(3) The slew rate can also be approximated as $V_{DD}/(2 T_d)$ \cite{digitalbook}.

It follows from Eqs. (\ref{sphiwinverter}) and (\ref{sphi1/finverter}) that for $M$ noisy inverters in a delay line,
\beq
S_{\Phi, white, DL}=M\frac{\pi^2 T_d}{I_D^2 T_{in}}[S_{I}(f)|_{NMOS}+S_{I}(f)|_{PMOS}] + M \frac{4 kT \pi^2 T_d}{I_D V_{DD} T_{in}},
\label{sphiwDL}
\eeq
\beq
S_{\Phi, 1/f, DL}=M\frac{\pi^2 T_d^2}{I_D^2 T_{in}^2}[S_{1/f}(f)|_{NMOS}+S_{1/f}(f)|_{PMOS}],
\label{sphi1/fDL}
\eeq
where it is assumed $I_D$ is the same for NMOS and PMOS devices.
In the special case where the input period is equal to the period of the corresponding ring oscillator, we have $1/f_{osc}=T_{in}=2MT_d$, and (\ref{sphiwDL}) and (\ref{sphi1/fDL})
reduce to 
\beq
S_{\Phi, white, DL}=\frac{\pi^2 }{2 I_D^2}[S_{I}(f)|_{NMOS}+S_{I}(f)|_{PMOS}] + \frac{2kT \pi^2}{I_D V_{DD}},
\label{sphiwDLsc}
\eeq
\beq
S_{\Phi, 1/f, DL}=\frac{\pi^2}{4 M I_D^2}[S_{1/f}(f)|_{NMOS}+S_{1/f}(f)|_{PMOS}].
\label{sphi1/fDLsc}
\eeq


With the aid of Eq. (\ref{approxrel}), we can now express the phase noise of an $M$-stage ring
oscillator as:
\beq
S_{\Phi, white, ring}(\Delta f)=\frac{f_{osc}^2}{\Delta f^2} \left\{\frac{1}{2 I_D^2}[S_{I}(\Delta f)|_{NMOS}+S_{I}(\Delta f)|_{PMOS}] + \frac{2kT}{I_D V_{DD}} \right\},
\label{sphiwring}
\eeq
\beq
S_{\Phi, 1/f, ring}(\Delta f)=\frac{f_{osc}^2}{4 M I_D^2 \Delta f^2}\left[S_{1/f}(\Delta f)|_{NMOS}+S_{1/f}(\Delta f)|_{PMOS}\right].
\label{sphi1/fring}
\eeq
Note that these spectra are two-sided (i.e., $-\infty < f < + \infty$).
Accounting for the factor of 2 difference between one-sided and two-sided spectra, we observe that the phase noise given by Eq. (\ref{sphi1/fring}) is still twice that reported in \cite{Abidi}.
As verified by the simulations in Section 3.5, our result is correct. The factor of 2 error in \cite{Abidi} can be explained as follows. For a voltage-controlled
oscillator (VCO) sensing a small sinusoidal voltage of peak $V_m$ and frequency $f_m$, the relative magnitude of the sideband at the output is given by $K_{VCO} V_m/(2 f_m)$, where $K_{VCO}$ is the gain in Hz/V. 
It is tempting, but incorrect, to use this result directly for random noise, i.e., to write $K_{VCO}^2 S_n/(4 f_m^2)$ for the phase noise resulting from noise with spectral density $S_n$ \cite{Abidi}. Since phase noise is in
fact the spectrum of $\Phi_n$ in $\cos (\omega_c t +\Phi_n)$, we integrate noise with respect to time and multiply the result by $K_{VCO}$, obtaining $S_{\Phi_n}(f_m)=K_{VCO}^2 S_n/f_m^2$. If $S_n$ denotes a one-sided
spectrum, then this result must be divided by a factor of 2 so as to represent a two-sided $S_{\Phi_n}(f_m)$, producing $S_{\Phi_n}(f_m)=K_{VCO}^2 S_n/(2 f_m^2)$.



Equation (\ref{sphiwring}) reveals that $S_{\Phi, white, ring}$ is independent of the number of stages, as recognized in prior work \cite{Abidi,Hajimiri}. To confirm that $S_{\Phi,
white, ring}$ is fundamentally related to the power consumption (also recognized in \cite{Abidi,Hajimiri}),
suppose two rings incorporate identical inverters, but one contains
$M_1$ stages and the other $M_2$, where $M_1>M_2$. We add enough capacitance to each
node in the second ring so that the gate delays of the two rings, $T_{d1}$ and
$T_{d2}$, respectively, satisfy the relation $M_1 T_{d1}=M_2 T_{d2}$ and thus yield the same oscillation frequency. 
Since the gate delays are proportional to the load capacitances, it follows that $M_1 C_{L1}=M_2 C_{L2}$ and hence $f_{osc}(M_1 C_{L1})V_{DD}^2=f_{osc}(M_2
C_{L2})V_{DD}^2$. That is, equal oscillation frequencies guarantee equal power consumptions in this case.
Since the inverters are identical in the two designs, $I_D$ and $S_I(\Delta f)$ in Eq. (\ref{sphiwring}) are the same for the two oscillators, yielding the same
$S_{\Phi,white, ring}(\Delta f)$.

Equation (\ref{sphi1/fring}) shows that the phase noise due to flicker noise {\em falls} as the number of stages increases \cite{Abidi}. This is also
observed in the simulation results of Section 3.5.



\subsection{Effect of Transition Symmetry on Flicker Noise Upconversion}

The fundamental relation expressed by Eq. (\ref{approxrel}) implies that if flicker noise is upconverted in a delay line, so is it in a ring oscillator utilizing that delay line.
Thus, the upconversion phenomenon can be studied in a simpler delay line environment.

The flicker-noise-induced phase noise of delay lines is formulated by Eq. (\ref{sphi1/finverter}), with $\Delta T$ representing a quantity roughly equal to half of the transition time
caused by the NMOS or PMOS transistor in each stage. Interestingly, this equation suggests that the flicker noise is upconverted regardless of the relationship between $\Delta
T_{NMOS}$ and $\Delta T_{PMOS}$, a point in contradiction to the analysis in \cite{Hajimiri}, which predicts zero upconversion if the rise and fall transitions are symmetric. In fact,
as shown in Fig. \ref{fig:P2N}(a), phase noise simulations of a 9-stage 2.4-GHz ring oscillator reveal that the phase noise changes by only a few decibels as the PMOS-to-NMOS width ratio
varies from 1/4 to 4/1 and the risetime-to-falltime ratio from 3 to 0.76. This weak dependence is also verified by examining the upconversion of a 1-MHz tone placed in series with
the gate of one NMOS transistor in the ring. Figure \ref{fig:P2N}(b) reveals that the FM sideband magnitude varies little.
\begin{figure}[htb!]
\centering
\includegraphics[scale=0.45]{FIGS/CH3/P2NR2F.eps}
\includegraphics[scale=0.7]{FIGS/a.ps}
\includegraphics[scale=0.45]{FIGS/CH3/spurzoom.eps}
\includegraphics[scale=0.7]{FIGS/b.ps}
\includegraphics[scale=0.7]{FIGS/CH3/ISF.ps}
\caption{(a) Phase noise and risetime-to-falltime ratio versus the PMOS-to-NMOS width ratio of a 9-stage 2.4-GHz ring oscillator, (b) spur power when a small sinusoidal voltage source is put in series with the gate of
one NMOS transistor in the ring, and (c) ISF reported in \cite{Hajimiri} and uncorrelated ISF's for NMOS and PMOS devices.}
\label{fig:P2N}
\end{figure}

The flaw in \cite{Hajimiri} can be explained as follows. Since the flicker noise currents injected by the NMOS and PMOS devices in a ring are uncorrelated, each must be characterized
by its own impulse sensitivity function (ISF). Depicted in Fig. \ref{fig:P2N}(c), the NMOS and PMOS ISFs cannot have zero time average with any choice of rise and fall transitions,
thereby upconverting flicker noise unconditionally.



\subsection{Effect of Scaling on Phase Noise}

The white-noise-induced phase noise appears to be fundamentally related to the power dissipation and not much to the other factors. The effect of flicker noise, on the other hand, can
be articulated by rewriting Eq. (\ref{sphi1/fring}) as 
\ber
\!\!\!\!\!\!\!\!\!\!S_{\Phi, 1/f, ring}(\Delta f) &=& \nonumber\\
&&\!\!\!\!\!\!\!\!\!\!\!\!\!\!\!\!\!\!\!\!\!\!\!\!\!\!\!\!\!\!\!\!\!\!\!\!\!\!\!\!\!
\frac{f_{osc}^2}{8 \Delta f^3}\left[\left\{\frac{K}{MWLC_{ox}(V_{DD}-V_{TH})^2}\right\}_{NMOS}+\left\{\frac{K}{MWLC_{ox}(V_{DD}-V_{TH})^2}\right\}_{PMOS}\right],
\nonumber\\
\label{sphi1/fringscaling}
\eer
where it is assumed $g_m=I_D/(V_{GS}-V_{TH})$ for velocity-saturated devices and $K/(WLC_{ox}\Delta f)$ is assumed to be one-sided and is therefore divided by 2. It follows that the principal parameter under the designer's control for reducing the phase noise is the
total gate capacitance, $MWLC_{ox}$, of the ring oscillator.
For example, as simulations confirm, $S_{\Phi,1/f,ring}$ varies by less than 1 dB as $M$ goes from 3 to 16 while $MWL$ and $f_{osc}$ are constant.
Notwithstanding changes in $K$ with technology scaling, $S_{\Phi,1/f,ring}$ rises with a lower $V_{DD}-V_{TH}$ if the total gate capacitance is kept constant.




\section{Simulation Results}

In this section, three sets of simulation results are presented: one to verify the fundamental shaping function, $f_{osc}^2/(\pi \Delta f)^2$, 
another to show the dependence of the phase noise on the number of delay stages,
and the third to check the validity of our
compact phase noise equations, (\ref{sphiwring}) and (\ref{sphi1/fring}).

In order to verify the relation expressed by Eq. (\ref{exactrel}), we have simulated 9-stage and 19-stage delay lines and ring oscillators in 65-nm CMOS technology. Each
inverter incorporates a channel width of 0.6~$\mu$m and 1.2~$\mu$m for the NMOS and PMOS devices, respectively, and a channel length of 60~nm. The circuits operate
with a 1-V supply. In each case, the frequency of the input applied to the delay line is chosen equal to the corresponding ring oscillator frequency. 

\begin{figure}[htb!]
\centering
\includegraphics[scale=0.6]{FIGS/CH3/sim9stage.eps}
\includegraphics[scale=1]{FIGS/a.ps}
\includegraphics[scale=0.6]{FIGS/CH3/sim19stage.eps}
\includegraphics[scale=1]{FIGS/b.ps}
\caption{Simulated phase noise of delay lines and ring oscillators as well as calculated phase noise of the ring oscillator using the phase noise of the delay line for (a)
9-stage, and (b) 19-stage configurations.}
\label{fig:sim}
\end{figure}
Figure \ref{fig:sim}(a) plots the simulated phase noise for the 9-stage delay line and the corresponding ring oscillator.
The latter's phase noise is obtained using
(\ref{exactrel}) as well as direct simulations. We note good agreement in both flicker noise and white noise regimes. The oscillation frequency is 3.8~GHz and the power
consumption 0.34~mW. Figure \ref{fig:sim}(b) repeats the results for a 19-stage arrangement operating at a frequency of 1.7~GHz and drawing 0.32~mW. The results agree well
in this case, too.

\begin{figure}[htb!]
\centering
\includegraphics[scale=0.6]{FIGS/CH3/numberofstages2.eps}
\caption{Simulated effect of number of delay cells on the phase noise of ring oscillators.}
\label{fig:numberofstages}
\end{figure}
Figure \ref{fig:numberofstages} plots the simulated phase noise of three ring oscillators operating at 9.54~GHz. Explicit capacitors are added to all nodes of 3-stage and
5-stage rings.
Since the power consumption varies slightly, from 1.39~mW to 1.47~mW, as the rings become longer, the phase noise plots
are normalized to the corresponding values. We observe that the white-noise-induced phase noise remains unchanged as the number of stages increases, but, as predicted by 
Eq. (\ref{sphi1/fring}), the flicker-noise-induced component decreases in proportion to $M$. 


\begin{figure}[htb!]
\centering
\includegraphics[scale=0.6]{FIGS/CH3/compacteq.eps}
%\caption{Simulated phase noise of a 9-stage ring oscillator and calculated phase noise using compact equations (\ref{sphiwring}) and (\ref{sphi1/fring}).}
%\caption{Simulated phase noise of a 9-stage ring oscillator and calculated phase noise using compact equations (\fragile{\ref{sphiwring}}) and (\fragile{\ref{sphi1/fring}}).}
\caption{Simulated phase noise of a 9-stage ring oscillator and calculated phase noise using compact equations (3.16) and (3.17).}
\label{fig:compacteq}
\end{figure}
Figure \ref{fig:compacteq} plots the simulated phase noise of the 9-stage ring oscillator as well as the calculated phase noise using
Equations (\ref{sphiwring}) and (\ref{sphi1/fring}). 
(The flicker and white current noise spectra, $S_{1/f}$ and $S_I$, respectively, are obtained from simulations in Cadence).\footnote{The value of $I_D$ is
obtained from transient simulations at the point when $V_{DS}\approx V_{DD}/2$. The $V_{GS}$ and $V_{DS}$ values corresponding to this case are then used in a simple noise
simulation of a single transistor.}




\section{Experimental Results}

The delay lines and ring oscillators described in Section 3.5 have been fabricated in 65-nm CMOS technology and characterized. Figure \ref{fig:die} shows a die photograph
of the prototypes. Each circuit is followed by an on-chip open-drain buffer for driving 50-$\Omega$ instrumentation.
\begin{figure}[htb!]
\centering
\includegraphics[scale=1]{FIGS/CH3/die.eps}
\caption{Die photograph.}
\label{fig:die}
\end{figure}

The low phase noise of delay lines poses difficulties in measurement. For this reason, the delay line prototype in fact incorporates 745 stages rather than 9 or 19,
producing a readily measurable phase noise (Fig. \ref{fig:dl}). This value is then scaled down by a factor equal to 745/9 or 745/19 to obtain the phase noise of the
respective delay lines.
\begin{figure}[htb!]
\centering
\includegraphics[scale=0.6]{FIGS/CH3/DL.eps}
\caption{Measured phase noise of 745-stage delay line at two different input frequencies.}
\label{fig:dl}
\end{figure}


The phase noise of ring oscillators also proves difficult to measure if low offset frequencies are of interest. The random fluctuations of the free-running center frequency
tend to smear the phase noise profile. It is therefore beneficial to phase-lock the oscillator to a low-noise input with a sufficiently small loop bandwidth so as
to negligibly affect the phase noise in the offset frequency range of interest. Figure \ref{fig:setup} shows the test setup constructed around each ring oscillator to create
a type-I phase-locked loop (PLL).
\begin{figure}[htb!]
\centering
\includegraphics[scale=1]{FIGS/CH3/setup.ps}
\caption{Phase-locking of the ring oscillators for phase noise measurements.}
\special{psfile=FIGS/CH3/setup hoffset=-50 voffset=15 vscale=120 hscale=120}
\label{fig:setup}
\end{figure}
Here, an off-the-shelf mixer serves as a phase detector, comparing the phases of an external RF signal and the ring oscillator output. The latter's supply line acts
as the control voltage. 
The loop bandwidth is set by the choice of the components in the low-pass filter.

\begin{figure}[htb!]
\centering
\includegraphics[scale=0.6]{FIGS/CH3/meas9stage.eps}
\includegraphics[scale=1]{FIGS/a.ps}
\includegraphics[scale=0.6]{FIGS/CH3/meas19stage.eps}
\includegraphics[scale=1]{FIGS/b.ps}
\caption{Measured phase noise of ring oscillators and the calculated phase noise using the measured phase noise of delay line for (a) 9-stage, and (b) 19-stage rings.}
\label{fig:meas}
\end{figure}

Figure \ref{fig:meas}(a) plots the phase noise of the 9-stage ring oscillator obtained by (a) direct measurement, and (b) by multiplying the measured delay line phase noise
by $f_{osc}^2/(\pi \Delta f)^2$. We observe a reasonable agreement. Figure \ref{fig:meas}(b) repeats the results for the 19-stage configuration. In both cases, the effect of
the PLL manifests itself at low offset frequencies.



\section{Conclusion}

It is shown that the closed-loop phase noise of a ring oscillator is equal to its open-loop phase noise multiplied by a simple shaping function, $f_{osc}^2/(\pi \Delta f)^2$. This relation
reveals why delay lines exhibit much less noise than do ring oscillators. It also leads to compact phase noise equations and shows why flicker noise is
upconverted even with symmetric rise and fall times. The flicker-noise-induced phase noise is not a strong function of the PMOS-to-NMOS ratio and the minimum phase noise
does not necessarily happen when the rise and fall times are symmetric.
The validity of the shaping function has been verified on two ring oscillators designed in 65-nm CMOS technology.

                         % etc.
\chapter{A 5-GHz 11.6-mW CMOS Receiver for IEEE 802.11a Applications}                        




\section{Introduction}

In today's mobile devices, the WiFi transceiver still consumes a relatively large amount of power. 
The receiver power is substantial due to the greater on-time of the receiver compared to the transmitter.
Figure \ref{fig:genericRX} shows a generic receiver system for 802.11a. 
\begin{figure}[htb!]
\centering
\includegraphics[scale=1]{FIGS/CH4/genericRX.ps}
\caption{Generic 802.11a Receiver.}
\label{fig:genericRX}
\end{figure}
While advances in the art have considerably reduced the power consumption of analog-to-digital converters and frequency synthesizers, the main
receiver (RX) chain draws disproportionately high power. For example, we can now realize each ADC in Fig. \ref{fig:genericRX} with about 3 mW of power and
the frequency synthesizer with about 5 mW. On the other hand, the receive path itself draws more than 45 mW \cite{Kan}. It is therefore desirable to
develop low-power RX front ends and baseband filters for WiFi applications. 

This chapter introduces a complete 5-GHz CMOS receiver that meets the 11a sensitivity, blocking, and filtering requirements while consuming 11.6 mW.
This fourfold reduction in power is achieved through the use of a transformer as a low-noise amplifier (LNA), passive mixers, and ``non-invasive''
baseband filtering \cite{Zolfaghari}.

Section 4.2 elaborates on the design of the transformer and Section 4.3 introduces the receiver architecture.
Section 4.4 analyzes the current-driven passive mixers in a general form which is used in Section 4.5 to design the mixers connected to the transformer. 
Section 4.6 describes the baseband channel-select filters and Section 4.7 presents the experimental results. Section 4.8 concludes this chapter.


 
\section{Transformers as LNAs}
In the generic receiver of Fig. \ref{fig:genericRX}, the low-noise amplifier (LNA) provides voltage gain and proper input matching, but dissipates
considerable amount of power. In order to realize a low-power receiver, we wish to remove the LNA or at least its power consumption. This becomes
possible if a passive device with zero power consumption can serve as an LNA. A 1-to-N on-chip transformer provides voltage gain at the cost of power
loss and noise figure degradation. It can also provide input matching as will be explained later. Thus, it seems that a transformer is a viable
substitute for the LNA. However, its design is not straightforward. The principal point is to achieve a high voltage gain with low power loss at the
frequency of interest. In particular, we would like to obtain a high coupling factor between the primary and the secondary. 
\begin{figure}[htb!]
\centering
\includegraphics[scale=1]{FIGS/CH4/Transformer_topologies.ps}
\caption{Various transformer topologies.}
\label{fig:Transformer_topologies}
\end{figure}
Consider the stacked structure \cite{Zolfaghari2} shown in Fig. \ref{fig:Transformer_topologies}(a) where spirals in high and low metal layers form the
secondary and one spiral in middle metal layer forms the primary. This structure has a high coupling factor but, the high capacitance between the
layers and the loss of the lower metal layers limit the performance. Note that the capacitance here sustains a larger voltage difference than in a
planar structure \cite{Long} shown in Fig. \ref{fig:Transformer_topologies}(b). The self-resonance frequency and conductivity of the planar topology is
much higher, but the coupling between the primary and the inner turns of the secondary is quite low. The optimum topology is shown in Fig.
\ref{fig:Transformer_topologies}(c) where a one-turn primary is stacked in the middle of the secondary. It is expected to have reasonable coupling, low
loss, and small capacitance to be able to work at 5-GHz band. To reduce the loss, an octagonal shape has been used as shown in Fig.
\ref{fig:transformer}, with a one-turn primary in metal 8 and a six-turn secondary in metal 9.
\begin{figure}[htb!]
\centering
\includegraphics[scale=0.6]{FIGS/CH4/Transformer2D.eps}
%\includegraphics[scale=1]{FIGS/CH4/Transformertext.ps}
\caption{Transformer geometry.}
\label{fig:transformer}
\end{figure}
The outer diameters of the primary and the secondary are 146 $\mu$m and 170 $\mu$m, respectively. The two thick metal layers have been used to
minimize the loss. Since the secondary has already much more parasitics than the primary, it uses the top metal to minimize the capacitance to the
substrate. The number of turns, diameter, and metal widths are chosen to provide maximum voltage gain with acceptable loss at the desired
frequency. If we increase the number of secondary turns, the voltage gain grows slowly because the new turns are farther from the primary, but the
capacitance and loss increase significantly. Also, a higher number of turns reduces the transformer input resistance.  According to HFSS
simulations, the above transformer has an insertion loss of 2.4 dB at 5.5 GHz, an unloaded voltage gain of 13.4 dB, and an unloaded input resistance
of 87 $\Omega$. With a matched load connected to the secondary, the voltage gain and input resistance drop to 12 dB and 44 $\Omega$, respectively. We then
need to understand how the transformer performs when it is connected to the following block, namely mixers. 



\section{Receiver Architecture}
Figure \ref{fig:architecture} shows the overall receiver architecture. 
\begin{figure}[htb!]
\centering
\includegraphics[scale=1]{FIGS/CH4/Architecture_corrected.ps}
\caption{Receiver architecture.}
\label{fig:architecture}
\end{figure}
The transformer described in the previous section serves as the LNA and single-ended to differential converter at the input and drives two sets of
passive mixers with 25\% duty cycle LOs. The downconverted baseband signals are then applied to 4-th order elliptic filters for channel selection. A
side-benefit is that the transformer also provides ESD protection. The LO frequency arrives at twice the carrier frequency and is divided by two to
generate quadrature phases. We then have some logic to produce the 25\% LO waveforms.

We should highlight two advantages of our approach over the LNA-less receiver in \cite{Nauta}. First, the input matching inherent in our receiver
provides a robust interface with the antenna in the presence of long external traces. Second, our front end has much less power consumption.

There are three important questions that we need to address. First, how do the transformer and mixers provide input matching? Second, What is the
conversion gain of the mixers? It seems that the current is integrated in the load capacitor, $C_1$. In that case, the trans-impedance conversion gain
will not be flat across the channel bandwidth. Finally, how should we calculate the noise figure?
To answer these three important questions, a new analysis technique is required. We will study the input impedance, conversion gain and noise figure
and see why the trans-impedance conversion gain is flat across the channel.

By virtue of its high turns ratio, the transformer in Fig. \ref{fig:architecture} exhibits a relatively high output impedance, approximating a current
source. The switches can therefore be viewed as current-driven mixers, thus contributing less noise than voltage-driven topologies \cite{Kaczman}.
Moreover, for the sake of analysis, since a capacitor is a good {\em keeper} of voltage, it is difficult to assume voltage excitations.




\section{Analysis of Current-Driven Passive Mixers} 

The 25\% duty cycle passive mixers have been used extensively since 2008 \cite{Blaakmeer}. But, their input impedance, conversion gain, and noise
figure have not been formulated accurately \cite{Mirzaei}-\cite{Andrews}. This section proposes an accurate analysis that helps the designer to insightfully
quantify the underlying principles. The method can be extended to other LO schemes as well.
\begin{figure}[htb!]
\centering
\includegraphics[scale=1]{FIGS/CH4/IQMixer.ps}
\caption{Current-driven passive mixer.}
\label{fig:IQMixer}
\end{figure}
Figure \ref{fig:IQMixer} shows the current-driven quadrature passive mixers with a source impedance $Z_s$. The circuit has
been redrawn in Fig. \ref{fig:IQMixer2} for convenience where the LO waveforms controlling the switches, $S_1$--$S_4$, are also shown.
\begin{figure}[htb!]
\centering
\includegraphics[scale=1]{FIGS/CH4/IQMixer2.ps}
\caption{Current-driven passive mixer and LO waveforms.}
\label{fig:IQMixer2}
\end{figure}
For simplicity, let us assume that the switch resistance, $R_{sw}$, is zero. We will add its effect later. The in-phase output voltage can be
written as  
\beq
V_{out,I}(t)=\left[I_{mix} \times (S_1-S_3)\right] \ast 2 Z_L. 
\label{vouti} 
\eeq 
The mixer input current, $I_{mix}$, is chopped by $S_1$ and $S_3$ and convolved with the differential load impedance, $2 Z_L$. 
Similarly, the quadrature output voltage would be
\beq 
V_{out,Q}(t)=\left[I_{mix} \times (S_2-S_4)\right] \ast 2 Z_L. 
\label{voutq} 
\eeq 
The input voltage is equal to $V_{out,I}$ during $S_1$, -$V_{out,I}$ during $S_3$, $V_{out,Q}$ during $S_2$, and -$V_{out,Q}$ during $S_4$.
Thus,
\beq
V_{in}(t)= V_{out,I}(t) \times (S_1-S_3) + V_{out,Q}(t) \times (S_2-S_4). 
\label{vint1} 
\eeq 
Substituting (\ref{vouti}) and (\ref{voutq}) in (\ref{vint1}), yields
\ber
V_{in}(t) &=& \left\{[I_{mix} \times (S_1-S_3)] \ast 2 Z_L \right\} \times (S_1-S_3) \nonumber\\
&+& \left\{[I_{mix} \times (S_2-S_4)] \ast 2 Z_L \right\} \times (S_2-S_4).  
\label{vint2} 
\eer 
Taking the Fourier transform of (\ref{vint2}) leads to
\ber
V_{in}(f) &=& \left\{[I_{mix} \ast (S_1-S_3)] \times 2 Z_L \right\} \ast (S_1-S_3) \nonumber\\ 
&+& \left\{[I_{mix} \ast (S_2-S_4)] \times 2 Z_L \right\} \ast (S_2-S_4),  
\label{vinf1} 
\eer
which is the key equation to derive the input impedance. 
Figure \ref{fig:S1S4} plots $(S_1-S_3)$ and $(S_2-S_4)$ in both time and frequency domains.
\begin{figure}[htb!]
\centering
\includegraphics[scale=1]{FIGS/CH4/S1S4.ps}
\caption{$(S_1-S_3)$ and $(S_2-S_4)$ in time and frequency domains.}
\label{fig:S1S4}
\end{figure}
Let us do a warm-up exercise and consider the case where the source impedance is infinite in Fig. \ref{fig:IQMixer2}.

\subsection{Infinite Source Impedance}
Infinite source impedance, $Z_s$, yields $I_{mix}=I_{in}$. This simplifies the calculations as $I_{mix}$ would be single-tone.
\subsubsection{Input Impedance Calculation} 
To calculate the input impedance, we apply a tone at $f_{LO}+f_{IF}$ as shown in Fig. \ref{fig:spectrumzmix} and would like to calculate the input voltage
at the same frequency. 
\begin{figure}[htb!]
\centering
\includegraphics[scale=1]{FIGS/CH4/spectrumzmix.ps}
\caption{A single-tone current applied to the mixer and the resultant input voltage spectrum.}
\label{fig:spectrumzmix}
\end{figure}
Let us first focus on the positive-frequency impulse of
$I_{mix}$ and see how it will be shifted along the spectrum to finally reside at the same frequency. The only way is to shift up and down by the
same amount. For example, from the spectrum of $(S_1-S_3)$, $I_{mix}$ will be shifted up by $\sqrt 2 \delta (f-f_{LO})/\pi$ and down by  $\sqrt 2
\delta (f+f_{LO})/\pi$. Another way is to be shifted up by $\sqrt 2 \delta (f-f_{LO})/(3\pi)$ and down by $\sqrt 2 \delta (f+f_{LO})/(3\pi)$.
Also, the mixing mechanism can happen with first shifting down and then up. The same power would come from mixing with $(S_2-S_4)$. Note that
because the spectrum of $(S_2-S_4)$ is odd, one component would be multiplied by $+j$ and the other by $-j$, resulting a real positive value, the
same as that of mixing with $(S_1-S_3)$.
An interesting practice is to see if there would be any component at the image frequency, $f_{LO}-f_{IF}$ in the $V_{in}$ spectrum. The sum of the
two shifting frequencies has to be $\pm 2 f_{LO}$ in order to generate image. For example, the impulse at $f_{LO}+f_{IF}$ would be shifted down twice by
$\sqrt 2 \delta (f+f_{LO})/\pi$ to reside at $-f_{LO}+f_{IF}$. However, shifting down twice by  $+j \sqrt 2 \delta (f+f_{LO})/\pi$ generates the
same amplitude with a negative sign. Similarly, shifting down by $3 f_{LO}$ and up by $f_{LO}$ will not generate any image because the I and Q
branches cancel each other. It is instructive to see at what frequencies $V_{in}$ has components in response to a single tone that resides at
$f_{LO}+f_{IF}$ and $-f_{LO}-f_{IF}$. An interested reader can prove that those frequencies are $(f_{LO}+f_{IF}+4kf_{LO})$ and
$(-f_{LO}-f_{IF}+4kf_{LO})$, where $k$ varies from $-\infty$ to $+\infty$ (Fig. \ref{fig:spectrumzmix}).

Knowing the frequency shifts that happen in (\ref{vinf1}) that contribute to the input voltage at $f_{LO}+f_{IF}$, we can write the mixer input
impedance, $Z_{mix}$, as
\beq
Z_{mix}(f)=\frac{8}{\pi^2}\left[Z_L(f \pm f_{LO})+\frac{1}{3^2}Z_L(f \pm 3f_{LO})+\frac{1}{5^2} Z_L(f \pm 5f_{LO}) + ... \right].
\label{zmix}
\eeq
If the load impedance is a capacitor, $1/(j 2 \pi C_L f)$, at the frequencies close to the LO frequency, $Z_L(f - f_{LO})$ dominates the summation in
(\ref{zmix}) and the input impedance would be approximately equal to
\beq
Z_{mix}|_{C_L}(f) \approx \frac{8}{\pi^2} Z_L(f - f_{LO}) = \frac{8}{\pi^2} \frac{1}{j 2 \pi C_L(f-f_{LO})}.
\label{zmixc}
\eeq
As another special case, if the load impedance is a resistor, $R_L$, the input impedance would be
\beq
Z_{mix}|_{R_L}(f)=\frac{16 R_L}{\pi^2}\left(1+\frac{1}{3^2}+\frac{1}{5^2}+...\right) = 2 R_L .
\label{zmixr}
\eeq
Since there is no memory in the system, the source would not recognize that the resistor is being switched and at any point of time the differential
input impedance is $2 R_L$.



\subsubsection{Conversion Gain Calculation} 
We are mostly interested in the trans-impedance conversion gain defined as the transfer function from the input current at $f_{LO}+f_{IF}$ to the
output voltage at $f_{IF}$. Since $I_{mix}$ is single-tone, only the fundamental of $(S_1-S_3)$ matters. Thus, from Eq. (\ref{vouti}) the
trans-impedance conversion gain, $A_R$, would be
\beq
A_R=\frac {2 \sqrt 2}{\pi} Z_L ~ \approx 0.9 Z_L.
\label{armix}
\eeq
Note that if $Z_L$ is a capacitor, $A_R$ is {\em not} constant across the channel bandwidth. 
Now, let us look at the voltage conversion gain, $A_V$, defined as the transfer function from the input voltage at $f_{LO}+f_{IF}$ to the output
voltage at $f_{IF}$. We have
\beq
A_V=\frac{V_{out,I}}{V_{in}}=\frac{V_{out,I}}{I_{in}} \times \frac{I_{in}}{V_{in}}=\frac{A_R}{Z_{mix}}.
\label{avar}
\eeq
Substituting (\ref{zmix}) and (\ref{armix}) in (\ref{avar}), we get
\beq
A_V=\frac{\frac {2 \sqrt 2}{\pi} Z_L}{\frac{8}{\pi^2}\left[Z_L(f \pm f_{LO})+\frac{1}{3^2}Z_L(f \pm 3f_{LO})+\frac{1}{5^2} Z_L(f \pm 5f_{LO}) + ... \right]}.
\label{avmix}
\eeq
We can simplify $A_V$ for the special cases of load capacitor and load resistor as
\beq
A_V|_{C_L} \approx \frac{\pi \sqrt 2}{4} \approx 0.9 \approx  0.9~{dB}, 
\label{avmixc}
\eeq
\beq
A_V|_{R_L} = \frac{\sqrt 2}{\pi} \approx 0.45 \approx -6.9~{dB}.
\label{avmixr}
\eeq
As expected, having a load capacitance is superior than a load resistance. Interestingly, $A_V|_{C_L}$ is greater than 1. It is not surprising,
because $V_{out}$ is the cause and $V_{in}$ is the effect here. The power of $V_{out}$ is only at $f_{IF}$. This power is then spread over the
specific harmonics at $V_{in}$. The same result can be achieved using Eq. (\ref{vint1}), where $V_{out}$ at $f_{IF}$ will be convolved with the
fundamental of $(S_1-S_3)$ and $(S_2-S_4)$ to form $V_{in}$ at $f_{LO}+f_{IF}$.


\subsubsection{Switch Resistance Effect}
\begin{figure}[htb!]
\centering
\includegraphics[scale=1]{FIGS/CH4/Rsw.ps}
\caption{Moving the switch resistance to the main path.}
\label{fig:rsw}
\end{figure}
Because only one switch out of the four switches that share one side is on at a time, we can move the switch resistance to the main path as shown in
Fig. \ref{fig:rsw}. 
Then, the switch resistance is in series with the input current source ($Z_s$ is infinite), and the input impedance would be
\beq
Z_{mix}(f)=2 R_{sw}+\frac{8}{\pi^2}\left[Z_L(f \pm f_{LO})+\frac{1}{3^2}Z_L(f \pm 3f_{LO})+\frac{1}{5^2} Z_L(f \pm 5f_{LO}) + ... \right] .
\label{zmixsw}
\eeq
The trans-impedance conversion gain does not change, but, the voltage conversion gain needs to be modified according to (\ref{zmixsw}) as
\beq
A_V=\frac{\frac {2 \sqrt 2}{\pi} Z_L}{2 R_{sw}+\frac{8}{\pi^2}\left[Z_L(f \pm f_{LO})+\frac{1}{3^2}Z_L(f \pm 3f_{LO})+\frac{1}{5^2} Z_L(f \pm 5f_{LO}) + ...
\right]}.
\label{avmixsw}
\eeq
Interested readers can simplify (\ref{zmixsw}) and (\ref{avmixsw}) for the special cases that we studied before.




\subsection{Finite Source Impedance}
With finite source impedance, $Z_s$, $I_{mix}$ is no longer equal to $I_{in}$. Instead,
\beq
I_{mix}=I_{in} - \frac{V_{in}}{Z_s}.
\label{imix}
\eeq
Although this might seem a small change to the case with infinite $Z_s$, the equations become much more complicated. Recall from Section 4.4.1 that
$V_{in}$ had components around odd harmonics. It can be shown that in the presence of $Z_s$, $V_{in}$ has the same frequency content but the
amplitudes needs to be modified. Note that (\ref{imix}) yields that $I_{mix}$ also has components around odd harmonics (Fig. \ref{fig:spectrumzin}).
\begin{figure}[htb!]
\centering
\includegraphics[scale=1]{FIGS/CH4/spectrumzin.ps}
\caption{A single-tone current applied at the input and the resultant spectrum of the mixer input current and voltage.}
\label{fig:spectrumzin}
\end{figure}

\subsubsection{Input Impedance Calculation} 
We can still use Eq. (\ref{vinf1}) to derive the input impedance. This time, the harmonics of $I_{mix}$ convolve with the harmonics of
$(S_1-S_3)$ and $(S_2-S_4)$, and fold to the baseband frequency. There are also other components around harmonics which we ignore after multiplication
by $Z_L$. Since the mixer is a downconverter, the load impedance, $Z_L$, would be small at high frequencies. In order to calculate the input
impedance, as shown in Fig. \ref{fig:spectrumzin}, we assign $a_i$ and $b_i$ as the phasors of $I_{mix}$ and $V_{in}$ around $i$th harmonic and find their
values using (\ref{vinf1}) and (\ref{imix}). Since the amplitude of the tone in $I_{in}$ is unity, the input impedance is equal to $b_1$. Section 4.8
finds the values of $a_i$ and $b_i$, yielding
\beq
Z_{in}(f)=\frac{\frac{8}{\pi^2} Z_L(f-f_{LO})}{1+\frac{8}{\pi^2} Z_L(f-f_{LO})\displaystyle{\SumAll k
\frac{1}{(4k+1)^2 Z_s(f+4k f_{LO})}}}.
\label{zin}
\eeq
If $Z_L$ is a capacitor, we can usually neglect $1$ in the denominator of (\ref{zin}) for reasonable values of $Z_s$. Thus,
\beq
\frac {1}{Z_{in}|_{C_L}(f)} \approx \SumAll k \frac{1}{(4k+1)^2 Z_s(f+4k f_{LO})}.
\label{zinc}
\eeq
The input impedance is the parallel combination of the scaled source impedance at certain frequencies and independent of the load impedance, $Z_L$.
Moreover, if $Z_s$ is a resistor, $R_s$, then 
\beq
Z_{in}|_{C_L,R_s}=\frac{8}{\pi^2}R_s \approx 0.81 R_s
\label{zincr}
\eeq


\subsubsection{Conversion Gain Calculation}
As mentioned earlier, we assume that $Z_L$ has a lowpass shape and ignore the harmonics at the output nodes.
The voltage conversion gain is the same as the case with infinite $Z_s$, while we do not include the switch resistance. Thus, for a capacitive load,
we have
\beq
A_V|_{C_L}=\frac{\pi \sqrt 2}{4} \approx 1.11 \approx 0.9~{dB}.
\label{avinc}
\eeq
The trans-impedance conversion gain would be $A_V \times Z_{in}$ and equal to
\beq
A_R|_{C_L} = \frac{\pi \sqrt 2}{4} \div \SumAll k \frac{1}{(4k+1)^2 Z_s(f+4k f_{LO})},
\label{arinc}
\eeq
independent of $C_L$ and $f_{IF}$.


\subsubsection{Switch Resistance Effect}
If we add the switch resistance, the input voltage is equal to the one in Eq. (\ref{vinf1}) plus $2 R_{sw}I_{mix}$. Thus, 
\ber
V_{in}(f) &=& \left\{[I_{mix} \ast (S_1-S_3)] \times 2 Z_L \right\} \ast (S_1-S_3) \nonumber\\ 
&+& \left\{[I_{mix} \ast (S_2-S_4)] \times 2 Z_L \right\} \ast (S_2-S_4) + 2R_{sw}I_{mix}.  
\label{vinf1sw} 
\eer
Equation (\ref{imix}) is still valid and along with (\ref{vinf1sw}) derives the input impedance. Note that we could assign $a_{sw,i}$ and $b_{sw,i}$
as the phasors of $I_{mix}$ and $V_{in}$ around $i$th harmonic and find their values similar to section 4.8. However, finding the solution is more
difficult in this case. We wish to perform some transforms so that we can utilize the equations that we already have. This is done by the
Norton-Thevenin-Norton conversion shown in Fig. \ref{fig:ntn}. 
\begin{figure}[htb!]
\centering
\includegraphics[scale=0.7]{FIGS/CH4/ntn.ps}
\caption{Norton-Thevenin-Norton Conversion.}
\label{fig:ntn}
\end{figure}
The circuit in Fig. \ref{fig:ntn}(c) is similar to the case without switch resistance, and we
can easily find $V_{mix}$ as the input current times the input impedance seen by the source, i. e., 
\beq
V_{mix}(f)=I_{in}\frac{Z_s}{Z_s+2R_{sw}} \times \frac{\frac{8}{\pi^2} Z_L(f-f_{LO})}{1+\frac{8}{\pi^2} Z_L(f-f_{LO})\displaystyle{\SumAll k
\frac{1}{(4k+1)^2 [Z_s(f+4k f_{LO})+2R_{sw}]}}}.
\label{vmix}
\eeq
Then from the original circuit in Fig. \ref{fig:ntn}(a), $V_{in}=V_{mix}+2R_{sw}I_{mix}$. Replacing $I_{mix}$ with $I_{in}-V_{in}/Z_s$, and using
(\ref{vmix}), we find the input impedance as 
\ber
&&\!\!\!\!\!\!\!\!\!\!\!\!\!\!\!\!\!\!\!\!\!\!Z_{in}(f)=Z_s||2R_{sw}+ \nonumber\\
&& (\frac{Z_s}{Z_s+2R_{sw}})^2 \times \frac{\frac{8}{\pi^2} Z_L(f-f_{LO})}{1+\frac{8}{\pi^2} Z_L(f-f_{LO})\displaystyle{\SumAll k
\frac{1}{(4k+1)^2 [Z_s(f+4k f_{LO})+2R_{sw}]}}}. \nonumber\\
\label{zinsw}
\eer
For a load capacitance and reasonable source impedance, we can usually ignore $1$ in the denominator of (\ref{zinsw}) and write
\beq
Z_{in}|_{C_L}(f)=Z_s||2R_{sw}+ (\frac{Z_s}{Z_s+2R_{sw}})^2 \div \displaystyle{\SumAll k \frac{1}{(4k+1)^2 [Z_s(f+4k f_{LO})+2R_{sw}]}}.
\label{zinswc}
\eeq
Again, we see that the input impedance is independent of the load impedance under certain conditions.
Now, we wish to find the voltage conversion gain, $A_V$. 
We can write
\beq
V_{in}(f)=\frac {2 \sqrt 2}{\pi} V_{out,I}(f-f_{LO}) + 2R_{sw}I_{mix}(f),
\label{vinfsw}
\eeq
where only low-frequency component of $V_{out,I}$ has been taken into account. Using Eq. (\ref{imix}) and the fact that $I_{in}=V_{in}/Z_{in}$, we get
\beq
I_{mix}=\frac {Z_s - Z_{in}}{Z_s Z_{in}} V_{in}
\label{imix2}
\eeq
Solving (\ref{vinfsw}) and (\ref{imix2}) yields
\beq
A_V|_{C_L}=\frac{\pi \sqrt 2}{4}\left(1-2R_{sw}\frac {Z_s - Z_{in}}{Z_s Z_{in}}\right)
\label{avincsw}
\eeq
The trans-impedance conversion gain, $A_R$, is equal to $A_V \times Z_{in}$, and can be found using (\ref{zinswc}) and (\ref{avincsw}).
Interestingly, $A_R$ is independent of $C_L$ and $f_{IF}$. The latter is very important because otherwise, we would not have flat response across the
channel.


\subsection{Noise Figure Calculation}
With infinite $Z_s$, the switch resistance and hence its noise is in series with the input current source. Thus, it does not contribute noise. 
Now, let us calculate the noise figure with finite purely resistive $Z_s$ equal to $R_s$. We include the switch resistance, but assume it noiseless
first. We add the effect of switch resistance noise later. We also assume that the load impedance is a capacitor and ignore the higher harmonics at the
output. If $I_{in}$ is the rms value of the input signal, the signal-to-noise ratio (SNR) at the input would
be
\beq
SNR_{in}=\frac{I_{in}^2}{\displaystyle{{4kT\frac{1}{R_s}}}}.
\label{snrin}
\eeq
The signal and noise around $f_{LO}$ will be downconverted at the output nodes with the same trans-impedance conversion gain, $A_R$. However, the noise
around $i$th harmonics of $f_{LO}$ will also fold on the signal with a conversion gain of $A_R/i$. Thus, the output SNR would be
\beq
SNR_{out}=\frac{I_{in}^2 A_R^2}{\displaystyle{{4kT\frac{1}{R_s}}}A_R^2 \left(1+\frac{1}{9}+\frac{1}{25}+...\right)},
\label{snrout}
\eeq
where we have assumed the signal is double-sideband (DSB).
By definition, the DSB noise figure is
\beq
NF=\frac{SNR_{in}}{SNR_{out}}=(1+\frac{1}{9}+\frac{1}{25}+...)=\frac{\pi^2}{8} \approx 0.9~{dB}.
\label{fig:nf}
\eeq 
Now, let us add the switch resistance noise. For noise analysis purpose, we remove the input signal source and notice that the switch resistance is in
series with $R_s$. Therefore, the noise figure will be degraded by $(1+2R_{sw}/R_s)$, i.e., 
\beq
NF=\frac{\pi^2}{8} (1+\frac{2R_{sw}}{R_s}).
\label{nftot}
\eeq 
Now, let us consider a general source impedance. We still assume that at the vicinity of $f_{LO}$, $Z_s=R_s$ and the noise current source is $4kT/R_s$.
We use the converted model in Fig. \ref{fig:ntn}(c). 
The input signal power is $I_{in}^2Z_s^2/(Z_s+2R_{sw})^2$, and the input noise current would be
\beq
\overline{i_{n}^2}=\frac{4kT Z_s^2}{R_s(Z_s+2R_{sw})^2}+\frac{4kT 2R_{sw}}{(Z_s+2R_{sw})^2}.
\eeq
Then the output signal is simply
\beq
V_{out}^2=\frac{Z_s(f_{LO})^2}{[Z_s(f_{LO})+2R_{sw}]^2}I_{in}^2 A_R^2,
\eeq
but for the noise we have to consider the noise around harmonics as well. It folllows that
\ber
\overline{V_{n,out}^2}&=&\left\{\frac{4kT Z_s(f_{LO})^2}{R_s[Z_s(f_{LO})+2R_{sw}]^2}+\frac{4kT 2R_{sw}}{[Z_s(f_{LO})+2R_{sw}]^2}\right\}A_R^2 \nonumber\\
&+&\left\{\frac{4kT Z_s(3f_{LO})^2}{R_s[Z_s(3f_{LO})+2R_{sw}]^2}+\frac{4kT 2R_{sw}}{[Z_s(3f_{LO})+2R_{sw}]^2}\right\}\frac{A_R^2}{9} \nonumber\\
&+&\left\{\frac{4kT Z_s(5f_{LO})^2}{R_s[Z_s(5f_{LO})+2R_{sw}]^2}+\frac{4kT 2R_{sw}}{[Z_s(5f_{LO})+2R_{sw}]^2}\right\}\frac{A_R^2}{25} + ...~.
\eer
Thus, we have
\beq
\frac{1}{SNR_{out}}=\frac{\displaystyle{{4kT\frac{1}{R_s}}}}{I_{in}^2} \frac{[Z_s(f_{LO})+2R_{sw}]^2}{Z_s(f_{LO})^2}
\sum_{k=0}^{+\infty} \frac{1}{(2k+1)^2} \left[ \frac {Z_s[(2k+1)f_{LO}]^2 + 2R_{sw}R_s} {\{Z_s[(2k+1)f_{LO}]+2R_{sw}\}^2} \right].
\label{1snrout}
\eeq
Using (\ref{snrin}) and (\ref{1snrout}), we can write the noise figure as
\beq
NF=\frac{[Z_s(f_{LO})+2R_{sw}]^2}{Z_s(f_{LO})^2}
\sum_{k=0}^{+\infty} \frac{1}{(2k+1)^2} \left[ \frac {Z_s[(2k+1)f_{LO}]^2+2R_{sw}R_s} {\{Z_s[(2k+1)f_{LO}]+2R_{sw}\}^2} \right].
\label{nfgeneral}
\eeq
Note that (\ref{nfgeneral}) simplifies to (\ref{nftot}) if $Z_s=R_s$. If $Z_s$ is an RLC tank resonating at $f_{LO}$ with a parallel resistor $R_s$, we
can neglect the tank impedance at the harmonic frequencies with respect to $R_{sw}$. Thus, (\ref{nfgeneral}) simplifies to 
\beq
NF|_{RLC}=1+\frac{2R_{sw}}{R_s}+(\frac{\pi^2}{8}-1)\frac{(R_S+2R_{sw})^2}{2R_{sw}R_s}.
\label{nfrlc}
\eeq
Note that if we do not neglect the tank impedance at the harmonics, the noise figure would be better than the one in (\ref{nfrlc}).
An interesting point is that reducing $R_{sw}$ does not necessarily reduce the noise figure. In fact, with an RLC tank as the source resistance, the
optimum switch resistance is
\beq
R_{sw, opt}=\frac{\sqrt{\pi^2-8}}{2\pi}R_s \approx 0.218R_s.
\label{rswopt}
\eeq
Plugging this optimum value of $R_{sw}$ in (\ref{nfrlc}) derives the minimum noise figure of the mixer as $2.54 \approx 4.05$ dB.



\section{Transformer-Mixer Design}
Now that we know how to analyze the 25\% duty cycle mixers, we get back to the design of our receiver front end. The transformer is designed and
optimized first to have maximum voltage gain and low power loss at the 5-GHz band. Then, looking through the secondary of the transformer, we build
the Norton equivalent circuit of the antenna-transformer cascade over a wide bandwidth as shown in Fig. \ref{fig:mixer}.
\begin{figure}[htb!]
\centering
\includegraphics[scale=1]{FIGS/CH4/Mixer.ps}
\caption{Transformer-mixer interface.}
\label{fig:mixer}
\end{figure}
Our transformer guarantees that if a load impedance of 800 $\Omega$ is attached to its secondary, then the input impedance seen from the primary is about
50 $\Omega$.  In other words, since $Z_s$ is equal to 800 $\Omega$ at the carrier frequency, for proper matching the input impedance of the mixer needs to be
800 $\Omega$ too. However, we know that for current-driven mixers, the input impedance depends on the source impedance itself. To avoid confusion, we
refer to the impedance seen by $I_{in}$ denoted by $Z_{in}$. Since $I_{in}$ is ideal, this resistance is unique and independent of $I_{in}$. Thus, the
combination of antenna-transformer and the mixers must have a secondary-referred input impedance of 400 $\Omega$. This composite input impedance is
calculated in (\ref{zinswc}) and repeated here as
\beq
Z_{in}(f)=Z_s||2R_{sw}+ (\frac{Z_s}{Z_s+2R_{sw}})^2 \div \displaystyle{\SumAll k \frac{1}{(4k+1)^2 [Z_s(f+4k f_{LO})+2R_{sw}]}}.
\label{zinswcrepeat}
\eeq
Due to the bandpass nature of $Z_s$, the summation on the right-hand side must be carried out for about 14 terms. Ideally, in the range of 5 to 6 GHz,
we must have $Re\{Z_{in}(f)\} \approx Z_s(f)/2 \approx 400$ $\Omega$ and $Im\{Z_{in}(f)\} \approx 0$. Thus, Eq. (\ref{zinswcrepeat})
yields $R_{sw}=57~\Omega$ corresponding to  $W/L=$10 $\mu$m/ 60 nm for the switches. The LO buffers driving eight such switches draw a total power of
$fCV_{DD}^2 \approx$ 0.4 mW at 6 GHz.
If $R_{sw} \ll Z_{in}(f)$, Eq. (\ref{zinswcrepeat}) can further be simplified to
\beq
\frac{1}{Z_{in}(f)}=\SumAll k \frac{1}{(4k+1)^2[Z_s(f+4kf_{LO})+R_{sw}]}.
\label{zinsimple}
\eeq
Equation \ref{zinsimple} reveals that the source impedance, $Z_s$, around integer multiples of $4 f_{LO}$ in series with the switch resistance
is scaled and put in parallel to form the composite input impedance, $Z_{in}$. In other words, since the IF port of the mixer has high impedance, the
impedance matching is achieved by folding of the source impedance itself.



Simulations indicate that the ``zero-power'' RF front end consisting of the transformer and the mixers exhibits a noise figure of 4.5 dB, a voltage
gain of 12 dB, an input $P_{1dB}$ of $-$5.2 dBm, and an $S_{11}$ of better than $-$12 dB across the 5 GHz band. For a target receiver NF of less than
6 dB, all of the subsequent stages must contribute no more than 1.5 dB, demanding additional circuit techniques to build low-noise yet linear baseband
filters with low power dissipation.

\section{Filter Design}
In the 11a standard, for the lowest data rate of 6 Mb/s, the adjacent and alternate adjacent channels can be higher than the desired channel by 16 dB
and 32 dB, respectively. For the highest data rate of 54 Mb/s, the maximum interferer power levels are reduced by 17 dB, relaxing the filtering
requirements. However, we design the baseband filters for the worst case which requires a sharp roll-off to reduce these channels to well below the
desired signal level $-$ unless the baseband ADCs offer a dynamic range wide enough and a sampling rate high enough to handle partially-attenuated
blockers. 
Because in OFDM, each subcarrier has a narrow bandwidth, the phase response can be assumed linear across each sub-channel. Thus, the phase response
of the filter is not critical suggesting that an elliptic filter implementation is acceptable. A fifth-order elliptic filter is sufficient for our
purpose and can be realized by cascading two biquadratic transfer functions and a single RC pole. Let us focus on designing a biquad section. 

\begin{figure}[htb!]
\centering
\includegraphics[scale=1]{FIGS/CH4/Biquad.ps}
\caption{(a) Conventional implementation of a biquad, (b) basic idea of noninvasive filtering, and (c) noninvasive implementation of a biquad.}
\label{fig:biquad}
\end{figure}
Figure \ref{fig:biquad}(a) shows a conventional implementation of such a transfer function using three $G_m$ stages.  
In order to study the noise behavior, we assume that each $G_m$ has an input-referred noise voltage and calculate its transfer function to the output.
The transfer function for the noise of $G_{m1}$ and $G_{m2}$ is proportional to $(s^2+\omega_z^2)/(s^2+as+b)$ and $1/(s^2+as+b)$, respectively, which
are both low-pass. Therefore, their noise is not attenuated in the signal band.
The noise of $G_{m3}$, however, is attenuated through the bandpass shaping function, $s/(s^2+as+b)$. 
The circuit also experiences nonlinearity because the signal and interferer are both amplified by $G_{m1}R_1$. As a result, $G_{m1}$ compresses at its
output and $G_{m2}$ compresses at its input. Another issue is that this architecture needs 4-input $G_m$ cells for fully-differential implementation.
All of these issues are mitigated in the noninvasive filtering architecture. 
The basic idea is illustrated in Fig. \ref{fig:biquad}(b), where a notch impedance similar to an LC trap is placed at the output of a $G_m$ stage
\cite{Zolfaghari}. The notch impedance, Z, is high in the signal band and low at the interferer frequency. The filter transfer function follows the
shape of Z and shunts the interferer to ground. 
The actual implementation is shown in Fig. \ref{fig:biquad}(c) where $G_{m3}$ and $G_{m4}$ form a gyrator that transforms $C_3$ to an emulated inductor,
$L_e$. $L_e$ resonates with $C_2$ and generates a low-impedance path to ground at the interferer frequency. 
Let us study the noise behavior. The transfer function for the noise of $G_{m1}$ is proportional to $(s^2+\omega_z^2)/(s^2+as+b)$ and does not
attenuate it in the signal band. But, the noise of $G_{m2}$ and $G_{m4}$ is highpass-filtered by $s^2/(s^2+as+b)$ and can be made negligible. The
noise of $G_{m3}$ is also attenuated by the bandpass shaping function of $s/(s^2+as+b)$. 
In summary, compared to the previous case which had two unattenuated noise sources, this case has only one, and we expect to see lower noise.  
This topology is also more linear since the signal is amplified by $G_{m1}R_1$, but the interferer is attenuated. As a result, $G_{m1}$ does not
compress at its output. Note that the interferer is amplified at the output of $G_{m2}$. This can cause $G_{m2}$ and $G_{m4}$ to compress at their
outputs and $G_{m3}$ at its input. The important point here is that the compression of $G_{m2}$, $G_{m3}$ and $G_{m4}$ only shifts the notch frequency
but it does not affect the transfer function in the signal band.\footnote{There is a second-order effect that slightly changes the gain in the signal
band and is verified by the measurement results in Section 4.7.} Because all the $G_m$ cells have one grounded input, this architecture readily lends
itself to differential implementation.




\begin{figure}[htb!]
\centering
\includegraphics[scale=1]{FIGS/CH4/Filter.ps}
\special{psfile=FIGS/Filter hoffset=-50 voffset=0 vscale=82 hscale=82}
\caption{(a) Fourth-order elliptic low-pass filter, and (b) $G_m$ implementations.}
\label{fig:filter}
\end{figure}
Figure \ref{fig:filter}(a) shows the realization of the fourth-order elliptic filter.  A first-order RC filter is then added on the PCB to reach the
fifth order. In this design, $G_{m1}$ and $G_{m2}$ are made variable to provide gain control to accommodate the wide range of input powers specified
by 11a. Figure \ref{fig:filter}(b) summarizes the $G_m$ implementations. $G_{m1}$ has a PMOS input for low flicker noise and to work
properly with an input common-mode level of 400 mV. Gain control is achieved by changing the source degeneration resistance of the input pair. Each
branch dissipates 720 uA from a 1-V supply.  $G_{m2}$ has an NMOS input and consumes less power as its noise is not as important as that of
$G_{m1}$. $G_{m3}$ to $G_{m8}$ are identical simple differential pairs.  The fourth-order filter exhibits an input-referred noise voltage of 2
n${V}/\sqrt{Hz}$ at 5 MHz, an in-channel $IIP_3$ of 193 m${V}_{rms}$ and a voltage gain of 39 dB while consuming 4.3 mW. The filter
voltage gain is programmable in steps of 2 to 3 dB for a total range of 43 dB.


\section{Experimental Results}
The receiver of Fig. \ref{fig:architecture} has been fabricated in 65-nm digital CMOS technology. Figure \ref{fig:die} shows the die photograph.
\begin{figure}[htb!]
\centering
\includegraphics[scale=1]{FIGS/CH4/die.eps}
\caption{Die photograph.}
\label{fig:die}
\end{figure}
The RF section occupies 350 $\mu$m $\times$ 240 $\mu$m and the baseband section 450 $\mu$m $\times$ 220 $\mu$m.\footnote{Due to limited silicon area,
the receiver layout is decomposed and placed within other unrelated circuits, but all of the connections are present on the chip.}
The die is bonded to a printed-circuit board and uses a 1-V supply for the main circuits, a 1.2-V source for the open-drain output drivers, and a
400-mV supply connected to the center tap of the transformer.



Figure \ref{fig:nf} plots the measured noise figure of the complete receiver as a function of the baseband frequency. The average noise figure is about
5.3 dB.
\begin{figure}[htb!]
\centering
\includegraphics[scale=0.6]{FIGS/CH4/nf.eps}
\caption{Measured noise figure.}
\label{fig:nf}
\end{figure}
The sensitivity of the receiver is measured with the aid of Agilent's N5182 MXG vector signal generator and N9020A MXA signal analyzer, which
respectively apply a 64-QAM 802.11a signal and sense the baseband outputs to construct the signal constellation. Figure \ref{fig:constel} shows the
results for a $-65$-dBm 5.7-GHz input at 54 MB/s.
\begin{figure}[htb!]
\centering
\includegraphics[scale=0.42]{FIGS/CH4/65dBm3_0002.eps}
\caption{Measured EVM at $P_{in}=-65$ dBm.}
\label{fig:constel}
\end{figure}
The error vector magnitude (EVM) is equal to $-28$ dB, exceeding the 11a specification by 5 dB, suggesting that the receiver sensitivity would be 5 dB
better. As expected, the sensitivity was measured to be $-$70 dBm with an EVM of $-$23.4 dB.

Figure \ref{fig:s11} plots the $S_{11}$ from 5 to 6 GHz, measured at each input frequency, while the mixers switch at the corresponding LO frequency.
\begin{figure}[htb!]
\centering
\includegraphics[scale=0.6]{FIGS/CH4/s11_2.eps}
\caption{Measured input return loss.}
\label{fig:s11}
\end{figure}
It is expected that a slightly larger transformer or adding more capacitance can yield $S_{11}<-10$ dB across the band.
\begin{figure}[htb!]
\centering
\includegraphics[scale=0.6]{FIGS/CH4/sensitivity.eps}
\caption{Measured sensitivity.}
\label{fig:sensitivity}
\end{figure}
Figure \ref{fig:sensitivity} shows the sensitivity of the receiver from 5 to 6 GHz. The sensitivity degrades a little at lower frequencies again because the
transformer is mistuned. 
\begin{figure}[htb!]
\centering
\includegraphics[scale=0.6]{FIGS/CH4/gainvsf.eps}
\caption{Measured passband gain.}
\label{fig:gainvsf}
\end{figure}
The receiver gain across the band in Fig. \ref{fig:gainvsf} also confirms the transformer mistuning.

\begin{figure}[htb!]
\centering
\includegraphics[scale=0.6]{FIGS/CH4/gain.eps}
\caption{Measured receiver transfer function.}
\label{fig:tf}
\end{figure}
Figure \ref{fig:tf} plots the measured receiver transfer function, revealing a passband peaking of 1 dB and a rejection of 22 dB at 20 MHz and 43 dB at
40 MHz.\footnote{In this measurement a first-order RC section follows each output on the PCB.}
Owing to the finite output resistance of the $G_m$ cells, the filter does not exhibit the deep notches that are characteristic of elliptic transfer
functions.
The performance of the baseband filter is ultimately tested when a large blocker accompanies a small desired signal. In such a case, the filter must
remain sufficiently selective and linear so that the desired signal does not experience compression. Figure \ref{fig:gainblocker} plots the measured
passband gain as a function of the power of an RF blocker in the adjacent or alternate adjacent channel.
\begin{figure}[htb!]
\centering
\includegraphics[scale=0.6]{FIGS/CH4/gainblocker.eps}
\caption{Measured passband gain in the presence of a blocker.}
\label{fig:gainblocker}
\end{figure}
Note that this measurement is done with the maximum receiver gain which corresponds to the signal levels around the sensitivity, e.g. $-$60 dBm. The
maximum adjacent channel interferer would be 1 dB lower or $-$61 dBm for which the variation of gain is negligible. Even with the maximum 11a power of
$-$30 dBm, the gain variation is only about 1 dB. The peaking of gain with large adjacent-channel interferer is also observed in simulations and is
due to the compression in the notch-impedance $G_m$ cells. For the alternate-adjacent-channel interferer the gain variation is much less and totally
negligible even upto $-$30 dBm. 

The filter nonlinearity resulting from a blocker may also corrupt the 11a 64-QAM OFDM signal by creating cross modulation among the sub-channels.
This effect is characterized by setting the RF input signal level 3 dB above the sensitivity, applying a blocker, and raising its level until the EVM
falls to $-$23 dB. Figure \ref{fig:evmint} plots the relative blocker level as a function of the frequency offset with respect to the desired signal
center frequency.
\begin{figure}[htb!]
\centering
\includegraphics[scale=0.6]{FIGS/CH4/interferer_rejection.eps}
\caption{Measured interferer rejection.}
\label{fig:evmint}
\end{figure}

Figure \ref{fig:evmpin} shows the measured EVM and the corresponding passband gain versus input power. As the input signal power increases, a lower gain
is chosen for the variable-gain $G_m$ cells. For small input powers EVM is noise-limited and improves with the input power, while at large input powers
EVM is nonlinearity-limited and will eventually go up. The variable gain range is such that even with maximum 11a signal of $-$30 dBm, the EVM is not
degraded due to nonlinearity.
\begin{figure}[htb!]
\centering
\includegraphics[scale=0.6]{FIGS/CH4/evm_AGC.eps}
\caption{Measured EVM and passband gain versus input power.}
\label{fig:evmpin}
\end{figure}

Table \ref{table:priorart} summarizes the receiver performance and compares it to that of prior art. This work has reduced the power consumption by about a factor of 4,
while demonstrating a sensitivity of $-$70 dBm. 
\begin{table}[htb]
\caption{Comparison with state-of-the-art.}
\vspace{0.1in}
\centering
\includegraphics[scale=1]{FIGS/CH4/table.ps}
\label{table:priorart}
\end{table}

\section{Derivation of Input Impedance with Finite Source Impedance}
Using Eq. (\ref{vinf1}), we can find $b_1$ as 
\beq
b_1=2 \left(\frac{\sqrt{2}}{\pi}\right)^2 \times 2Z_L(f-f_{LO}) \left[a_1+\frac{a_{-3}}{3}+\frac{a_{5}}{-5}+\frac{a_{-7}}{-7}+\frac{a_{9}}{9}+...\right].
\label{b1}
\eeq
Similarly, we can find $b_{-3}$ and $b_5$ as
\beq
b_{-3}=2 \left(\frac{\sqrt{2}}{\pi}\right)^2 \times \frac{1}{3}\times 2Z_L(f-f_{LO}) \left[a_1+\frac{a_{-3}}{3}+\frac{a_{5}}{-5}+\frac{a_{-7}}{-7}+\frac{a_{9}}{9}+...\right],
\label{b-3}
\eeq
\beq
b_{5}=2 \left(\frac{\sqrt{2}}{\pi}\right)^2 \times \frac{1}{-5}\times 2Z_L(f-f_{LO}) \left[a_1+\frac{a_{-3}}{3}+\frac{a_{5}}{-5}+\frac{a_{-7}}{-7}+\frac{a_{9}}{9}+...\right].
\label{b5}
\eeq
Note that $b_{-3}=b_1/3$, $b_{5}=b_1/(-5)$, $b_{-7}=b_1/(-7)$, etc.
Another sets of equation comes from (\ref{imix}) as
\beq
a_1=1-\frac{b_1}{Z_s(f)},
\label{a1}
\eeq
\beq
a_{-3}=-\frac{b_{-3}}{Z_s(f-4f_{LO})},
\label{a-3}
\eeq
\beq
a_{5}=-\frac{b_{5}}{Z_s(f+4f_{LO})}.
\label{a5}
\eeq
%\beq
%a_{-7}=-\frac{b_{-7}}{Z_s(f-8f_{LO})}.
%\label{a-7}
%\eeq
If we substitute $b_1$, $b_{-3}$, and $b_{5}$, we get
\beq
a_1=1-\frac{1}{Z_s(f)} \frac{8}{\pi^2}Z_L(f-f_{LO})\left[a_1+\frac{a_{-3}}{3}+\frac{a_{5}}{-5}+\frac{a_{-7}}{-7}+\frac{a_{9}}{9}+...\right],
\label{a12}
\eeq
\beq
a_{-3}=-\frac{1}{3Z_s(f-4f_{LO})} \frac{8}{\pi^2}Z_L(f-f_{LO})\left[a_1+\frac{a_{-3}}{3}+\frac{a_{5}}{-5}+\frac{a_{-7}}{-7}+\frac{a_{9}}{9}+...\right],
\label{a-32}
\eeq
\beq
a_{5}=-\frac{1}{-5Z_s(f+4f_{LO})} \frac{8}{\pi^2}Z_L(f-f_{LO})\left[a_1+\frac{a_{-3}}{3}+\frac{a_{5}}{-5}+\frac{a_{-7}}{-7}+\frac{a_{9}}{9}+...\right].
\label{a52}
\eeq 
Then, we can write $a_{-3}$ and $a_{5}$ as
\beq
a_{-3}=(a_1-1)\frac{Z_s(f)}{3Z_s(f-4f_{LO})},
\label{a-33}
\eeq
\beq
a_{5}=(a_1-1)\frac{Z_s(f)}{-5Z_s(f-4f_{LO})}.
\label{a53}
\eeq
Substituting the recent values for $a_{-3}$ and $a_{5}$ in (\ref{a12}), we get
\ber
a_1&=&1-\frac{1}{Z_s(f)} \frac{8}{\pi^2}Z_L(f-f_{LO}) \times \nonumber\\
&&\!\!\!\!\!\!\!\!\!\!\!\!\!\!\!\!\!\!\!\!\!\!\!\! \left[1+(a_1-1)\frac{Z_s(f)}{Z_s(f)}+(a_1-1)\frac{Z_s(f)}{9Z_s(f-4f_{LO})}+(a_1-1)\frac{Z_s(f)}{25Z_s(f+4f_{LO})}+...\right].\nonumber\\
\label{a13}
\eer
It follows that
\beq
(a_1-1)\left[1+\frac{8}{\pi^2}Z_L(f-f_{LO})\displaystyle{\SumAll k \frac{1}{(4k+1)^2 Z_s(f+4k f_{LO})}} \right]=-\frac{8}{\pi^2}\frac{Z_L(f-f_{LO})}{Z_s(f)}.
\label{a14}
\eeq
Note that the input impedance is equal to $b_1$ and from (\ref{a1}) we get $Z_{in}=-(a_1-1){Z_s(f)}$. Therefore, from (\ref{a14}) we write the input impedance
as
\beq
Z_{in}(f)=\frac{\frac{8}{\pi^2} Z_L(f-f_{LO})}{1+\frac{8}{\pi^2} Z_L(f-f_{LO})\displaystyle{\SumAll k
\frac{1}{(4k+1)^2 Z_s(f+4k f_{LO})}}}.
\label{zincalc}
\eeq


\section{Conclusion}
This chapter suggests the use of transformers in place of active LNAs to save power and provide ESD protection. The ``zero-power'' front end consisting
of a transformer followed by passive mixers has reasonable performance and combined with noninvasive filtering, exceeds the 11a requirements while
consuming only 11.6 mW.
The proposed analysis of current-driven passive mixers provides insight into their properties.


\chapter{Future Work}

This dissertation introduced three novel analyses that helps the designer understand the circuits better and optimize their performance. In addition to optimization, it
helps us make changes and propose new circuits. For example, the PFD phase noise analysis suggests to use various digital gates in the PFD circuit rather than all-NAND or
all-NOR topologies. Another possible research path is to find the relation between the phase noise of LC oscillator and that of a single common-source amplifier with an
RLC tank as the load. That might give us more insight and guidelines on the LC oscillator design. Finally, the analysis of current-driven mixers allows us to optimize the
mixers for minimum noise figure or may suggest circuit modifications to create new topologies that exhibit lower noise or higher conversion gain or better harmonic
rejection.

%\input {ch6}


%\bibliography {bib/network,bib/naming}    % bibliography references
%\bibliographystyle {thesis}

\bibliographystyle {IEEEtran}
\bibliography {IEEEfull,thesisBib} 


\end {document}

