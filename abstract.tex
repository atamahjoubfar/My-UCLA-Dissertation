\abstract {
High-throughput real-time optical sensing and imaging instruments for capture and analysis of fast phenomena are among the most essential tools for scientific, industrial, military, and most importantly biomedical applications. The key challenge in these instruments is the fundamental trade-off between speed and sensitivity of the measurement system due to the limited signal energy collected in each measurement window. Based on two enabling technologies, namely photonic time-stretch dispersive Fourier transform and optical amplification, we developed several novel high-throughput optical measurement tools for applications such as flow cytometry, vibrometry, and volumetric scanning.

We demonstrated optical Raman amplification at about 800 nm wavelength for the first time and extended time-stretch dispersive Fourier transform to this region of electromagnetic spectrum. We used this enabling technology to make an ultrafast three-dimensional laser scanner with about hundred thousand scans per second and an imaging vibrometer with nanometer-scale axial resolution. 

We also employed our high-speed laser scanner to perform label-free cell screening in flow. One of the fundamental challenges in cell analysis is the undesirable impact of cell labeling on cellular behavior. To eliminate the need for these labels, while keeping the cell classification accuracy high, additional label-free parameters such as precise measurement of the cell protein density is required. We introduced a high-accuracy label-free imaging flow cytometer based on simultaneous measurement of morphology and optical path length through the cell at flow speeds as high as a few meters per second. 

Finally, the ultimate challenge in the ultra-high-throughput instrumentation is the storage and analysis of the torrent of data generated. As an example, our imaging flow cytometer generates about ten terabytes of cell images over a course of one hour acquisition, which captures images of every single cell in more than two milliliters of sample e.g. blood. We enabled practical use of these big data volumes by efficient combination of analog preprocessing techniques such as quadrature demodulation with parallel storage and digital post-processing.
}
