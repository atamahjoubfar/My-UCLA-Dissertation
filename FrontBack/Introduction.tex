\chapter{Introduction}
\label{chp:INTRO_Chapter}

Optical instruments are widely used in scientific, industrial, military, and biomedical applications that high-speed non-invasive measurements are required. However, as the speed of measurement increases, less photons can be collected in each scan, and the signal energy drops. This leads to a reduction in the signal-to-noise ratio of the measurement, which ultimately limits the resolution and sensitivity of the sensing or imaging application. One way to collect more photons is to increase the intensity of the illumination or the interrogation light, but this is often undesirable in biological applications because the biological samples can easily get damaged by the intense light, especially when an objective lens is focusing the light on the specimen. In this dissertation, we rely on the power of two unique technologies, photonic time-stretch dispersive Fourier transform and optical amplification, to develop several novel high-throughput optical sensing and imaging instruments.

In Chapter \ref{chp:APL2011_Chapter}, we combine photonic time-stretch dispersive Fourier transform imaging and interferometry to form an ultra-high-speed imaging vibrometer. We also take advantage of optical amplification in this vibrometer to achieve nanometer-scale axial resolution in measurement of surface vibrations with no need for a feedback stabilization mechanism.

In Chapter \ref{chp:PW2013_Chapter}, we introduce a scanning technique, which substantially improves the speed of laser scanners by performing an inertia-free scan through mapping of space to spectrum, and subsequently, to time. We also optically amplify the interrogation optical signals before photodetection to achieve superior sensitivity. By employing our scanning technology in one of the spatial dimensions of a three-dimensional laser scanner, we achieve extremely high volumetric scan rate of hundred thousand scans per second. We experimentally demonstrate numerous applications of this high-speed scanner for both surface and volumetric scans.

We also utilize our ultrafast imaging technique to perform label-free cell screening in flow. One of the fundamental challenges in cell analysis is the change in cellular behavior induced by the labels, which are used to mark cells for better identification. Labeled cells are usually not a good representative of their intact form and therefore, unfavorable for downstream studies. So, it is highly preferred to identify cells by using additional label-free parameters such as accurate measurement of the cell biophysical properties e.g. protein density. We use our high-speed imaging technique to capture quantitative phase and intensity images of suspended cells at flow speeds as high as a few meters per second. From these images, Optical refractive indices of the cells are extracted with high-accuracies, which correlate with their protein densities. In Chapter \ref{chp:BOE2013_Chapter}, we show that our imaging flow cytometer is capable of label-free cell classification at extremely high throughputs and accuracies. 

Chapter \ref{chp:JOSAA2013_Chapter} is a theoretical study of noise sources in optical Raman amplification. Our study shows that the Raman amplification can provide an approach to improvement of sensitivity of biological imaging and sensing systems. In Chapter \ref{chp:CLEO2010_Chapter}, we experimentally demonstrate this by optical Raman amplification at about 800 nm Stokes wavelength for the first time. We also extend the empowering time-stretch dispersive Fourier transform technology to this region of electromagnetic spectrum. 

At last, the fundamental challenge of storage and analysis of big data volumes, generated by extremely high-throughput instruments, is addressed in Chapter \ref{chp:CLEO2015_Chapter}. We explain our solution to this problem in the context of an example, our imaging flow cytometer data, which is about ten terabytes of cell images for about one hour of experiment. These big volume and velocity of data is essentially required to capture pictures of every single cell in more than two milliliters of sample e.g. blood. To enable the practical use of this data, we combine analog preprocessing techniques such as quadrature demodulation with parallel storage and digital post-processing methods.
