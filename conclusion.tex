\chapter{Conclusion and future work}
\label{chp:CONCLU_Chapter}

We proposed and demonstrated several optical measurement, sensing, and imaging instruments, specifically for biomedical applications. First, a high-speed line imaging-based vibrometer and velocimeter was introduced, which achieves nanometer-scale axial resolution without the need for mechanical scanning. We experimentally showed real-time line imaging of 30 KHz acoustic waves at a frame rate of 36.7 MHz. Each line image contains 1200 image pixels, and the probing time of each line image is only 30 ps, which is essential for freezing the motion and reducing the measurement ambiguity.

We also introduced a new high-speed laser scanning method. The speed of conventional laser scanners such as galvanometric mirrors and acousto-optic deflectors is usually not enough for many applications, resulting in motion blur. Therefore, they are not commonly used for capturing fast transient phenomena. Here, we presented a novel type of laser scanner that offers roughly three orders of magnitude higher scan rates than conventional methods. Our technique is based on inertia-free laser scanning by dispersing a train of broadband optical pulses both temporally and spatially i.e. each broadband pulse is temporally chirped by time-stretch dispersive Fourier transform and further formed into a serial rainbow in space by one or more diffractive elements such as prisms and gratings. In our proof-of-principle demonstration, we showed one-dimensional line scans at a record high scan rate of 91 MHz and two-dimensional raster scans and three dimensional volumetric scans at an unprecedented scan rate of 105 kHz. The axial resolution of the volumetric scan was in the order of a few nanometers. The method holds promise for a broad range of scientific, industrial, and biomedical applications.

We also demonstrated a novel method for label-free imaging flow cytometry based on interferometric serial time-encoded amplified microscopy. Our flow cytometer is capable of classifying cells based on their size, scattering, and protein concentration in flow rates as a high as a few meters per second. To do this, our cell analyzer measures size and total optical path difference of cells simultaneously and extracts the refractive index, which corresponds to the protein concentration of the cells, as an additional parameter for classification. Our experimental results clearly show the enhancement of the separation of OT-II T cell hybridoma from SW480 epithelium cancer cells by adopting the additional protein concentration parameter. To enable the practical use of this flow cytometer, we used quadrature demodulation for preprocessing of the analog photodetector output signal to reduce the required sampling rate of the analog-to-digital converter. This also reduces the size of digital data that is generated and makes the storage, transmission, and post processing of data easier. As a proof-of-principle demonstration, we showed an acquisition system capable of continuously recording 10.8 TB of phase and intensity images of the label-free imaging flow cytometer. This corresponds to images of every single cell in 2.7 mL of sample e.g. blood. 

Finally, we showed that an optical postamplifier such as a Raman amplifier can improve the sensitivity of a high-speed photodetection system in the visible to near-infrared spectral range. More specifically, we have demonstrated a sensitivity improvement of about 20 dB in the amplifier-enhanced detection system at Stokes wavelengths from 500 to 1000 nm using a 100 MHz bandwidth, which supports relatively high scan rates. This improvement is particularly valuable for high-speed detection in biomedical sensing and imaging applications, in which samples cannot tolerate high-intensity illumination. We also reported the first experimental demonstration of Raman amplification in a fiber at wavelengths near 800 nm. 

For future work, we recommend implementation of the flow-cytometer phase recovery algorithm on a field programmable gate array (FPGA). This will enable immediate classification of the cells, which is essential for cell sorting. We also suggest the interferometric architecture of the Coherent-STEAM setup for characterization of ultrafast optical pulses. 